%! TeX root = report.tex

\section{Spec 4c in SMT using Finite Sets \& Cardinalities}

In addition to \SpecFour{}, we have developed a manual encoding of the 3SF
protocol in SMT, directly utilizing the CVC5 solver~\cite{DBLP:conf/tacas/BarbosaBBKLMMMN22}.
In contrast to the \tlap{}-based specifications, we manually encode an
arbitrary state of the protocol
directly, using SMT-LIB constraints. At the cost of using a lower-level language and
requiring a specialized solver, this has the following advantages:
\begin{itemize}
  \item The manual encoding is more succinct than the SMT encoding produced by
    Apalache from \tlap{}.
  \item SMT-LIB supports recursive functions, which allows us to express the
    recursion inherent to the 3SF protocol more naturally.
\end{itemize}

Structurally, this specification combines some aspects of \SpecTwo{} and
\SpecThree{}: like \SpecTwo{}, it does not encode a state machine, but rather
an arbitrary state of the protocol. However, it uses data structures and
constraints similar to \SpecThree{} to model key components of the 3SF
protocol, including checkpoints, FFG votes, justification and finalization,
and slashing conditions.
To model finite sets and cardinalities, we use the non-standard SMT theory of
sets and cardinalities~\cite{DBLP:journals/lmcs/BansalBRT18} provided by CVC5.

\subsection{Modeling}
The SMT spec explicitly introduces hashes, checkpoints and nodes as atoms over
finite domains. Votes are modeled as any possible combination
of a source and target checkpoint and a sending node:

\begin{lstlisting}[language=smt]
(declare-datatype Hash ((Hash1) (Hash2) (Hash3)))
(declare-datatype Checkpoint ((C1) (C2) (C3) (C4) (C5)))
(declare-datatype Node ((Alice) (Bob) (Charlie) (David)))
(declare-datatype Vote ((Vote (source Checkpoint) (target Checkpoint) (sender Node))))
\end{lstlisting}

To remain within the decidable SMT fragment, we have to model unbounded data
using functions. For example, we model the slot number of a block as a function
from block hashes to integers:

\begin{lstlisting}[language=smt]
(declare-fun slot (Hash) Int)
; genesis' slot is 0
(assert (= (slot genesis) 0))
; slots are increasing from parent to child
(assert (forall ((h Hash)) (=> (not (= h genesis)) (> (slot h) (slot (parent_of h))))))
\end{lstlisting}

We encode all protocol rules as declarative constraints in the SMT model. For
example, the constraint that a checkpoint is justified if and only if there is
a supermajority of validators that cast a justifying vote from an already
justified checkpoint is encoded as follows:

\begin{lstlisting}[language=smt]
(declare-const justified_checkpoints (Set Checkpoint))
(assert (= justified_checkpoints (set.comprehension ((c Checkpoint))
  (or
    ; L3: genesis is justified
    (= c genesis_checkpoint)
    ; L4: there is a quorum of validators that cast a vote from a justified checkpoint to c
    (>= (* 3 (set.card (set.comprehension ((node Node))
        (exists ((vote Vote)) (and
          ; L4+5: vote is a valid vote cast by node
          (set.member vote votes)
          (= (sender vote) node)
          ; L6: the source of the vote is justified
          (set.member (source vote) justified_checkpoints)
          ; L7: there is a chain source.block ->* checkpoint.block ->* target.block
          (and
            (set.member
              (tuple (checkpoint_block (source vote)) (checkpoint_block c))
              ancestor_descendant_relationship)
            (set.member
              (tuple (checkpoint_block c) (checkpoint_block (target vote)))
              ancestor_descendant_relationship)
          )
          ; L8: the target checkpoint slot is the same as the checkpoint's
          (= (checkpoint_slot (target vote)) (checkpoint_slot c))
        ))
        node
      )))
      (* 2 N)
    )
  )
  c
)))
\end{lstlisting}

\subsection{Checking the Specification}

Similar to model checking with Apalache, we can use the CVC5 solver to find
examples of reachable protocol states, often within seconds. For example, to
find a finalized checkpoint besides the genesis checkpoint, we append the
following SMT-LIB script to our specification:

\begin{lstlisting}[language=smt]
; find a finalized checkpoint (besides genesis)
(assert (not (= finalized_checkpoints (set.singleton genesis_checkpoint))))
(check-sat)
(get-model)
\end{lstlisting}

Ultimately, we want to show that \textit{AccountableSafety} holds. To check that
this property holds with an SMT solver, we negate the property and check for
unsatisfiability:

\begin{lstlisting}[language=smt]
; there is a counterexample to AccountableSafety
(assert (and
  (exists ((block1 Hash) (block2 Hash))
    (and
      (set.member (tuple block1 block2) conflicting_blocks)
      (set.member block1 finalized_blocks)
      (set.member block2 finalized_blocks)
  ) )
  (< (* 3 (set.card slashable_nodes)) N)
))
(check-sat)
\end{lstlisting}

If the constraint above was satisfiable, we would have found a counterexample
to \textit{AccountableSafety}. Thus, \textit{AccountableSafety} holds if the
solver returns unsatisfiable.

\subsection{Experimental Results}

Similar to our experiments with Apalache, we instantiate the SMT model with
different numbers of blocks and checkpoints to verify
\textit{AccountableSafety} and evaluate the runtime of the solver.
Table~\ref{tab:smt} shows the results of our experiments.

\begin{table}
    \centering
    \begin{tabular}{rrrrrrrr}
        \tbh{blk} & \tbh{chk} & \tbh{Time} \\ \toprule
        3 & 5 & 8 min 11 sec \\
        4 & 5 & 22 min 00 sec \\
        5 & 5 & 1 h 40 min 19 sec \\ \midrule
        3 & 6 & 74 h 1 min 41 sec \\
        4 & 6 & timeout ($>80$ h) \\
        5 & 6 & timeout ($>80$ h) \\ \bottomrule
    \end{tabular}
    \caption{Model checking experiments with CVC5.\ \textbf{blk} is the number of blocks, \textbf{chk} is the number
      of checkpoints.}\label{tab:smt}
\end{table}

As we can see, similar to our \tlap{} experiments, the runtime of the SMT solver
grows exponentially with the number of blocks and checkpoints. The solver is
able to verify \textit{AccountableSafety} for up to 5 blocks and 5 checkpoints
or 3 blocks and 6 checkpoints, but it times out for larger instances.
