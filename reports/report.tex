\documentclass[12pt]{article}
\usepackage[english]{babel}
\usepackage[utf8]{inputenc}

\usepackage{hyperref}
\usepackage{booktabs}
\usepackage{subcaption}
\usepackage{tikz}
\usetikzlibrary{positioning}
\usetikzlibrary{arrows.meta,arrows}

\newcommand{\tth}[1]{\footnote{\textcolor{blue}{#1}}}
\newcommand{\tbh}[1]{\textsc{\textbf{#1}}}

\title{\vspace{-2em}Technical Report: Exploring Automatic Model-Checking of the Ethereum specification\footnote{%
		The specifications, scripts, and experimental results can be accessed at: \url{https://github.com/freespek/ssf-mc}}}
\author{%
    Igor Konnov\thanks{This work was supported by Ethereum Foundation
    under grant FY24--1535.}\\ \footnotesize Independent Researcher%
    \and
    Jure Kukovec\footnotemark[2] \\ \footnotesize Independent Researcher%
    \and
    Thomas Pani\footnotemark[2] \\ \footnotesize Independent Researcher%
    \and
    Roberto Saltini\thanks{This work was supported by Ethereum Foundation
    under grant FY24--1536 and was partially done while at Consensys.} \\
    \footnotesize Independent Researcher%
    \and
    Thanh Hai Tran\footnotemark[3] \\ \footnotesize Independent Researcher%
}
\date{}

\usepackage{mathpartir}
\usepackage{amsthm}
\usepackage{amsmath}
\usepackage{amssymb}
\usepackage{mathtools}

\newtheorem{theorem}{Theorem}[section]
\newtheorem{lemma}[theorem]{Lemma}
\newtheorem{corollary}[theorem]{Corollary}

\newcommand{\iteDef}[4]{
  #1 \coloneqq \left\{
\begin{array}{ll}
      #2 &; #3 \\
      #4 &; \text{otherwise}\\
\end{array} 
\right. 
}

\newcommand{\tlap}{$\textsc{TLA}^{+}$}

\newcommand{\nat}{\mathbb N_0}

\newcommand{\op}{\textsc{Op}}

\newcommand{\mop}{\textsc{mapOp}}
\newcommand{\mapg}{\textsc{mapG}}

\begin{document}

\maketitle

\begin{abstract}%
We investigate automated model-checking of the Ethereum specification, focusing
on the \emph{Accountable Safety} property of the 3SF consensus protocol. We
select 3SF due to its relevance and the unique challenges it poses for formal
verification. Our primary tools are~\tlap{} for specification and the Apalache
model checker for verification.

Our formalization builds on the executable Python specification of 3SF\@. To
begin, we manually translate this specification into~\tlap{}, revealing
significant combinatorial complexity in the definition of Accountable Safety.
To address these challenges, we introduce several layers of manual abstraction:
(1) replacing recursion with folds, (2) substituting abstract graphs with
integers, and (3) decomposing chain configurations.
To cross-validate our results, we develop alternative encodings in SMT (CVC5)
and Alloy.

Despite the inherent complexity, our results demonstrate that exhaustive
verification of Accountable Safety is feasible for small instances --
supporting up to 7 checkpoints and 24 validator votes. Moreover, no violations
of Accountable Safety are observed, even in slightly larger configurations.
Beyond these findings, our study highlights the importance of manual
abstraction and domain expertise in enhancing model-checking efficiency and
showcases the flexibility of~\tlap{} for managing intricate specifications.%
\end{abstract}%
\newpage%

\setcounter{tocdepth}{2}  % hide anything at or below \subsubsection
\tableofcontents

%! TeX root = report.tex

\section{Introduction}

\colorbox{yellow}{RS, TH: A short intro to the protocol}

\paragraph{}\colorbox{yellow}{TODO:}

\begin{itemize}
  \item what we want to model-check
  \item why \tlap{}
  \item introduce Apalache
\end{itemize}

\paragraph{Complexity of (model-checking) the protocol.} The 3SF protocol is
intricate, with a high degree of combinatorial complexity, making it challenging
to reason about. We have observed multiple layers of complexity in the protocol:
\begin{itemize}
  \item The Python specification considers all possible graphs over all proposed
    blocks. From graph theory~\cite{cayley1878theorem}, we know that the number
    of labelled rooted forests on $n$ vertices is ${(n+1)}^{n-1}$. (Observe that
    this number grows faster than the factorial~$n!$.) This is the number of
    possible block graphs that the model checker has to consider for $n$ blocks.
  \item The protocol introduces a directed graph of checkpoints (pairs $(b,n)$
    of a block $b$ and an integer $n$) \emph{on top} of the block graph.
    Validator-signed votes form a third labeled directed graph over pairs of
    checkpoints. In addition, all of these edges have to satisfy arithmetic
    constraints.
  \item Justified and finalized checkpoints introduce an inductive structure
    that the model checker has to reason about. Essentially, the solvers have to
    reason about chains of checkpoints on top of chains of blocks.
  \item Finally, the protocol introduces cardinalities, both for determining a
    quorum of validators and as a threshold for \textit{AccountableSafety}.
    Cardinalities are known to be a source of inefficiency in automated
    reasoning.
\end{itemize}

\subsection{Key Outcomes}

\colorbox{yellow}{TODO: summarize the key outcomes from Section~\ref{sec:discussion}}

\subsection{Overview of the report}

Figure~\ref{fig:artifacts} depicts the relations between the specifications
that we have produced in the project:

\begin{enumerate}
    \item We have started with the executable specification in Python.

    \item \textbf{Spec 1}: This is the specification
        \texttt{spec1-2/ffg\_recursive.tla}. It is the result of a manual
        mechanical translation of the original executable specification in
        Python, which can be found in \texttt{ffg.py}. This specification is
        using mutually recursive operators, which are not supported by
        Apalache. As a result, we are not checking this specification. This
        specification is the result of our work in Milestone 1.

    \item \textbf{Spec 2}: This is the specification \texttt{spec1-2/ffg.tla}. It
        is a manual adaptation of Spec 1. The main difference between Spec 2
        and Spec 1 is that Spec 2 uses ``folds'' (also known as ``reduce'')
        instead of recursion.

      \item \colorbox{yellow}{TBD}

\end{enumerate}

\begin{figure}
  %! TeX root = report.tex

\begin{tikzpicture}[node distance=2cm, >=latex]
    \tikzset{%
        mynode/.style={%
            draw, rectangle, minimum width=3cm, minimum height=1cm, align=center,
            rounded corners
        }
    }

    % Define the nodes
    \node[mynode, minimum width=2cm, minimum height=1cm] (py)
        {\small\texttt{ffg.py}};

    \node[mynode, minimum width=4cm,
        minimum height=1cm, below=1cm of py] (spec1)
        {\SpecOne{}: {\small\texttt{spec1-2/ffg\_recursive.tla}}};

    \node[mynode, minimum width=3cm,
        minimum height=1cm, right=1cm of spec1] (spec2)
        {\SpecTwo{}: {\small\texttt{spec1-2/ffg.tla}}};

    \node[mynode, minimum width=3cm,
        minimum height=1cm, below=1cm of spec2] (spec3)
        {\SpecThree{}: {\small\texttt{spec3/ffg.tla}}};

    \node[mynode, minimum width=3cm,
        minimum height=1cm, left=1cm of spec3] (spec4)
        {\SpecFour{}: {\small\texttt{spec4/ffg\_inductive.tla}}};

    \node[mynode, minimum width=3cm,
        minimum height=1cm, below=1cm of spec4] (spec4b)
        {\SpecFourB{}};

    \node[mynode, minimum width=3cm,
        minimum height=1cm, below left=1cm and -2cm of spec3] (spec3b)
        {\SpecThreeB{}: SMT};

    \node[mynode, minimum width=3cm,
        minimum height=1cm, below right=1cm and -2cm of spec3] (spec3c)
        {\SpecThreeC{}: Alloy/SAT };

    \draw[a] (py) -- (spec1);
    \draw[a] (spec1) -- (spec2);
    \draw[a] (spec2) -- (spec3);
    \draw[a] (spec3) -- (spec4);
    \draw[a] (spec3) -- (spec3b);
    \draw[a] (spec3) -- (spec3c);
    \draw[a] (spec4) -- (spec4b);

\end{tikzpicture}

  \caption{The relation between the specification artifacts}\label{fig:artifacts}
\end{figure}

Figure~\ref{fig:block-graphs} shows small graphs, most of which form one or
more chains. In the simple cases~\ref{fig:three}--\ref{fig:single}, we have one
or two chains that form a tree. In a more general case like in
Figure~\ref{fig:forest}, a graph is a forest. The graphs in
Figures~\ref{fig:tricky1}--\ref{fig:tricky2} do not represent chains.  The
graph~\texttt{I1} is a direct-acyclic graph but not a forest. The
graph~\texttt{I2} has a loop.

\begin{figure}
  %! TeX root = report.tex

% inside figure!
\centering
\begin{subfigure}{.35\textwidth}
  \centering
  \begin{tikzpicture}
    \tikzset{n/.style={circle, fill=black, inner sep=2pt}}

    \node[n] (n0) at (0,0) {};
    \node[n] (n1) at (1,.5) {};
    \node[n] (n1') at (1,-.5) {};
      
    \draw[a] (n1) -- (n0);
    \draw[a] (n1') -- (n0);
  \end{tikzpicture}

  \caption{Short fork [M3]}\label{fig:three}
\end{subfigure}
%
\begin{subfigure}{.3\textwidth}
  \centering
  \begin{tikzpicture}
    \tikzset{n/.style={circle, fill=black, inner sep=2pt}}

    \node[n] (n0) at (0,0) {};
    \node[n] (n1) at (1,.5) {};
    \node[n] (n2) at (2,.5) {};
    \node[n] (n1') at (1,-.5) {};
      
    \draw[a] (n1) -- (n0);
    \draw[a] (n2) -- (n1);
    \draw[a] (n1') -- (n0);
  \end{tikzpicture}

  \caption{Four blocks [M4a]}\label{fig:four-top}
\end{subfigure}
%
\begin{subfigure}{.3\textwidth}
  \centering
  \begin{tikzpicture}
    \tikzset{n/.style={circle, fill=black, inner sep=2pt}}

    \node[n] (n0) at (0,0) {};
    \node[n] (n1) at (1,.5) {};
    \node[n] (n1') at (1,-.5) {};
    \node[n] (n2') at (2,-.5) {};
      
    \draw[a] (n1) -- (n0);
    \draw[a] (n1') -- (n0);
    \draw[a] (n2') -- (n1');
  \end{tikzpicture}

  \caption{Four blocks [M4b]}\label{fig:four-bottom}
\end{subfigure}
%
\begin{subfigure}{.3\textwidth}
  \centering
  \begin{tikzpicture}
    \tikzset{n/.style={circle, fill=black, inner sep=2pt}}

    \node[n] (n0) at (0,0) {};
    \node[n] (n1) at (1,.5) {};
    \node[n] (n2) at (2,.5) {};
    \node[n] (n1') at (1,-.5) {};
    \node[n] (n2') at (2,-.5) {};
      
    \draw[a] (n1) -- (n0);
    \draw[a] (n2) -- (n1);
    \draw[a] (n1') -- (n0);
    \draw[a] (n2') -- (n1');
  \end{tikzpicture}

  \caption{Five blocks [M5a]}\label{fig:five1}
\end{subfigure}
%
\begin{subfigure}{.3\textwidth}
  \centering
  \begin{tikzpicture}
    \tikzset{n/.style={circle, fill=black, inner sep=2pt}}

    \node[n] (n0) at (0,0) {};
    \node[n] (n1) at (1,0) {};
    \node[n] (n2) at (2,.5) {};
    \node[n] (n2') at (2,-.5) {};
      
    \draw[a] (n1) -- (n0);
    \draw[a] (n2) -- (n1);
    \draw[a] (n2') -- (n1);
  \end{tikzpicture}

  \caption{Five blocks [M5b]}\label{fig:five2}
\end{subfigure}
%
\begin{subfigure}{.35\textwidth}
  \centering
  \begin{tikzpicture}
    \tikzset{n/.style={circle, fill=black, inner sep=2pt}}

    \node[n] (n0) at (0,0) {};
    \node[n] (n1) at (1,.5) {};
    \node[n] (n2) at (2,.5) {};
    \node[n] (n3) at (3,.5) {};
    \node[n] (n1') at (1,-.5) {};
    \node[n] (n2') at (2,-.5) {};
    \node[n] (n3') at (3,-.5) {};
      
    \draw[a] (n1) -- (n0);
    \draw[a] (n2) -- (n1);
    \draw[a] (n3) -- (n2);
    \draw[a] (n1') -- (n0);
    \draw[a] (n2') -- (n1');
    \draw[a] (n3') -- (n2');
  \end{tikzpicture}

  \caption{Seven blocks [M7]}\label{fig:seven1}
\end{subfigure}
%
\begin{subfigure}{.3\textwidth}
  \centering
  \begin{tikzpicture}
    \tikzset{n/.style={circle, fill=black, inner sep=2pt}}

    \node[n] (n0) at (0,0) {};
    \node[n] (n1) at (1,0) {};
    \node[n] (n2) at (2,0) {};
    \node[n] (n3) at (3,0) {};
    \node[n] (n4) at (4,0) {};
      
    \draw[a] (n1) -- (n0);
    \draw[a] (n2) -- (n1);
    \draw[a] (n3) -- (n2);
    \draw[a] (n4) -- (n3);
  \end{tikzpicture}

  \caption{Single chain}\label{fig:single}
\end{subfigure}
%
%
\begin{subfigure}{.35\textwidth}
  \centering
  \begin{tikzpicture}
    \tikzset{n/.style={circle, fill=black, inner sep=2pt}}

    \node[n] (n0) at (0,0) {};
    \node[n] (n1) at (1,0.5) {};
    \node[n] (n2) at (2,0.5) {};
    \node[n] (n3) at (3,0.5) {};
    \node[n] (n4) at (4,0.5) {};
    \node[n] (n5) at (1,-.5) {};
    \node[n] (n6) at (2,-.5) {};
    \node[n] (n7) at (3,-.5) {};
    \node[n] (n8) at (1,-1) {};
    \node[n] (n9) at (1,-1.5) {};
    \node[n] (n10) at (2,-1.5) {};
      
    \draw[a] (n1) -- (n0);
    \draw[a] (n2) -- (n1);
    \draw[a] (n3) -- (n2);
    \draw[a] (n4) -- (n3);
    \draw[a] (n5) -- (n0);
    \draw[a] (n6) -- (n5);
    \draw[a] (n7) -- (n6);

    \draw[a] (n10) -- (n9);
  \end{tikzpicture}

  \caption{Forest}\label{fig:forest}
\end{subfigure}
%
\begin{subfigure}{.3\textwidth}
  \centering
  \begin{tikzpicture}
    \tikzset{n/.style={circle, fill=black, inner sep=2pt}}

    \node[n] (n0) at (0,0) {};
    \node[n] (n1) at (1,0) {};
    \node[n] (n2) at (2,0) {};
    \node[n] (n3) at (3,0) {};
      
    \draw[a] (n1) -- (n0);
    \draw[a] (n2) -- (n1);
    \draw[a] (n3) -- (n2);
    \draw[a,bend right=45] (n3) to (n1);
  \end{tikzpicture}

  \caption{Impossible chains [I1]}\label{fig:tricky1}
\end{subfigure}
%
\begin{subfigure}{.3\textwidth}
  \centering
  \begin{tikzpicture}
    \tikzset{n/.style={circle, fill=black, inner sep=2pt}}

    \node[n] (n0) at (0,0) {};
    \node[n] (n1) at (1,0) {};
    \node[n] (n2) at (2,0) {};
    \node[n] (n3) at (3,0) {};
      
    \draw[a] (n1) -- (n0);
    \draw[a] (n2) -- (n1);
    \draw[a] (n3) -- (n2);
    \draw[a,bend right=45] (n1) to (n3);
  \end{tikzpicture}

  \caption{Impossible chains [I2]}\label{fig:tricky2}
\end{subfigure}


  \caption{Small instances of chains and non-chains}\label{fig:block-graphs}
\end{figure}



%! TeX root = report.tex

\section{Basic 3SF concepts}\label{sec:3sf}

In this section, we summarize the basic concepts of the 3SF protocol that this project depends on.
We refer the reader to the 3SF paper~\cite{d20243} for a more comprehensive explanation.

\paragraph*{Validators.} Participants of the protocol are referred to as \emph{validators} with $n$ being their total number.
Every validator is identified by a unique cryptographic identity and the public keys are common knowledge. 
Each validator has a \emph{stake} but for the purpose of this project each validator's stake is set to 1.
% If a validator $v_i$ misbehaves and a proof of this misbehavior is given, it loses a part of its stake ($v_i$ gets \emph{slashed}). 

\paragraph*{Slots.} Time is divided into \emph{slots}. 
In ideal condition a new block (see below) is expected to be proposed at the beginning of each slot.

\paragraph{Blocks and Chains.} 
A \emph{block} is a pair of elements, denoted as \( B = (b,p) \). Here, \( b \) represents the \emph{block body} -- essentially, the main content of the block which contains a batch of transactions grouped together.
Each block body contains a reference pointing to its \emph{parent} block. 
The second element of the pair, \( p \geq 0 \), indicates the \emph{slot} where the block \( B \) is proposed.
By definition, if $B_p$ is the parent of $B$, then $B_p.p < B.p$.
The \emph{genesis block} is the only block that does not have a parent {and has a negative slot}.
Given the definition above, each different block $B$ implicitly identifies a different finite \emph{chain} of blocks starting from block $B$, down to the genesis block, by recursively moving from a block to its parent.
Hence, there is no real distinction between a block and the chain that it identifies.
So, by chain $\chain$, we mean the chain identified by the block $\chain$.
We write $\chain_1 \preceq \chain_2$ to mean that $\chain_1$ is a non-strict prefix of $\chain_2$.
We say that $\chain_1$ \emph{conflicts} with $\chain_2$ if and only if neither $\chain_1 \preceq \chain_2$ nor $\chain_2 \preceq \chain_1$ holds. 

\paragraph{Checkpoints.}
In the protocol described in~\cite{d20243}, a \emph{checkpoint} is a tuple $(\chain, c)$, where \(\chain \) is a chain and \( c \) is a slot signifying where \( \chain \) is proposed for justification (this concept is introduced and explained below).
However, for efficiency reasons, in the specification targeted by this project, a \emph{valid} checkpoint $\C$ is a triple $(H, c, p)$ where $H$ is the hash of a chain $\chain$, $c$ is as per the definition above and $p =  \chain.p$.
The total pre-order among checkpoints is defined:
$\C \leq \C'$ if and only if either \(\C.c < \C'.c\) or, in the case where \(\C.c = \C'.c\), then \(\C.p \leq \C'.p\). 
Also, $\C < \C'$ means $\C \leq \C' \land \C \neq \C'$.


\paragraph*{FFG Votes.}
Validators cast two main types of votes: \textsc{ffg-vote}s and \textsc{vote}s. 
Each \textsc{vote} includes an \textsc{ffg-vote}.
The specification targeted by this project only deals with \textsc{ffg-vote}s as the extra information included in  \textsc{vote}s has no impact on AccountableSafety.
An \textsc{ffg-vote} is represented as a tuple \([\textsc{ffg-vote}, \C_1, \C_2, v_i]\), where {$v_i$ is the validator sending the \textsc{ffg-vote}\footnote{Digital signatures are employed to ensure that $v_i$ is the actual sender and it is assumed that such digital signatures are unforgeable.}, while} \(\C_1\) and \(\C_2\) are checkpoints.
These checkpoints are respectively referred to as the \emph{source} (\(\C_1\)) and the \emph{target} (\(\C_2\)) of the \textsc{ffg-vote}.
Such an \textsc{ffg-vote} is \emph{valid}
%(i.e., $\mathrm{valid}(\C_1 \to \C_2$)) 
if and only if both checkpoints are valid, 
\(\C_1.c < \C_2.c\) and  \(\C_1.\chain \preceq \C_2.\chain\). 
\textsc{ffg-vote}s effectively act as \emph{links} connecting the source and target checkpoints. Sometimes the whole \textsc{ffg-vote} is simply denoted as \(\C_1 \to \C_2\).

\paragraph*{Justification.}
A set of \textsc{ffg-vote}s is a \emph{supermajority set} if it contains valid \textsc{ffg-vote}s from at least \(\frac{2}{3}n\) distinct validators.
A checkpoint \(\C\) is considered \emph{justified} if it either corresponds to the genesis block, i.e., \(\C = (B_\text{genesis}, 0)\), or if there exists a supermajority set of links \(\{\C_i \to \C_j\}\) satisfying the following conditions. First,  for each link $\C_i \to \C_j$ in the set, {\(\C_i \to \C_j\) is valid and} \(\C_i.\chain \preceq \C.\chain \preceq \C_j.\chain\). Moreover, all source checkpoints \(\C_i\) in these links need to be already justified, and the checkpoint slot of \(\C_j\) needs to be the same as that of \(\C\) (\(\C_j.c=\C.c\)), for every \(j\). It is important to note that the source and target chain may vary across different votes.
% This justification rule is formalized by the binary function $\mathsf{J}(\V,\C)$ in Algorithm~\ref{alg:justification-finalization} which outputs \textsc{true} if and only if checkpoint $\C$ is justified according to the set of messages $\V$.
% Lastly, we say that a {chain \(\chain\)} \emph{is justified} if and only if there exists a justified checkpoint \(\C\) for which {\(\C.\chain = \chain\)}.

% \paragraph*{Greatest justified checkpoint and chain.}
% A checkpoint is considered justified in a view \(\V\) if \(\V\) contains a supermajority set of links justifying it. A \emph{justified} checkpoint which is maximal in \(\V\) with respect to the previously defined lexicographic ordering is referred to as the \emph{greatest justified checkpoint} in \(\V\), denoted as \(\GJ(\V)\). In the event of ties, they are broken arbitrarily. Chain \(\GJ(\V).\chain\) is referred to as the \emph{greatest justified chain}.

\paragraph*{Finality.}
A checkpoint \(\C\) is \emph{finalized} if it is justified and there exists a supermajority link with source \(\C\) and potentially different targets \(\C_i\) where \(\C_i.c = \C.c + 1\). A chain \(\chain\) is finalized if there exists a finalized checkpoint \(\C\) with \(\chain = \C.\chain\). The checkpoint \(\C = (B_\text{genesis}, 0)\) is finalized by definition.
% This finalization rule is formalized by the binary function $\mathsf{F}(\V,\C)$ in Algorithm~\ref{alg:justification-finalization} which outputs \textsc{true} if and only if checkpoint $\C$ is finalized according to the set of messages $\V$.
% {Given a view $\V$, a finalized checkpoint which is maximal in \(\V\) with respect to the previously defined lexicographic ordering is referred to as the \emph{greatest finalized checkpoint} in \(\V\), denoted as \(\GF(\V)\).
% In the event of ties, they are broken arbitrarily.
% % We say that a chain 
% We say that a chain $\chain$ is finalized according to a view $\V$ if and only if $\chain \preceq \GF(\V).\chain$.


\paragraph*{Slashing.}
A validator \(v_i\) is subject to slashing for sending two \emph{distinct} \textsc{ffg-vote}s \(\C_1 \to \C_2\) and \(\C_3 \to \C_4\) if either of the following conditions holds: {\(\mathbf{E_1}\) (Double voting)} if \(\C_2.c = \C_4.c\), implying that a validator must not cast distinct \textsc{ffg-vote}s for the same checkpoint slot; or {\(\mathbf{E_2}\) (Surround voting)} if \(\C_3 < \C_1\)
 and \(\C_2.c < \C_4.c\), indicating that a validator must not vote using a lower checkpoint as source and must avoid voting within the span of its other votes.


 \begin{definition}[AccountableSafety]
  \label{def:acc-safety}
  AccountableSafety holds if and only if, upon two conflicting chains being finalized, 
 {by having access to all messages sent,} it is possible to slash at least $\frac{n}{3}$ of the validators.
  % Finally, we also say that a chain is $f$-\emph{accountable} if the protocol outputting it has Accountable Safety with resilience $f$.
  % If a protocol $\Pi$ comprises $k$ subprotocols, $\Pi_i$, each of which outputs a chain, $\Chain_i$, we say that $\Chain_i$ is $f$-accountable if $\Pi_i$ is $f$-accountable.
\end{definition}

% Milestone 1 - Spec 1
%! TeX root = report.tex

\section{M1: Spec 1}

TODO\footnote{JK: explain how we translated things}


% Milestone 2 - Spec 2
%! TeX root = report.tex

\section{M2: Spec 2: Fold-Based Specification}

\SpecTwo{} addresses some of the limitations inherent in the straightforward
translation of \SpecOne{} from the Python executable specification.

\subsection{Translating Recursive \tlap{} Operators}

The primary goal of \SpecTwo{} is to maintain the semantic structure of
\SpecOne{} while eliminating recursion. In \SpecOne{}, (mutually) recursive
operators model key aspects of protocol behavior, such as the block tree and
block justification. Apalache does not natively support recursive
operators\footnote{\url{https://apalache-mc.org/docs/apalache/principles/recursive.html}},
thus it cannot be used immediately to model-check \SpecOne{}. While the
explicit-state \tlap{} model checker TLC supports recursive operators, it does
not scale to model-checking of this problem.

To resolve this, we reformulate \SpecOne{} into \SpecTwo{}, by substituting
(mutually) recursive constructs with bounded
\texttt{fold}~operations\footnote{In functional programming, \texttt{fold} is a
higher-order function that accepts a combining operation and an iterable data
structure, and applies the operation to each element of the data structure
to compute a single return value. \texttt{fold} is also known as
\texttt{reduce} in some languages.}, which enable the same iterative
computations to be performed in a non-recursive manner. Consider the following
example from \SpecOne{}:

\begin{lstlisting}[language=tla]
RECURSIVE is_ancestor_descendant_relationship(_, _, _)
is_ancestor_descendant_relationship(ancestor, descendant, node_state) ==
    IF ancestor = descendant THEN TRUE
    ELSE IF descendant = node_state.configuration.genesis THEN FALSE
    ELSE
        /\ has_parent(descendant, node_state)
        /\ is_ancestor_descendant_relationship(ancestor, get_parent(descendant, node_state), node_state)
\end{lstlisting}

Its corresponding fold-based formulation is shown below:

\begin{lstlisting}[language=tla]
is_ancestor_descendant_relationship(ancestor, descendant, node_state) ==
    LET FindAncestor(last_block_and_flag, slot) ==
        LET
            last_block == last_block_and_flag[1]
            flag == last_block_and_flag[2]
        IN
        IF flag THEN Pair(last_block, TRUE)
        ELSE IF last_block = node_state.configuration.genesis \/ ~has_parent(last_block, node_state) THEN Pair(last_block, FALSE)
        ELSE LET parent == get_parent(last_block, node_state) IN Pair(parent, parent = ancestor)
    IN
    ApaFoldSeqLeft(FindAncestor, Pair(descendant, descendant = ancestor), MkSeq(MAX_SLOT, LAMBDA i: i))[2]
\end{lstlisting}

\subsection{An Optimization: Flattening Nested Folds}

Initial model checking experiments with \SpecTwo{} revealed significant
challenges related to memory consumption, stemming from the high number of SMT
constraints emitted by Apalache for nested fold operations, which in turn mirror
the complexity of the original nested recursive structures from the Python
specification.

To address these issues, we introduce a manual optimization strategy that
involves flattening nested fold operations. This technique transforms nested
folds into a more manageable structure by employing additional \tlap{} state
variables, similar to memoization or prophecy variables.

For example, we introduce a new \tlap{} state variable
\texttt{PRECOMPUTED\_IS\_ANCESTOR\_DESCENDANT\_RELATIONSHIP} to store
precomputed ancestor-descendant relationships and initialize it with the results
of the fold operation above:

\begin{lstlisting}[language=tla]
LET all_blocks == get_all_blocks(single_node_state) IN
PRECOMPUTED__IS_ANCESTOR_DESCENDANT_RELATIONSHIP =
    [ descendant \in all_blocks |-> { ancestor \in all_blocks : is_ancestor_descendant_relationship(ancestor, descendant, single_node_state) } ]
\end{lstlisting}

Instead of re-evaluating the fold operation each time we need to check if two
blocks are in an ancestor-descendant relationship, we can directly access the
memoized result in a much more efficient map lookup:

\begin{lstlisting}[language=tla]
are_conflicting(chain1, chain2, node_state) ==
    /\ chain1 \notin PRECOMPUTED__is_ancestor_descendant_relationship[chain2]
    /\ chain2 \notin PRECOMPUTED__is_ancestor_descendant_relationship[chain1]
\end{lstlisting}

To further improve our confidence in the correctness of this optimization, we
could produce a proof in TLAPS or run Apalache to show functional equivalence.

\subsection{Checking the Specification}

We can query the specification for reachable protocol states using Apalache.
For example, we can check if a nontrivial finalized checkpoint exists by writing an
invariant that we expect not to hold. If the invariant below is violated,
Apalache will produce an example of a finalized checkpoint as a counterexample:

\begin{lstlisting}[language=tla]
\* Find a finalized checkpoint (in addition to the genesis checkpoint)
FinalizedCheckpoint_Example ==
    get_finalized_checkpoints(single_node_state) = { genesis_checkpoint(single_node_state) }
\end{lstlisting}

Obviously, we can also check \textit{AccountableSafety} by supplying it as an
invariant to Apalache:

\begin{lstlisting}[language=tla]
AccountableSafety ==
  LET
    finalized_checkpoints == get_finalized_checkpoints(single_node_state)
    finalized_blocks == { get_block_from_hash(checkpoint.block_hash, single_node_state) : checkpoint \in finalized_checkpoints }
    there_are_conflicting_finalized_blocks == \E block1, block2 \in finalized_blocks : are_conflicting(block1, block2, single_node_state)
    slashable_nodes == get_slashable_nodes(single_node_state.view_votes)
  IN there_are_conflicting_finalized_blocks => Cardinality(slashable_nodes) * 3 >= Cardinality(Nodes)
\end{lstlisting}

Table~\ref{tab:spec2} shows the results of model checking \SpecTwo{} with
Apalache. We can see that generating examples of reachable protocol states and
verifying \textit{AccountableSafety} is infeasible due to the high computational
complexity of the specification.

\begin{table}
    \centering
    \begin{tabular}{ll}
        \tbh{Property} & \tbh{Time} \\ \toprule
        Example: conflicting blocks & timeout ($>40$h) \\
        Example: finalized \& conflicting blocks & timeout ($>40$h) \\
        AccountableSafety & timeout ($>40$h) \\ \bottomrule
    \end{tabular}
    \caption{Model checking \SpecTwo{} with Apalache.}\label{tab:spec2}
\end{table}

These results are not surprising -- the solver has to consider both reachability
properties for all possible block graphs, and all possible FFG voting scenarios
on top of these graphs. To further evaluate \SpecTwo{}, we fix the block graph
-- this way the solver only has to reason about voting. We encode three example
block graphs: a single, linear chain (\texttt{SingleChain}), a minimal forked
chain of three blocks (\texttt{ShortFork}), and a forest of disconnected chains
(\texttt{Forest}). Table~\ref{tab:spec2_fixed} shows the results of model
checking \SpecTwo{} for these fixed block graphs.

\begin{table}
    \centering
    \begin{tabular}{llr}
      \tbh{Property} & \tbh{Block graph} & \tbh{Time} \\ \toprule
      Example: conflicting blocks & \texttt{SingleChain} & 1 min 3 sec \\
      Example: conflicting blocks & \texttt{ShortFork} & 52 sec \\
      Example: conflicting blocks & \texttt{Forest} & 2 min 21 sec \\ \midrule
      Example: fin.\ \& confl.\ blocks & \texttt{SingleChain} & 1 min 5
      sec \\
      Example: fin.\ \& confl.\ blocks & \texttt{ShortFork} & 10 hours
      49 min 47 sec \\
      Example: fin.\ \& confl.\ blocks & \texttt{Forest} & timeout
      ($>40$h) \\ \midrule
      AccountableSafety & \texttt{SingleChain} & 1 min 13 sec \\
      AccountableSafety & \texttt{ShortFork} & timeout ($>40$h) \\
      AccountableSafety & \texttt{Forest} & timeout ($>40$h) \\ \bottomrule
    \end{tabular}
    \caption{Model checking \SpecTwo{} for fixed block
    graphs.}\label{tab:spec2_fixed}
\end{table}

We can see that the solver can handle the single chain block graph (where
\textit{AccountableSafety} trivially holds due to absence of conflicting
blocks), but struggles with the more complex scenarios even when given a fixed
block graph. This suggests that the complexity inherent in the specification is
due to the high combinatorial complexity of voting scenarios, rather than just
the block graph.

\subsection{Discussion}~\footnote{\color{red}TODO: perhaps move this to a general
discussion section?} Applying translation rules to obtain checkable
specifications from other artifacts can provide a useful starting point.
However, such translation are also likely to introduce unwanted
inefficiencies. Translation rules cannot account for the nuances of the specific
context, which can lead to suboptimal performance. Consequently, leveraging
human ingenuity to develop specifications from the outset can be more effective.
When relying on translated specifications, it becomes crucial to conduct
manual optimizations to ensure efficiency.


% Milestone 4 - Specs 3-4

%! TeX root = report.tex

\section{Spec 3: Adding a state machine}\label{sec:spec3}

In the course of writing \texttt{Spec 2}, we realized
that the executable Python specification is essentially sequential. In other
words, even though the 3SF algorithm is distributed, its Python
specification as well as \texttt{Spec 1} and \texttt{Spec 2} are encoding
all possible protocol states in a single specification state.

\subsection{State machines in \tlap{}}

Since \tlap{} is designed for reasoning about state machines, Apalache is tuned
towards incremental model checking of the executions. For instance, if a state
machine is composed of $n$ kinds of state-transitions (called actions), that
is, $\mathit{Next} = A_1 \vee \dots \vee A_N$, the model checker tries to find
a violation to a state invariant~$\textit{Inv}$ by assuming that a single
symbolic transition $A_i$ took place. If there is no violation, that instance
of the invariant~$\textit{Inv}$ can be discarded. By doing so, the model
checker reduces the number of constraints for the SMT solver to process.  The
same applies to checking an inductive invariant. When there is no such
decomposition of $\mathit{Next}$, the model checker produces harder problems
for SMT\@.

\paragraph{Symbolic simulation.} In addition to model checking, Apalache offers
support for symbolic simulation. In this mode, the model checker
non-deterministically selects \emph{one} symbolic transition at each step and
applies it to the current state. While symbolic simulation is not exhaustive
like model checking, it allows for more efficient exploration of deep system
states. Importantly, any counterexample discovered through symbolic simulation
is just as valid as one found via exhaustive model checking.

\subsection{Introducing a state machine}

Having the observations above in mind, we introduce \texttt{Spec 3} in which we
specify a state machine that incrementally builds possible protocol states by
constructing the following data structures:

\begin{itemize}

    \item The set of proposed blocks, and the graph containing these blocks,
        called $\textit{blocks}$ and $\textit{graph}$, respectively.

    \item The ancestor-descendant relation, called
        $\textit{block\_graph\_closure}$.

    \item The announced FFG votes and the validators' votes on them, called
        $\textit{ffg\_votes}$ and $\textit{votes}$, respectively.

    \item The set of justified checkpoints that is computed as the greatest
        fixpoint, called $\textit{justified\_checkpoints}$.

\end{itemize}

The most essential part of the specification is shown in
Figure~\ref{fig:abstract-ffg-cast-votes}. It represents an abstract
state-transition that corresponds to a subset of validators sending votes for
the same FFG vote. The most interesting part of this transition can be seen in
the last lines: Instead of directly computing the set of justified checkpoints
in the current state, we just ``guess'' it and impose the required constraints
on this set. Mathematically, $\textit{justified\_checkpoints}$ is the greatest
fixpoint among the checkpoints that satisfy the predicate
$\textit{IsJustified}$.


\begin{figure}
    \centering
    \includegraphics[width=\textwidth]{images/abstract-ffg-cast-votes.pdf}  % Include the PDF file
    \caption{A state-transition that casts votes}\label{fig:abstract-ffg-cast-votes}
\end{figure}

Figure~\ref{fig:abstract-ffg-next} shows the transition predicate of the
specification. It just non-deterministically chooses the inputs to
$\textit{ProposeBlock}$ or $\textit{CastVotes}$ and fires one of those two
actions.

\begin{figure}
    \centering
    \includegraphics[width=\textwidth]{images/abstract-ffg-next.pdf}  % Include the PDF file
    \caption{The transition predicate of Spec 3}\label{fig:abstract-ffg-next}
\end{figure}

\subsection{Model checking experiments}

We have conducted model checking experiments with this specification. They are
summarized in Table~\ref{tab:abstract-ffg-mc}.

% 

\begin{table}
    \centering
    \begin{tabular}{llr}
        \tbh{State invariant}
            & \tbh{Command}
            & \tbh{Time}
            \\ \toprule
        $\textit{ExistsJustifiedNonGenesisInv}$
            & \texttt{check}
            & 5s
            \\ \midrule
        $\textit{ExistTwoFinalizedConflictingBlocks}$
            & \texttt{check}
            & 1h 27min
            \\ \midrule
        $\textit{AccountableSafety}$
            & \texttt{simulate}
            & N
            \\ \midrule
        $\textit{AccountableSafety}$
            & \texttt{check}
            & N
            \\ \bottomrule
    \end{tabular}
    \caption{Model checking experiments with Spec 3}\label{tab:abstract-ffg-mc}
\end{table}


%! TeX root = report.tex

\section{Spec 3b in SMT}\label{sec:smt}

In addition to \SpecThree{}, we have developed a manual encoding of the 3SF
protocol in SMT, directly utilizing the CVC5 solver~\cite{BarbosaBBKLMMMN22}.
In contrast to the \tlap{}-based specifications, we manually encode an
arbitrary state of the protocol
directly, using SMT-LIB constraints. At the cost of using a lower-level language and
requiring a specialized solver, this has the following advantages:
\begin{itemize}
  \item The manual encoding is more succinct than the SMT encoding produced by
    Apalache from \tlap{}.
  \item SMT-LIB supports recursive functions, which allows us to express the
    recursion inherent to the 3SF protocol more naturally.
\end{itemize}

Structurally, this specification combines some aspects of \SpecTwo{} and
\SpecThree{}: like \SpecTwo{}, it does not encode a state machine, but rather
an arbitrary state of the protocol. However, it uses data structures and
constraints similar to \SpecThree{} to model key components of the 3SF
protocol, including checkpoints, FFG votes, justification and finalization,
and slashing conditions.
To model finite sets and cardinalities, we use the non-standard SMT theory of
sets and cardinalities~\cite{DBLP:journals/lmcs/BansalBRT18} provided by CVC5.

\subsection{Modeling}
The SMT spec explicitly introduces hashes, checkpoints and nodes as atoms over
finite domains. Votes are modeled as any possible combination
of a source and target checkpoint and a sending node:

\begin{lstlisting}[language=smt]
(declare-datatype Hash ((Hash1) (Hash2) (Hash3)))
(declare-datatype Checkpoint ((C1) (C2) (C3) (C4) (C5)))
(declare-datatype Node ((Alice) (Bob) (Charlie) (David)))
(declare-datatype Vote ((Vote (source Checkpoint) (target Checkpoint) (sender Node))))
\end{lstlisting}

To remain within the decidable SMT fragment, we have to model unbounded data
using functions. For example, we model the slot number of a block as a function
from block hashes to integers:

\begin{lstlisting}[language=smt]
(declare-fun slot (Hash) Int)
; genesis' slot is 0
(assert (= (slot genesis) 0))
; slots are increasing from parent to child
(assert (forall ((h Hash)) (=> (not (= h genesis)) (> (slot h) (slot (parent_of h))))))
\end{lstlisting}

We encode all protocol rules as declarative constraints in the SMT model. For
example, the constraint that a checkpoint is justified if and only if there is
a supermajority of validators that cast a justifying vote from an already
justified checkpoint is encoded as follows:

\begin{lstlisting}[language=smt]
(declare-const justified_checkpoints (Set Checkpoint))
(assert (= justified_checkpoints (set.comprehension ((c Checkpoint))
  (or
    ; L3: genesis is justified
    (= c genesis_checkpoint)
    ; L4: there is a quorum of validators that cast a vote from a justified checkpoint to c
    (>= (* 3 (set.card (set.comprehension ((node Node))
        (exists ((vote Vote)) (and
          ; L4+5: vote is a valid vote cast by node
          (set.member vote votes)
          (= (sender vote) node)
          ; L6: the source of the vote is justified
          (set.member (source vote) justified_checkpoints)
          ; L7: there is a chain source.block ->* checkpoint.block ->* target.block
          (and
            (set.member
              (tuple (checkpoint_block (source vote)) (checkpoint_block c))
              ancestor_descendant_relationship)
            (set.member
              (tuple (checkpoint_block c) (checkpoint_block (target vote)))
              ancestor_descendant_relationship)
          )
          ; L8: the target checkpoint slot is the same as the checkpoint's
          (= (checkpoint_slot (target vote)) (checkpoint_slot c))
        ))
        node
      )))
      (* 2 N)
    )
  )
  c
)))
\end{lstlisting}

\subsection{Checking the Specification}

Similar to model checking with Apalache, we can use the CVC5 solver to find
examples of reachable protocol states, often within seconds. For example, to
find a finalized checkpoint besides the genesis checkpoint, we append the
following SMT-LIB script to our specification:

\begin{lstlisting}[language=smt]
; find a finalized checkpoint (besides genesis)
(assert (not (= finalized_checkpoints (set.singleton genesis_checkpoint))))
(check-sat)
(get-model)
\end{lstlisting}

Ultimately, we want to show that \textit{AccountableSafety} holds. To check that
this property holds with an SMT solver, we negate the property and check for
unsatisfiability:

\begin{lstlisting}[language=smt]
; there is a counterexample to AccountableSafety
(assert (and
  (exists ((block1 Hash) (block2 Hash))
    (and
      (set.member (tuple block1 block2) conflicting_blocks)
      (set.member block1 finalized_blocks)
      (set.member block2 finalized_blocks)
  ) )
  (< (* 3 (set.card slashable_nodes)) N)
))
(check-sat)
\end{lstlisting}

If the constraint above was satisfiable, we would have found a counterexample
to \textit{AccountableSafety}. Thus, \textit{AccountableSafety} holds if the
solver returns unsatisfiable.

\subsection{Experimental Results}

Similar to our experiments with Apalache, we instantiate the SMT model with
different numbers of blocks and checkpoints to verify
\textit{AccountableSafety} and evaluate the runtime of the solver.
Table~\ref{tab:smt} shows the results of our experiments.

\begin{table}
    \centering
    \begin{tabular}{rrrrrrrr}
        \tbh{blk} & \tbh{chk} & \tbh{Time} \\ \toprule
        3 & 5 & 8 min 11 sec \\
        4 & 5 & 22 min 00 sec \\
        5 & 5 & 1 h 40 min 19 sec \\ \midrule
        3 & 6 & 74 h 1 min 41 sec \\
        4 & 6 & timeout ($>80$ h) \\
        5 & 6 & timeout ($>80$ h) \\ \bottomrule
    \end{tabular}
    \caption{Model checking experiments with CVC5.\ \textbf{blk} is the number of blocks, \textbf{chk} is the number
      of checkpoints.}\label{tab:smt}
\end{table}

As we can see, similar to our \tlap{} experiments, the runtime of the SMT solver
grows exponentially with the number of blocks and checkpoints. The solver is
able to verify \textit{AccountableSafety} for up to 5 blocks and 5 checkpoints
or 3 blocks and 6 checkpoints, but it times out for larger instances.

%! TeX root = report.tex

\section{\SpecThreeB{} in Alloy}\label{sec:alloy}

Once we saw that our specifications~\SpecThree{} and~\SpecFour{}
were too challenging for the \tlap{} tools, we have decided to employ
Alloy~\cite{jackson2012software,alloytools}, as an alternative to our tooling.
Alloy has two features that are attractive for our project:

\begin{itemize}

  \item Alloy allows to precisely control the search scope by setting the
      number of objects of particular type in the universe. For example, we can
      restrict the number of checkpoints and votes to 5 and 12, respectively.
      Although we introduced similar restrictions with Apalache, Alloy has even
      finer level of parameter tuning.
      \rs{Is there some inherent limitation of Apalache that we should discuss?}

  \item Alloy translates the model checking problem to a Boolean satisfiability
      problem (SAT), in constrast to Apalache, which translates it to satisfiability-modulo-theory (SMT).  This
      allows us to run the latest off-the-shelf solvers such as Kissat, the
      winner of the SAT Competition 2024~\cite{SAT-Competition-2024-solvers}.

\end{itemize}

\subsection{From \tlap{} to Alloy}

Having~\SpecThree{} in~\tlap{}, it was relatively easy for us to write down
the Alloy specification, as~\SpecThree{} is already quite abstract. However, as
Alloy's core abstractions are objects and relations between them, the
specification looks quite differently. For example, here is how we declare
signatures for the blocks and checkpoints:

\begin{lstlisting}[language=alloy,columns=fullflexible]
  sig Payload {}
  sig Signature {}
  fact atLeastFourSignatures { #Signature >= 4 }

  sig Block {
    slot: Int,
    body: Payload,
    parent: Block
  }
  sig Checkpoint {
    block: Block,
    slot: Int
  }

  one sig GenesisPayload extends Payload {}
  one sig GenesisBlock extends Block {}
\end{lstlisting}

In contrast to~\SpecThree{}, our Alloy specification does not describe a
state machine, but it specifies an arbitrary single state of the protocol.\rs{Like the SMT spec, right? if so, perhaps we should say so} As
in~\SpecThree{}, we compute the set of justified checkpoints as a fixpoint:

\begin{lstlisting}[language=alloy,columns=fullflexible]
  one sig JustifiedCheckpoints {
    justified: set Checkpoint
  }
  fact justifiedCheckpointsAreJustified {
    all c: JustifiedCheckpoints.justified |
      c.slot = 0 or 3.mul[#justifyingVotes[c].validator] >= 2.mul[#Signature]
  }
\end{lstlisting}

\subsection{Model Checking with Alloy}

Similar to our experiments with Apalache, we produce examples of configurations
that satisfy simple properties. For example:

\begin{lstlisting}[language=alloy,columns=fullflexible]
  run someFinalizedCheckpoint { some c: Checkpoint |
    isFinalized[c] and c.slot != 0
  }
  for 10 but 6 Block, 6 Checkpoint, 12 Vote, 5 int
\end{lstlisting}

For small search scopes, Alloy quickly finds examples, often in a matter
of seconds.

Ultimately, we are interested in showing that~$\textit{AccountableSafety}$
holds true. To this end, we want Alloy to show that the
formula~$\textit{noAccountableSafety}$ does not have a model (unsatisfiable).
Similar to Apalache, Alloy allows us to show safety for bounded scopes.

\begin{lstlisting}[language=alloy,columns=fullflexible]
  pred noAccountableSafety {
    disagreement and (3.mul[#slashableNodes] < #Signature)
  }
  run noAccountableSafety for 10 but 6 Block, 6 Checkpoint,
                                     4 Signature, 6 FfgVote, 15 Vote, 5 int
\end{lstlisting}

Thus it is interesting to increase these parameter values as much as possible
to cover more scenarios, while keeping them small enough that the SAT solver
still terminates within reasonable time. Table~\ref{tab:alloy-mc} summarizes
our experiments with Alloy when checking $\textit{noAccountableSafety}$ with
Kissat.  Since a vote includes an FFG vote and a validator's signature, the
number of FFG votes and validators bound the number of votes. Hence, if we only
increase the number of votes without increasing the number of FFG votes, we may
miss the cases when longer chains of justifications must be constructed.

\subsection{Model Checking Results}\label{sec:alloy-results}

As we can see, we have managed to show \textit{AccountableSafety} for chain
configurations of up to 6 blocks, including the genesis block. While the
configurations up to 4 blocks are important to analyze, they do not capture the
general case, namely, of two finalized checkpoints building on top of justified
checkpoints that are not the genesis. Thus, configurations of 5 or more blocks
are the most important ones, as they capture the general case.

To see if we could go a bit further beyond 5 blocks, we have tried a
configuration with 7 blocks. This configuration happened to be quite challenging
for Alloy and Kissat. We have run Kissat for over 16 days, and it was stuck at
the remaining seven percent for many days. At that point, it has introduced:

\begin{itemize}
  \item 60 thousand Boolean variables,
  \item 582 thousand ``irredundant'' clauses and 199 thousand binary clauses,
  \item 1.3 billion conflicts,
  \item 56 million restarts.
\end{itemize}

We conjecture that the inductive structure of justified and finalized
checkpoints make it challenging for the solvers (both SAT and SMT) to analyze
the specification. Essentially, the solvers have to reason about chains of
checkpoints on top of chains of blocks.

\begin{table}
    \centering
    \begin{tabular}{rrrrrrrr}
        \tbh{\#}
            & \tbh{blk}
            & \tbh{chk}
            & \tbh{sig}
            & \tbh{ffg}
            & \tbh{votes}
            & \tbh{Time}
            & \tbh{Memory}
            \\ \toprule
        1 & 3 & 5 & 4 & 5  & 12 & 4s & 35M
            \\
        2 & 4 & 5 & 4 & 5 & 12 & 10s & 40M
            \\
        3 & 5 & 5 & 4 & 5 & 12 & 15s & 45M
            \\
        4 & 3 & 6 & 4 & 6 & 15 & 57s & 52M
            \\
        5 & 4 & 6 & 4 & 6 & 15 & 167s & 55M
            \\
        6 & 5 & 6 & 4 & 6 & 15 & 245s & 57M
            \\
        7 & 6 & 6 & 4 & 6 & 15 & 360s & 82M
            \\
        8 & 5 & 7 & 4 & 6 & 24 & 1h 27m & 156M
            \\
        9 & 5 & 10 & 4 & 8 & 24 & over 8 days (timeout) & 198 MB
            \\
        9 & 5 & 10 & 4 & 8 & 32 & over 8 days (timeout) & 220 MB
            \\
        10 & 3 & 15 & 4 & 5 & 12 & 31s & 56M
            \\
        11 & 4 & 20 & 4 & 5 & 12 & 152s & 94M
            \\
        12 & 5 & 25 & 4 & 5 & 12 & 234s & 117M
            \\
        13 & 7 & 15 & 4 & 10 & 40 & over 16 days (timeout) & 300 MB
            \\ \bottomrule
    \end{tabular}
    \caption{Model checking experiments with Alloy and Kissat:
      \textbf{blk} is the number of blocks, \textbf{chk} is the number
      of checkpoints, \textbf{sig} is the number of validator signatures,
      \textbf{ffg} is the number of FFG votes, \textbf{votes} is the number
      of validator votes.
    }\label{tab:alloy-mc}
\end{table}


%! TeX root = report.tex

\section{\SpecFour{}: Two chains in \tlap{}}\label{sec:spec4}

We obtain this specification by preserving the vote and checkpoint behavior from \SpecThree{}, but we restrict the block-graph to exactly two chains, which are allowed to fork, i.e., they always share a common prefix.
To this end, we restrict block bodies of the two chains as follows:
\begin{enumerate}
	\item Block bodies on the first chain have nonnegative sequential integer numbers starting from 0 (genesis), e.g. $0, 1, 2,3, 4$.
	\item Block bodies on the second chain are the same as the ones on the first chain \emph{up to the fork point}, after which they change sign, but otherwise maintain their sequence in absolute value terms, e.g. $0, 1,2,-3,-4$.
\end{enumerate}

The aim of this encoding is to decrease complexity for the SMT solver, since it simplifies block graph ancestry and closure reasoning significantly (e.g. we can check whether one block is an ancestor of another by simply comparing block body integer values), while preserving the behavior which arises from voting on checkpoint justification.

\subsection{Inductive Invariant}\label{sec:spec4-indinv}

This specification defines an inductive invariant \textit{IndInv}. Recall, an inductive invariant satisfies the following two conditions, assuming we have an initial-state predicate \textit{Init}, and transition predicate \textit{Next}:
\begin{enumerate}
	\item It is implied by the initial state: $\mathit{Init} \Rightarrow \mathit{IndInv}$
	\item It is preserved by the transition predicate: $\mathit{IndInv} \land \mathit{Next} \Rightarrow \mathit{IndInv}$
\end{enumerate}
Because of this characterization, inductive invariants lend themselves especially nicely to bounded symbolic model checking; with Apalache, one can prove (or disprove) an inductive invariant by running two queries of depth at most 1, corresponding to the above properties.
If no violation is found, we are assured that \textit{IndInv} holds in all possible reachable states.
Since our goal is ultimately to prove or disprove \textit{AccountableSafety}, we can additionally prove $\mathit{IndInv} \Rightarrow \mathit{AccountableSafety}$.

The challenge, typically, is that inductive invariants are more difficult to write, compared to state invariants.
They are usually composed of several lemmas, i.e. properties that we are less interested in on their own, but which are crucial in establishing property (2.).

As we explain below, the inductive invariant introduced in \SpecFour{} mainly contains of two sets of lemmas, one for characterizing the chain-fork scenarios, and a second one for characterizing justified checkpoints, shown in Figure~\ref{figFork} and Figure~\ref{figJC} respectively.

\begin{figure}
  \includegraphics[width=\textwidth]{images/chain-and-fork-inv.pdf}
  \caption{Chain and forking lemmas in \textit{IndInv}}\label{figFork}
\end{figure}

\begin{figure}
  \includegraphics[width=\textwidth]{images/justified-inv.pdf}
  \caption{Justified checkpoint lemmas in \textit{IndInv}}\label{figJC}
\end{figure}

Since we represent a fork by means of a sign-change on block numbers, we have to specify that their absolute values are contiguous, that the chains coincide in the absence of a fork, as well as on the pre-fork prefix, and that both chains are monotone w.r.t. block numbers after the fork point.
Additionally, we require validity predicates for the vote set and checkpoints, as well as a precise characterization of the set of justified checkpoints. The latter merits further discussion.

\paragraph{Justified checkpoints.} To accurately describe the set of all justified checkpoints, we require two constraints:
\begin{enumerate}
	\item Consistency: Every checkpoint in the set is justified, and 
	\item Completeness: Every justified checkpoint belongs to the set.
\end{enumerate}
It is worth noting that both of these properties are required for an inductive invariant; if we don't specify consistency, the solver can trivially infer that the set contains all possible checkpoints, and that all checkpoints are finalized, which leads to bogus counterexamples to \textit{AccountableSafety}.
On the other hand, if we don't specify completeness, the solver can trivially infer that the set of justified checkpoints is empty (or contains exactly the genesis checkpoint). This leads us to be unable to detect real violations of \textit{AccountableSafety}, since we can never observe conflicting finalized checkpoints, even when the votes cast necessitate their existence.

This is critical, because including both constraints places a heavy burden on the solver. No matter how we define the justification predicate, it will inevitably appear in both positive and negative form, forcing the solver to contend with both quantifier alternations and double universal quantification, both of which are known to be hard.
Fundamentally, this demonstrates the intrinsic complexity of the problem itself, regardless of the particular characterization of justified sets in \tlap{}.

\subsection{Model checking experiments}

Table~\ref{tab:spec4-experiments} summarizes our experiments on checking
inductiveness and accountable safety of~\SpecFourB{}. Interestingly, checking
inductiveness of~\texttt{IndInv} takes seconds. However,
checking~\texttt{AccountableSafety} against the inductive invariant times out.

\begin{table}
    \centering
    \begin{tabular}{lllrrr}
            \tbh{Init}
            & \tbh{Step}
            & \tbh{Invariant}
            & \tbh{Depth}
            & \tbh{Memory}
            & \tbh{Time}
            \\ \toprule
            \texttt{Init}
            & \texttt{Next}
            & \texttt{IndInv}
            & 0
            & 0.6 GB
            & 2sec
            \\
            \texttt{IndInit}
            & \texttt{Next}
            & \texttt{IndInv}
            & 1
            & 0.6 GB
            & 2sec
            \\
            \texttt{IndInit}
            & \texttt{Next}
            & \texttt{AccountableSafety}
            & 0
            & 1.5 GB
            & TO ($> 6$d)
            \\
            \bottomrule
    \end{tabular}
    \caption{Model checking experiments
        with~\SpecFour{} on the instance~\texttt{MC\_ffg\_b3\_ffg5\_v12}
        (``TO'' means timeout)
    }\label{tab:spec4-experiments}
\end{table}


%! TeX root = ./report.tex

\section{\SpecFourB{}: Decomposition \& abstractions}\label{sec:spec4b}

Disappointed with the results in Section~\ref{sec:spec4}, we have decided to
push abstractions even further. These abstractions helped us to verify
accountable safety for the configuration in Figure~\ref{fig:three}.
Unfortunately, they do not scale to larger configurations. In any case, we
find these abstractions quite important for further research on model checking
of algorithms similar to 3SF\@.

\subsection{Constraints over set cardinalities}

\SpecThree{} contains several comparisons over set cardinalities. For example:

\begin{equation}
    3 * \textit{Cardinality}(\textit{validatorsWhoCastJustifyingVote}) \ge 2 \cdot N
    \label{eq:card-comparison}
\end{equation}

In the general case, the symbolic model checker has to encode constraints for
the cardinality computation in Equation~(\ref{eq:card-comparison}). If a set
$S$ contains up to $n$~elements, Apalache produces $O(n^2)$~constraints for
$\textit{Cardinality(S)}$.

To partially remediate the above issue in~\SpecFour{}, we introduce a
constant~$T$ for the upper bound on the number of faulty process, and further
refine Equation~(\ref{eq:card-comparison}) to:

\begin{equation}
    \textit{Cardinality}(\textit{validatorsWhoCastJustifyingVote}) \ge 2 \cdot T + 1
    \label{eq:card-comparison2}
\end{equation}

Apalache applies an optimized translation rule for
Equation~(\ref{eq:card-comparison2}). Essentially, the solver has to find $2
\cdot T + 1$ set elements to show that Equation~(\ref{eq:card-comparison2})
holds true. This gives us a linear number of constraints, instead of a
quadratic one. For example, when $T=1$ and the set may contain up to~$n$
elements, the model checker produces $3 \cdot O(n)$ constraints to check
Equation~(\ref{eq:card-comparison2}). A similar optimization is used in the
specification of Tendermint~\cite{TendermintSpec2020}.

\subsection{Quorum sets}

To further optimize the constraints over set cardinalities, we have applied the
well-known pattern of replacing cardinality tests with quorum sets. For
example, this approach is used in the specification of Paxos\footnote{%
    \tlap{} specification of Paxos:
\url{https://github.com/tlaplus/Examples/blob/master/specifications/Paxos/Paxos.tla}.}

To this end, we introduce quorum sets such as in Figure~\ref{fig:quorum-sets}.

\begin{figure}[!h]
    \includegraphics[width=\textwidth]{images/quorum-sets}
    \caption{Quorum sets}\label{fig:quorum-sets}
\end{figure}

By using quorum sets, we further replace cardinality comparisons like in
Equation~(\ref{eq:card-comparison2}) with membership tests like in
Equation~(\ref{eq:card-comparison3}):

\begin{equation}
    \textit{validatorsWhoCastJustifyingVote} \in \textit{LargeQuorums}
    \label{eq:card-comparison3}
\end{equation}


\subsection{Decomposition of chain configurations}\label{sec:decomposition}

Recall Figure~\ref{figFork} from Section~\ref{sec:spec4-indinv}, which poses
constraints on the chains and the fork points. While these constraints should
be easier for an SMT solver than general reachability properties, they
still produce a number of arithmetic constraints. On the other hand, when we do
model checking for small parameters, there is a relatively small set of
possible chain configurations.  Figure~\ref{fig:block-graphs} shows some of
these configurations for the graphs of 3 to 7 blocks.

This observation led us to the following idea. Instead of using the constraints
over blocks such as in Figure~\ref{figFork}, we introduce one instance per
chain configuration. It is thus sufficient to verify accountable safety for all
of these instances and aggregate the model checking results.

Figure~\ref{fig:indinit-c3} shows an initialization predicate for the
configuration shown in Figure~\ref{fig:five2}. This predicate replaces the
general initialization predicate that we discussed in
Section~\ref{sec:spec4-indinv}.

\begin{figure}
    \includegraphics[width=\textwidth]{images/indinit-c3}
    \caption{Initialization predicate for the
             configuration~\text{M5b}}\label{fig:indinit-c3}
\end{figure}

\subsection{Model checking experiments}

Tables~\ref{tab:spec4b-experiments} and~\ref{tab:spec4b-inductiveness}
summarizes our experiments with Apalache for various configurations. One
interesting effect of the optimizations, especially of the ones presented in
Section~\ref{sec:decomposition}, is a significant drop in the memory
consumption of the SMT solver. In our experiments, Z3 required from 700~MB to
1.5~GB\@. While this is still a factor of 20 in comparison to the Alloy
experiments in Section~\ref{sec:alloy}, this is significantly better than our
initial experiments with~\SpecTwo{} and~\SpecThree{}, which required up to
20~GB of RAM\@. Interestingly, inductiveness checks in
Table~\ref{tab:spec4b-inductiveness} take significantly longer than in the case
of~\SpecFour{}. We conjecture that this is caused by the need to check more
specialized graphs in conjunction with steps.

We notice significant variations in the running times on different
configurations. This is probably due to the effect of different runs of the Z3
SMT solve having great variety of running times, which is well-known in the
computer-aided community. To further confirm these variations, we could run
multiple experiments with `hyperfine`.

\begin{table}
    \centering
    \begin{tabular}{lllrr}
        \tbh{Configuration}
            & \tbh{Instance}
            & \tbh{Init}
            & \tbh{Memory}
            & \tbh{Time}
            \\ \toprule
        M3: Fig.~\ref{fig:three}
            & \texttt{MC\_ffg\_b1\_ffg5\_v12}
            & \texttt{Init\_C1}
            & 1.2 GB
            & 11h 31min
            \\
        M4a: Fig.~\ref{fig:four-top}
            & \texttt{MC\_ffg\_b3\_ffg5\_v12}
            & \texttt{Init\_C4}
            & XXX
            & XXX
            \\
        M4b: Fig.~\ref{fig:four-bottom}
            & \texttt{MC\_ffg\_b3\_ffg5\_v12}
            & \texttt{Init\_C2}
            & 1.3 GB
            & 1day 6h
            \\
        M5a: Fig.~\ref{fig:five1}
            & \texttt{MC\_ffg\_b3\_ffg5\_v12}
            & \texttt{Init\_C3}
            & 1.2 GB
            & 1h 53min
            \\
        M5b: Fig.~\ref{fig:five2}
            & \texttt{MC\_ffg\_b3\_ffg5\_v12}
            & \texttt{Init\_C1}
            & XXX
            & XXX
            \\
            \bottomrule
    \end{tabular}
    \caption{Checking accountable safety
             against~\SpecFourB{}}\label{tab:spec4b-experiments}
\end{table}

\begin{table}
    \centering
    \begin{tabular}{lllrr}
        \tbh{Instance}
            & \tbh{Init}
            & \tbh{Invariant}
            & \tbh{Memory}
            & \tbh{Time}
            \\ \toprule
        \texttt{MC\_ffg\_b1\_ffg5\_v12}
            & \texttt{Init}
            & \texttt{IndInv}
            & 580 MB
            & 7s
            \\
        \texttt{MC\_ffg\_b3\_ffg5\_v12}
            & \texttt{Init}
            & \texttt{IndInv}
            & 700 MB
            & 7s
            \\
        \texttt{MC\_ffg\_b1\_ffg5\_v12}
            & \texttt{Init\_C1}
            & \texttt{IndInv}
            & 1.4 GB
            & 2min 8s
            \\
        \texttt{MC\_ffg\_b3\_ffg5\_v12}
            & \texttt{Init\_C1}
            & \texttt{IndInv}
            & 1.8 GB
            & 19min 10s
            \\
        \texttt{MC\_ffg\_b3\_ffg5\_v12}
            & \texttt{Init\_C2}
            & \texttt{IndInv}
            & 1.6 GB
            & 13min 16s
            \\
        \texttt{MC\_ffg\_b3\_ffg5\_v12}
            & \texttt{Init\_C3}
            & \texttt{IndInv}
            &  1.6 GB
            & 17min 39s
            \\
        \texttt{MC\_ffg\_b3\_ffg5\_v12}
            & \texttt{Init\_C4}
            & \texttt{IndInv}
            & 1.6 GB
            & 16min 23s
            \\
            \bottomrule
    \end{tabular}
    \caption{Checking inductiveness
             for~\SpecFourB{}}\label{tab:spec4b-inductiveness}
\end{table}



\section{Summary}\label{sec:summary}
We have presented a series of specifications modeling the 3SF protocol from
various perspectives. Initially, we developed a direct translation of the
protocol's Python specification into \tlap{}, but this approach proved
unsatisfactory due to the reliance on recursion. To address this, we modified
the specification to use folds in place of recursion, theoretically enabling
model-checking with Apalache. However, this approach also proved impractical
due to the high computational complexity involved. Subsequently, we applied a
series of abstractions to improve the model-checking efficiency.

In addition to the \tlap{} specifications, we also introduced an SMT encoding
and an Alloy specification. The SMT encoding proved to be fairly performant,
while the Alloy specification demonstrated exceptional performance in
combination with the Kissat SAT solver.

To revisit the key outcomes of the project, see Section~\ref{sec:outcomes}.

\bibliographystyle{plain}
\bibliography{ref}

% appendix
\pagebreak

\appendix

The work done in this section is the main contribution of Milestone~3. Since
this section is quite long and technical, we have decided to add it in the
appendix.

% Milestone 3 - Translation rules
\section{Python to \tlap{} translation rules}

On the Python side, we have the $\texttt{typing}$ package and types defined in $\texttt{pyrsistent}$. 

We assume the existence of a type-translation from those types to Apalache's type system.

We maintain the following convention: if $\tau$ is the Python type, we label the corresponding Apalache type as $\htau$.

\subsection{Translation rules}

We will be using the \href{https://github.com/konnov/tlaki/blob/main/src/Lists.tla}{$\texttt{Lists}$} module, in lieu of $\texttt{Sequences}$, to better match the 0-indexing convention of Python. To that end, we introduce the type notation:
\[
\List(\htau) \coloneqq \{ es\colon \Seq(\htau) \}
\]
that is, the instantiation of the $\texttt{list}$ alias defined in $\texttt{Lists.tla}$ with the concrete type $\hat{t}$.

\subsubsection{Singleton vector}
\href{https://github.com/saltiniroberto/ssf/blob/7ea6e18093d9da3154b4e396dd435549f687e6b9/high_level/common/pythonic_code_generic.py#L15-L16}{Source}.


%     a: t
%   ==========   pvector_of_one_element(a) 
%     e: t^
% ==========================================
%         List(<< e >>) : List(t^)

\begin{mathpar}
    \inferrule*[right=(\textsc{Vec})]
    {
        \inferrule{a\colon \tau}{e \colon \htau}
        \\
        \mathrm{pvector\_of\_one\_element}(a)
    }{
        \List(\tup{e})\colon \List(\htau)
    }
\end{mathpar}
A singleton Python vector is translated to a single-element list, and annotated as such.

\subsubsection{Vector concatenation}
\href{https://github.com/saltiniroberto/ssf/blob/7ea6e18093d9da3154b4e396dd435549f687e6b9/high_level/common/pythonic_code_generic.py#L19-L20}{Source}.


%    a: PVec[t]        b: PVec[t]
%  ===============   ===============   pvector_concat(a, b) 
%    e: List(t^)       f: List(t^)
% ============================================================
%                  Concat(e,f) : List(t^)

\begin{mathpar}
    \inferrule*[right=(\textsc{Concat})]
    {
        \inferrule{a\colon \PVec[\tau]}{e \colon \List(\htau)}
        \\
        \inferrule{b\colon \PVec[\tau]}{f \colon \List(\htau)}
        \\
        \mathrm{pvector\_concat}(a, b)
    }{
        \Concat(e,f) \colon \List(\htau)
    }
\end{mathpar}
Vector concatenation is translated to the list concatenation.

\subsubsection{Set sequentialization}
\href{https://github.com/saltiniroberto/ssf/blob/7ea6e18093d9da3154b4e396dd435549f687e6b9/high_level/common/pythonic_code_generic.py#L23-L24}{Source}.


%          a: PSet[t]         
%        ==============   from_set_to_pvector(a) 
%          e: Set(t^)       
% ======================================================
%   ApaFoldSet( Push, List(<<>>: Seq(t^)), e ) : List(t^)

\begin{mathpar}
    \inferrule*[right=(\textsc{SetToVec})]
    {
        \inferrule{a\colon \PSet[\tau]}{e \colon \Set(\htau)}
        \\
        \mathrm{from\_set\_to\_pvector}(a)
    }{
        \ApaFoldSet( \Push, \List(\tup{}\colon \Seq(\htau)), e ) \colon \List(\htau)
    }
\end{mathpar}
We use fold, to create a sequence (in some order) from the set.

\subsubsection{ Empty set}
\href{https://github.com/saltiniroberto/ssf/blob/7ea6e18093d9da3154b4e396dd435549f687e6b9/high_level/common/pythonic_code_generic.py#L27-L28}{Source}.


%   pset_get_empty : PSet[t]
% ============================
%        {} : Set(t^)

\begin{mathpar}
    \inferrule*[right=(\textsc{EmptySet})]
    {
        \mathrm{pset\_get\_empty}() \colon \PSet[t]
    }{
        \{\} \colon \Set(\htau)
    }
\end{mathpar}
The only relevant part here is that we need a type annotation on the Python side to correctly annotate the empty set in \tlap{}.

\subsubsection{ Set union}
\href{https://github.com/saltiniroberto/ssf/blob/7ea6e18093d9da3154b4e396dd435549f687e6b9/high_level/common/pythonic_code_generic.py#L31-L32}{Source}.


%     a: PSet[t]       b: PSet[t]
%   ==============   ==============   pset_merge(a, b) 
%     e: Set(t^)       f: Set(t^)
% ======================================================
%                 e \cup f : Set(t^)

\begin{mathpar}
    \inferrule*[right=(\textsc{Union})]
    {
        \inferrule{a\colon \PSet[\tau]}{e \colon \Set(\htau)}
        \\
        \inferrule{b\colon \PSet[\tau]}{f \colon \Set(\htau)}
        \\
        \mathrm{pset\_merge}(a, b)
    }{
        e \cup f \colon \Set(\htau)
    }
\end{mathpar}
Set union is translated to the \tlap{} native set union.

\subsubsection{ Set flattening}
\href{https://github.com/saltiniroberto/ssf/blob/7ea6e18093d9da3154b4e396dd435549f687e6b9/high_level/common/pythonic_code_generic.py#L35-L36}{Source}.


%          a: PSet[PSet[t]]         
%        ====================   pset_merge_flatten(a) 
%          e: Set(Set(t^))       
% =====================================================
%                 UNION e : Set(t^)

\begin{mathpar}
    \inferrule*[right=(\textsc{BigUnion})]
    {
        \inferrule{ a\colon \PSet[\PSet[\tau]]}{e \colon \Set(\Set(\htau))}
        \\
        \mathrm{pset\_merge\_flatten}(a)
    }{
        \UNION\; e \colon \Set(\htau)
    }
\end{mathpar}
Set flattening is translated to the \tlap{} native big $\UNION$.

\subsubsection{Set intersection}
\href{https://github.com/saltiniroberto/ssf/blob/7ea6e18093d9da3154b4e396dd435549f687e6b9/high_level/common/pythonic_code_generic.py#L42-L43}{Source}.


%     a: PSet[t]       b: PSet[t]
%   ==============   ==============   pset_intersection(a, b) 
%     e: Set(t^)       f: Set(t^)
% =============================================================
%                    e \cap f : Set(t^)

\begin{mathpar}
    \inferrule*[right=(\textsc{Intersection})]
    {
        \inferrule{a\colon \PSet[\tau]}{e \colon \Set(\htau)}
        \\
        \inferrule{b\colon \PSet[\tau]}{f \colon \Set(\htau)}
        \\
        \mathrm{pset\_intersection}(a, b)
    }{
        e \cap f \colon \Set(\htau)
    }
\end{mathpar}
Set intersection is translated to the \tlap{} native set intersection.

\subsubsection{Set difference}
\href{https://github.com/saltiniroberto/ssf/blob/7ea6e18093d9da3154b4e396dd435549f687e6b9/high_level/common/pythonic_code_generic.py#L46-L47}{Source}.


%     a: PSet[t]       b: PSet[t]
%   ==============   ==============   pset_difference(a, b) 
%     e: Set(t^)       f: Set(t^)
% ===========================================================
%                       e \ f : Set(t^)

\begin{mathpar}
    \inferrule*[right=(\textsc{SetDiff})]
    {
        \inferrule{a\colon \PSet[\tau]}{e \colon \Set(\htau)}
        \\
        \inferrule{b\colon \PSet[\tau]}{f \colon \Set(\htau)}
        \\
        \mathrm{pset\_difference}(a, b)
    }{
        e \setminus f \colon \Set(\htau)
    }
\end{mathpar}
Set difference is translated to the \tlap{} native set difference.

\subsubsection{Singleton set}
\href{https://github.com/saltiniroberto/ssf/blob/7ea6e18093d9da3154b4e396dd435549f687e6b9/high_level/common/pythonic_code_generic.py#L50-L51}{Source}.

%     a: t
%   =========   pset_get_singleton(a) 
%     e: t^
% =====================================
%            { e } : Set(t^)

\begin{mathpar}
    \inferrule*[right=(\textsc{Singleton})]
    {
        \inferrule{a\colon \tau}{e \colon \htau}
        \\
        \mathrm{pset\_get\_singleton}(a)
    }{
        \{e\} \colon \Set(\htau)
    }
\end{mathpar}
A singleton Python set is translated to a \tlap{} native single-element set.

\subsubsection{Set extension}
\href{https://github.com/saltiniroberto/ssf/blob/7ea6e18093d9da3154b4e396dd435549f687e6b9/high_level/common/pythonic_code_generic.py#L54-L55}{Source}.


%     a: PSet[t]       b: t
%   ==============   =========  pset_add(a, b) 
%     e: Set(t^)       f: t^
% ==============================================
%             e \cup { f } : Set(t^) 

\begin{mathpar}
    \inferrule*[right=(\textsc{SetExt})]
    {
        \inferrule{a\colon \PSet[\tau]}{e \colon \Set(\htau)}
        \\
        \inferrule{b\colon \tau}{f \colon \htau}
        \\
        \mathrm{pset\_add}(a, b)
    }{
        e \cup \{f\} \colon \Set(\htau)
    }
\end{mathpar}
A set extension is translated to a combination of union and singleton-set construction. Semantically, this is the equivalence
\[
\mathrm{pset\_add}(a,b) = \mathrm{pset\_merge}(a, \mathrm{pset\_get\_singleton}(b))
\]

\subsubsection{Element choice}
\href{https://github.com/saltiniroberto/ssf/blob/7ea6e18093d9da3154b4e396dd435549f687e6b9/high_level/common/pythonic_code_generic.py#L58-L60}{Source.}


%     a: PSet[t]
%   ==============   pset_pick_element(a) 
%     e: Set(t^)
% =========================================
%          CHOOSE x \in e: TRUE: t^

\begin{mathpar}
    \inferrule*[right=(\textsc{Choice})]
    {
        \inferrule{a\colon \PSet[\tau]}{e \colon \Set(\htau)}
        \\
        \mathrm{pset\_pick\_element}(a)
    }{
        (\CHOOSE\; x \in e\colon\TRUE) \colon\htau
    }
\end{mathpar}
We translate this to the built in deterministic choice in \tlap{}. We cannot account for the dynamic non-emptiness requirement. Instead, in that scenario, the value of this expression is some unspecified element of the correct type.

\subsubsection{Set filter}
\href{https://github.com/saltiniroberto/ssf/blob/7ea6e18093d9da3154b4e396dd435549f687e6b9/high_level/common/pythonic_code_generic.py#L63-L70}{Source}.


%     a: Callable[[t], bool]        b: PSet[t]
%   ===========================   ==============   pset_filter(a, b) 
%          e: t^ -> bool            f: Set(t^)
% ====================================================================
%                       { x \in f: e[x] }: Set(t^)

\begin{mathpar}
    \inferrule*[right=(\textsc{Filter})]
    {
        \inferrule{a\colon \Callable[[\tau], \bool]}{e \colon \htau \to \Bool}
        \\
        \inferrule{b\colon \PSet[\tau]}{f \colon \Set(\htau)}
        \\
        \mathrm{pset\_filter}(a, b)
    }{
        \{ x \in f \colon e[x] \} \colon \Set(\htau)
    }
\end{mathpar}
Set filtering is translated to the \tlap native filter operation.

\subsubsection{ Set maximum}
\href{https://github.com/saltiniroberto/ssf/blob/7ea6e18093d9da3154b4e396dd435549f687e6b9/high_level/common/pythonic_code_generic.py#L74-L76}{Source}.


%     a: PSet[t]       b: Callable[[t], T]        
%   ==============   =======================   pset_max(a, b) 
%     e: Set(t^)           f: t^ -> T^            
% =============================================================
%       CHOOSE max \in e: \A x \in e: Le(f[x], f[max])

\begin{mathpar}
    \inferrule*[right=(\textsc{Max})]
    {
        \inferrule{a\colon \PSet[\tau]}{e \colon \Set(\htau)}
        \\
        \inferrule{b\colon \Callable[[\tau], T]}{f \colon \htau \to \hat{T}}       
        \\
        \mathrm{pset\_max}(a, b)
    }{
        (\CHOOSE\; m \in e\colon \forall x \in e\colon \Le(f[x], f[m]))\colon \htau
    }
\end{mathpar}
Here, the translation depends on the type $T$ (resp. type $\hat{T}$), since there is no built-in notion of ordering in \tlap{}. If $\hat{T}$ is an integer type, then 
\[
\Le(x,y) \defeq x <= y
\]
However, if $\hat{T}$ is a tuple type $\tup{\Int,\Int}$, it is instead 

\begin{align*}
\Le(x,y) \defeq&\IF\; x[1] > y[1]\\
  &\THEN\;\FALSE\\
  &\ELSE\;\IF\; x[1] < y[1]\\
       &\phantom{\ELSE\;}\THEN\; \TRUE\\
       &\phantom{\ELSE\;}\ELSE\; x[2] \le y[2]\\
\end{align*}

\subsubsection{ Set sum}
\href{https://github.com/saltiniroberto/ssf/blob/7ea6e18093d9da3154b4e396dd435549f687e6b9/high_level/common/pythonic_code_generic.py#L79-L80}{Source}.


%             a: PSet[int]
%           ================   pset_sum(a) 
%             e: Set(int)
% ===========================================================
%   LET Plus(x,y) == x + y IN ApaFoldSet(Plus, 0, e ): int

\begin{mathpar}
    \inferrule*[right=(\textsc{Sum})]
    {
        \inferrule{a\colon \PSet[\pyint]}{e \colon \Set(\Int)}
        \\
        \mathrm{pset\_sum}(a)
    }{
        \LET\; \mathrm{Plus}(x,y) \defeq x + y\; \IN\; \ApaFoldSet(\mathrm{Plus}, 0, e) \colon \Int
    }
\end{mathpar}
We translate a set sum with an aggregator fold.

\subsubsection{ Set emptiness check}
\href{https://github.com/saltiniroberto/ssf/blob/7ea6e18093d9da3154b4e396dd435549f687e6b9/high_level/common/pythonic_code_generic.py#L83-L84}{Source}.


%      a: PSet[t]         
%    ==============   pset_is_empty(a) 
%      e: Set(t^)       
% ======================================
%             e = {} : bool

\begin{mathpar}
    \inferrule*[right=(\textsc{IsEmpty})]
    {
        \inferrule{a\colon \PSet[\tau]}{e \colon \Set(\htau)}
        \\
        \mathrm{pset\_is\_empty}(a)
    }{
        e = \{\} \colon \Bool
    }
\end{mathpar}
The emptiness check is translated to a comparison with the explicitly constructed empty set.

\subsubsection{ Vector-to-Set conversion}
\href{https://github.com/saltiniroberto/ssf/blob/7ea6e18093d9da3154b4e396dd435549f687e6b9/high_level/common/pythonic_code_generic.py#L87-L88}{Source}.


%     a: PVec[t]         
%   ===============   from_pvector_to_pset(a) 
%     e: List(t^)       
% =============================================
%   { At(e, i) : i \in Indices(e) }: Set(t^)            

\begin{mathpar}
    \inferrule*[right=(\textsc{VecToSet})]
    {
        \inferrule{a\colon \PVec[\tau]}{e \colon \List(\htau)}
        \\
        \mathrm{from\_pvector\_to\_pset}(a)
    }{
        \{ \At(e, i)\colon i \in \Indices(e) \} \colon \Set(\htau)  
    }
\end{mathpar}
We translate the set-conversion, by mapping the accessor method over $\Indices$.

\subsubsection{ Set mapping}
\href{https://github.com/saltiniroberto/ssf/blob/7ea6e18093d9da3154b4e396dd435549f687e6b9/high_level/common/pythonic_code_generic.py#L91-L97}{Source}.


%     a: Callable[[t1], t2]       b: PSet[t1]
%   =========================   ==============   pset_map(a, b) 
%         e: t1^ -> t2^           f: Set(t1^)
% ===============================================================
%                 { e[x]: x \in f}: Set(t2^)

\begin{mathpar}
    \inferrule*[right=(\textsc{Map})]
    {
        \inferrule{a\colon \Callable[[\tau_1], \tau_2]}{e \colon \htau_1 \to \htau_2}
        \\
        \inferrule{b\colon \PSet[\tau_1]}{f \colon \Set(\htau_1)}
        \\
        \mathrm{pset\_map}(a, b)
    }{
        \{ e[x]\colon x \in f\} \colon \Set(\htau_2)
    }
\end{mathpar}
Set mapping is translated to the \tlap{} native map operation.

\subsubsection{ Function domain inclusion check}
\href{https://github.com/saltiniroberto/ssf/blob/7ea6e18093d9da3154b4e396dd435549f687e6b9/high_level/common/pythonic_code_generic.py#L100-L101}{Source}.


%     a: PMap[t1, t2]       b: t1
%   ===================   =========   pmap_has(a, b) 
%      e: t1^ -> t2^        f: t1^
% ===================================================
%               f \in DOMAIN e: bool 

\begin{mathpar}
    \inferrule*[right=(\textsc{InDom})]
    {
        \inferrule{a\colon \PMap[\tau_1, \tau_2]}{e \colon \htau_1 \to \htau_2}
        \\
        \inferrule{b\colon \tau_1}{f \colon \htau_1}
        \\
        \mathrm{pmap\_has}(a, b)
    }{
        f \in \DOMAIN\;e\colon \Bool
    }
\end{mathpar}
Function domain inclusion checking is translated to the \tlap{} native set-inclusion operation for $\DOMAIN$.

\subsubsection{ Function application}
\href{https://github.com/saltiniroberto/ssf/blob/7ea6e18093d9da3154b4e396dd435549f687e6b9/high_level/common/pythonic_code_generic.py#L104-L106}{Source}.


%     a: PMap[t1, t2]       b: t1
%   ===================   =========   pmap_get(a, b) 
%      e: t1^ -> t2^        f: t1^
% ===================================================
%                   e[f]: t2^

\begin{mathpar}
    \inferrule*[right=(\textsc{App})]
    {
        \inferrule{a\colon \PMap[\tau_1, \tau_2]}{e \colon \htau_1 \to \htau_2}
        \\
        \inferrule{b\colon \tau_1}{f \colon \htau_1}
        \\
        \mathrm{pmap\_get}(a, b)
    }{
        e[f]\colon \htau_2
    }
\end{mathpar}
Function application is translated to the \tlap{} native function application. We cannot account for the dynamic domain-membership requirement. Instead, in that scenario, the value of this expression is some unspecified element of the correct type.


\subsubsection{ Empty function}
\href{https://github.com/saltiniroberto/ssf/blob/7ea6e18093d9da3154b4e396dd435549f687e6b9/high_level/common/pythonic_code_generic.py#L109-L110}{Source}.


%          pmap_get_empty : PMap[t1, t2]
% ===============================================
%   SetAsFun({}: Set(<<t1^, t2^>>)): t1^ -> t2^

\begin{mathpar}
    \inferrule*[right=(\textsc{EmptyFun})]
    {
        \mathrm{pmap\_get\_empty}()\colon \PMap[\tau_1,\tau_2]
    }{
      	\SetAsFun(\{\}\colon \Set(\tup{\htau_1, \htau_2}))\colon \htau_1 \to \htau_2
    }
\end{mathpar}
We use Apalache's $\SetAsFun$, since we only need to annotate the empty set with the correct tuple type. The native construction via $[ \_ \mapsto \_]$ would require us to invent a codomain value, which we might not have access to if $\tau_1 \ne \tau_2$ (but could be e.g. $\mathrm{Gen}$'d since we know it will never be used).

\subsubsection{ Function update}
\href{https://github.com/saltiniroberto/ssf/blob/7ea6e18093d9da3154b4e396dd435549f687e6b9/high_level/common/pythonic_code_generic.py#L113-L114}{Source}.


%     a: PMap[t1, t2]       b: t1       c: t2
%   ===================   =========   ==========   pmap_set(a, b, c) 
%      e: t1^ -> t2^        f: t1^      g: t2^
% ===================================================================
%                 [e EXCEPT ![f] = g] : t1^ -> t2^

\begin{mathpar}
    \inferrule*[right=(\textsc{Update})]
    {
        \inferrule{a\colon \PMap[\tau_1, \tau_2]}{e \colon \htau_1 \to \htau_2}
        \\
        \inferrule{b\colon \tau_1}{f \colon \htau_1}
        \\
        \inferrule{c\colon \tau_2}{g \colon \htau_2}
        \\
        \mathrm{pmap\_set}(a, b, c)
    }{
        [e\;\EXCEPT\;![f] = g] \colon \htau_1 \to \htau_2
    }
\end{mathpar}
Function update is translated to the \tlap{} native $\EXCEPT$.

\subsubsection{ Function combination}
\href{https://github.com/saltiniroberto/ssf/blob/7ea6e18093d9da3154b4e396dd435549f687e6b9/high_level/common/pythonic_code_generic.py#L117-L118}{Source}.


%                 a: PMap[t1, t2]       b: PMap[t1, t2]
%               ===================   ===================   pmap_merge(a, b) 
%                  e: t1^ -> t2^         f: t1^ -> t2^
% ============================================================================================
%   [ x \in (DOMAIN e \cup DOMAIN f) |-> IF x \in DOMAIN f THEN f[x] ELSE e[x] ]: t1^ -> t2^

\begin{mathpar}
    \inferrule*[right=(\textsc{FnMerge})]
    {
        \inferrule{a\colon \PMap[\tau_1, \tau_2]}{e \colon \htau_1 \to \htau_2}
        \\
        \inferrule{b\colon \PMap[\tau_1, \tau_2]}{f \colon \htau_1 \to \htau_2}
        \\
        \mathrm{pmap\_merge}(a,b)
    }{
        [x \in (\DOMAIN\; e \cup \DOMAIN\;f) \mapsto \IF\; x \in \DOMAIN\;f\;\THEN\; f[x]\;\ELSE\;e[x]]\colon \htau_1\to\htau_2
    }
\end{mathpar}
Function combination is translated to a new function, defined over the union of both domains. Note that the second map dominates in the case of key/domain collisions.


\subsubsection{ Function domain}
\href{https://github.com/saltiniroberto/ssf/blob/7ea6e18093d9da3154b4e396dd435549f687e6b9/high_level/common/pythonic_code_generic.py#L121-L122}{Source}.


%     a: PMap[t1, t2]   
%   ===================   pmap_keys(a) 
%      e: t1^ -> t2^    
% ======================================
%           DOMAIN e: Set(t1^)

\begin{mathpar}
    \inferrule*[right=(\textsc{FnDomain})]
    {
        \inferrule{a\colon \PMap[\tau_1, \tau_2]}{e \colon \htau_1 \to \htau_2}
        \\
        \mathrm{pmap\_keys}(a)
    }{
        \DOMAIN\;e\colon \Set(\htau_1)
    }
\end{mathpar}
We translate this to the \tlap{} native $\DOMAIN$.


\subsubsection{ Function codomain}
\href{https://github.com/saltiniroberto/ssf/blob/7ea6e18093d9da3154b4e396dd435549f687e6b9/high_level/common/pythonic_code_generic.py#L125-L126}{Source}.


%     a: PMap[t1, t2]   
%   ===================   pmap_values(a) 
%      e: t1^ -> t2^    
% ========================================
%    { e[x]: x \in DOMAIN e }: Set(t2^)

\begin{mathpar}
    \inferrule*[right=(\textsc{FnCodomain})]
    {
        \inferrule{a\colon \PMap[\tau_1, \tau_2]}{e \colon \htau_1 \to \htau_2}
        \\
        \mathrm{pmap\_values}(a)
    }{
        \{ e[x]\colon x \in \DOMAIN\; e \}\colon \Set(\htau_2)
    }
\end{mathpar}
We translate this by mapping the function over its $\DOMAIN$.



















\input{recursion-proofs}

\end{document}
