\appendix 
\section{Translation rules}

\subsection{Preliminaries}
Let $f$ be any \tlap function. We use the shorthand $D_f \coloneqq \DOMAIN f$.
We use $\nat$ to refer to the set of all natural numbers (i.e. nonnegative integers).

We assume the following precondition: There exists a $\gamma\colon a \to \nat$, with the following property:
\[
\forall x\colon a \ .\ \forall y \in V(x) \ .\ \gamma(y) < \gamma(x) 
\]

Recall the following definitions:
\[
\iteDef{\op(x)}{e}{P(x)}{G(x, F(V(x), \op))},
\]
\[
\iteDef{\mop(f)}{[ x \in D_f \mapsto e ]}{\forall x \in D_f \ .\ P(x)}{\mapg(f, \mop(\bb(f))}
\]
and
\[
\mapg(f, m)[x] \coloneqq \left\{
\begin{array}{ll}
      m[x] &; x \in D_m \\
      e &; x \notin D_m \land P(x) \\
      G(x, F(f[x], m)) &; \text{otherwise}\\
\end{array} 
\right. 
\]
with $D_{\mapg(f,m)} = D_f \cup D_m$.

\subsection{Proofs}

\begin{lemma}\label{lemma1}
Let $f$ be any function of type $a \to Set(a)$. Then
\[
D_f \subseteq D_{\mop(f)}
\]
\end{lemma}
\begin{proof}
Let $g \coloneqq \mop(f)$. To prove the first part, we have two options to consider, based on the definition of $\op$. If $\forall x \in D_f \ .\ P(x)$, then $D_f = D_g$.
Otherwise, $g = \mapg(f,\mop(\bb(f)))$. By definition, $D_{\mapg(f,m)} = D_f \cup D_m$, so it follows that $D_f \subseteq D_{\mapg(f,m)}$, for any $m$. 
\end{proof}
\begin{corollary}
If $D_f = \emptyset$, then $D_{g} = \emptyset$.
\end{corollary}
\begin{proof}
If $D_f$ is empty, then it is vacuously true that $\forall x \in D_f \ .\ P(x)$, and so $D_f = D_g$ by definition.
\end{proof}


% \begin{lemma}\label{lemma1}
% Let $x$ be any value of type $a$, s.t. $P(x)$ holds. Then, $h(V(x)) = \emptyset$
% \end{lemma}

% \begin{proof}
% By definition $h(V(x))$ is defined s.t. computing $\op(x)$ requires us to recursively compute $Op(v)$ for each $v \in h(V(x))$. Since for $x$ it is the case that $P(x)$ holds, and $\op$ is defined as:
% \[
% \iteDef{\op(x)}{e}{P(x)}{G(x, F(h(V(x)), \op))}
% \]
% we do not need to consider the second branch of the definition to compute $\op(x)$. Thus, we do not need to recursively compute $\op(v)$ for any $v$, and $h(V(x))$ is empty.
% \end{proof}

We define an auxiliary function $\alpha$ that assigns every function of the type $a \to Set(a)$ a value in $\nat \cup \{-\infty\}$, defined as:
\[
\alpha(f) \coloneqq \sup\left\{ \gamma(v) \mid v \in D_f \right\}
\]

\begin{lemma}\label{lemma2}
Let $f$ be a function, s.t. for any $x \in D_f$ it is the case that $f[x] = V(x)$. Then
\[
\alpha(f) \ge 0 \Rightarrow \alpha(\bb(f)) < \alpha(f)
\]
\end{lemma}

\begin{proof}
Assume $\alpha(f) \ge 0$. This is trivially equivalent to saying $D_f \ne \emptyset$.
We see that $\bb$ is defined as
\[
\bb(f)[y] \coloneqq V(y)
\]
where 
\[
D_{\bb(f)} \coloneqq \bigcup_{v \in D_f} f[v]
\]
It follows that 
\begin{align*} 
\alpha(\bb(f)) &= \sup\left\{ \gamma(v) \mid v \in D_{\bb(f)} \right\} \\
&= \sup\left\{ \gamma(v) \mid v \in \bigcup_{w \in D_f} f[w] \right\} \\
&= \sup\left\{ \gamma(v) \mid v \in \bigcup_{w \in D_f} V(w) \right\} \\
\end{align*}

By the precondition, we know that 
\[
\forall y \in V(x) \ .\ \gamma(y) < \gamma(x) 
\]

If we take an arbitrary $v \in \bigcup_{w \in D_f} V(w)$, there exists a
$w \in D_f$, s.t. $v \in V(w)$. The precondition then implies, that $\gamma(v) < \gamma(w)$.
Additionally, since $w \in D_f$, by the definition of $\alpha(f)$, it follows that $\gamma(w) \le \alpha(f)$.
Since all sets in question are finite, $D_{\bb(f)}$ is either empty, in which case $\alpha(\bb(f)) = -\infty$ and the lemma trivially holds, or the supremum is actually a maximum, and strict inequality is maintained when we infer
\[
\alpha(\bb(f)) = \sup\left\{ \gamma(v) \mid v \in \bigcup_{w \in D_f} V(w) \right\} < \alpha(f)
\]
\end{proof}
As a trivial corollary to this lemma, we observe that $\alpha(\bb(f)) = \alpha(f)$ iff $\alpha(f) = -\infty$.

\recallthm{thm}{\thmBody}
\begin{proof}
We prove this by using induction over $\alpha(f)$.

Base case $\alpha(f) = -\infty$: This can only be true if $D_f$ is empty. However, it is then vacuously true that 
\[
\forall x \in D_f \ .\ P(x)
\]
and it follows from the corollary to Lemma \ref{lemma1} that $D_g$ is empty too. In that case, any universally quantified statement over $D_g$ vacuously holds.

General case: Let $\alpha(f) = N \ge 0$ and assume the theorem holds for any $f'$, for which $\alpha(f') < N$.
We observe that $f' \coloneqq \bb(f)$ satisfies all of the requirements. By definition, 
\[
\bb(f)[y] \coloneqq V(y)
\]
for every element in its domain, so the precondition of the theorem is met. Additionally, by Lemma \ref{lemma2}, we know that $\alpha(f') < N$, so we can use the induction assumption to conclude 
\[
\forall x \in D_{\mop(f')} \ .\ \mop(f')[x] = \op(x)
\]

If it is the case that $\forall x \in D_f \ .\ P(x)$, then $\mop(f)[x] = e$ for all domain elements, since 
$D_f = D_{\mop(f)}$, by the same reasoning we used in the case where $D_f$ was empty. Similarly, for each such $x$, $\op(x) = e$ by the definition of $\op$, since $P(x)$ holds.

Next, we select an arbitrary $x \in D_{\mop(f)}$. We look at the case split in $\mapg$:
\begin{enumerate}
\item If $x \in D_{\mop(f')}$, by the induction hypothesis, we know that $\mop(f')[x] = \op(x)$.
\item If $x \notin D_{\mop(f')}$, but $P(x)$ holds, we know $\op(x) = e$. Here, $\mapg(f, \mop(f'))$ trivially evaluates to $e$ as well.
\item Otherwise, it remains to be shown that
\[
G(x, F(f[x], \mop(f'))) = \op(x)
\]
Since we know $P(x)$ does not hold for this $x$, it follows that
\[
\op(x) = G(x, F(V(x), Op))
\]
so it suffices to see that 
\[
F(f[x], \mop(f')) = F(V(x), Op)
\]
By the property of $f$, $f[x] = V(x)$. As $x \in D_f$, $f[x] \subseteq D_{f'}$ by definition and $D_{f'} \subseteq D_{\mop(f')}$ by Lemma \ref{lemma1}, so for every element $v \in V(x)$ it is the case that $\op(v) = \mop(f')[v]$. Since $F$ is either a map or a filter, the equality above follows.

\end{enumerate}

\end{proof}

\recallcorollary{corollary}{\corollaryBody}
\begin{proof}
\[
f \coloneqq [ v \in \{x\} \mapsto V(x) ]
\]
obviously satisfies the precondition of Theorem \ref{thm}:
\[
\forall y \in D_f \ .\ f[y] = V(y)
\]
Since $x \in D_f \subseteq D_{\mop(f)}$ by Lemma~\ref{lemma1}, we can use Theorem~\ref{thm}, to conclude that $\mop(f)[x] = \op(x)$.
\end{proof}

