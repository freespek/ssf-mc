\usepackage{mathpartir}
\usepackage{amsthm}
\usepackage{amsmath}
\usepackage{amssymb}
\usepackage{mathtools}
\usepackage{listings}
\usepackage{tlalatex}

\lstdefinelanguage{tla}
  {morekeywords={MODULE,EXTENDS,CONSTANTS,CONSTANT,ASSUME,VARIABLES,VARIABLE,
          EXCEPT,UNCHANGED,TRUE,FALSE,IF,THEN,ELSE,LET,IN,SUBSET,DOMAIN,RECURSIVE},
          comment=[l]{\\*},
  morecomment=[s]{(*}{*)},
  mathescape=true,escapechar={@},
  basicstyle=\sffamily,%\small,
  commentstyle=\itshape\rmfamily,%\small,
  keywordstyle=\sffamily%\bfseries
}
\lstdefinelanguage{alloy}
  {morekeywords={pred,sig,fact,set,one,extends,all,or,and,run,for,but},
          comment=[l]{---},
  mathescape=true,escapechar={@},
  basicstyle=\sffamily,%\small,
  commentstyle=\itshape\rmfamily,%\small,
  keywordstyle=\sffamily\bfseries
}

\newtheorem{theorem}{Theorem}[section]
\newtheorem{lemma}[theorem]{Lemma}
\newtheorem{corollary}[theorem]{Corollary}

\newcommand{\iteDef}[4]{
  #1 \coloneqq \left\{
\begin{array}{ll}
      #2 &; #3 \\
      #4 &; \text{otherwise}\\
\end{array} 
\right. 
}

\newcommand{\tlap}{$\textsc{TLA}^{+}$}
% \newcommand{\defeq}{\;\mathrel{\smash   %% keep this symbol from being too tall
%     {{\stackrel{\scriptscriptstyle\Delta}{=}}}}\;}

\newcommand{\nat}{\mathbb N_0}

\newcommand{\op}{\mathrm{R}}
\newcommand{\nrop}{\mathrm{I}}
\newcommand{\mop}{\mathrm{R}_m}
\newcommand{\mapg}{\mathrm{G}_m}
\newcommand{\bb}{\mathrm{next}}
\newcommand{\Chain}{\mathrm{Stack}}

\newcommand{\tup}[1]{\left<\left<#1\right>\right>}
\newcommand{\htau}{\hat{\tau}}

\newcommand{\List}{\mathrm{List}}
\newcommand{\Seq}{\mathrm{Seq}}
\newcommand{\Set}{\mathrm{Set}}
\newcommand{\PVec}{\mathrm{PVec}}
\newcommand{\PSet}{\mathrm{PSet}}
\newcommand{\PMap}{\mathrm{PMap}}
\newcommand{\Concat}{\mathrm{Concat}}
\newcommand{\Callable}{\mathrm{Callable}}
\newcommand{\Le}{\mathrm{Le}}
\newcommand{\bool}{\mathrm{bool}}
\newcommand{\Bool}{\mathrm{Bool}}
\newcommand{\pyint}{\mathrm{int}}
\newcommand{\Int}{\mathrm{Int}}
\newcommand{\ApaFoldSet}{\mathrm{ApaFoldSet}}
\newcommand{\ApaFoldSeqLeft}{\mathrm{ApaFoldSeqLeft}}
\newcommand{\MkSeq}{\mathrm{MkSeq}}
\newcommand{\SetAsFun}{\mathrm{SetAsFun}}
\newcommand{\Push}{\mathrm{Push}}
\newcommand{\At}{\mathrm{At}}
\newcommand{\Indices}{\mathrm{Indices}}
% \newcommand{\UNION}{\mathrm{UNION}}
% \newcommand{\CHOOSE}{\mathrm{CHOOSE}}
% \newcommand{\TRUE}{\mathrm{TRUE}}
% \newcommand{\FALSE}{\mathrm{FALSE}}
% \newcommand{\IF}{\mathrm{IF}}
% \newcommand{\THEN}{\mathrm{THEN}}
% \newcommand{\ELSE}{\mathrm{ELSE}}
% \newcommand{\LET}{\mathrm{LET}}
% \newcommand{\IN}{\mathrm{IN}}
% \newcommand{\DOMAIN}{\mathrm{DOMAIN}}
% \newcommand{\EXCEPT}{\mathrm{EXCEPT}}
% \newcommand{\RECURSIVE}{\mathrm{RECURSIVE}}



% recall a theorem mentioned before with the same label
\newcommand{\recallthm}[2]{%
  {\medskip\noindent\bfseries Theorem~\ref{#1}.~}{\itshape #2}
}
% recall a proposition mentioned before with the same label
\newcommand{\recallproposition}[2]{%
  {\medskip\noindent\bfseries Proposition~\ref{#1}.~}{\itshape #2}
}
% recall a lemma mentioned before with the same label
\newcommand{\recalllemma}[2]{%
  {\medskip\noindent\bfseries Lemma~\ref{#1}.~}{\itshape #2}
}
% recall a corollary mentioned before with the same label
\newcommand{\recallcorollary}[2]{%
  {\medskip\noindent\bfseries Corollary~\ref{#1}.~}{\itshape #2}
}
