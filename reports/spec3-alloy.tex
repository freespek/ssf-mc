%! TeX root = report.tex

\section{\SpecThreeB{} in Alloy}\label{sec:alloy}

Once we saw that our specifications~\SpecThree{} and~\SpecFour{}
were too challenging for the \tlap{} tools, we have decided to employ
Alloy~\cite{jackson2012software,alloytools}, as an alternative to our tooling.
Alloy has two features that are attractive for our project:

\begin{itemize}

  \item Alloy allows to precisely control the search scope by setting the
      number of objects of particular type in the universe. For example, we can
      restrict the number of checkpoints and votes to 5 and 12, respectively.
      Although we introduced similar restrictions with Apalache, Alloy has even
      finer level of parameter tuning.

  \item Alloy translates the model checking problem to a Boolean satisfiability
      problem (SAT), in constrast to Apalache, which translates it to satisfiability-modulo-theory (SMT).  This
      allows us to run the latest off-the-shelf solvers such as Kissat, the
      winner of the SAT Competition 2024~\cite{SAT-Competition-2024-solvers}.

\end{itemize}

\subsection{From \tlap{} to Alloy}

Having~\SpecThree{} in~\tlap{}, it was relatively easy for us to write down
the Alloy specification, as~\SpecThree{} is already quite abstract. However, as
Alloy's core abstractions are objects and relations between them, the
specification looks quite differently. For example, here is how we declare
signatures for the blocks and checkpoints:

\begin{lstlisting}[language=alloy,columns=fullflexible]
  sig Payload {}
  sig Signature {}
  fact atLeastFourSignatures { #Signature >= 4 }

  sig Block {
    slot: Int,
    body: Payload,
    parent: Block
  }
  sig Checkpoint {
    block: Block,
    slot: Int
  }

  one sig GenesisPayload extends Payload {}
  one sig GenesisBlock extends Block {}
\end{lstlisting}

In contrast to~\SpecThree{}, our Alloy specification does not describe a
state machine, but it specifies an arbitrary single state of the protocol. As
in~\SpecThree{}, we compute the set of justified checkpoints as a fixpoint:

\begin{lstlisting}[language=alloy,columns=fullflexible]
  one sig JustifiedCheckpoints {
    justified: set Checkpoint
  }
  fact justifiedCheckpointsAreJustified {
    all c: JustifiedCheckpoints.justified |
      c.slot = 0 or 3.mul[#justifyingVotes[c].validator] >= 2.mul[#Signature]
  }
\end{lstlisting}

\subsection{Model Checking with Alloy}

Similar to our experiments with Apalache, we produce examples of configurations
that satisfy simple properties. For example:

\begin{lstlisting}[language=alloy,columns=fullflexible]
  run someFinalizedCheckpoint { some c: Checkpoint |
    isFinalized[c] and c.slot != 0
  }
  for 10 but 6 Block, 6 Checkpoint, 12 Vote, 5 int
\end{lstlisting}

For small search scopes, Alloy quickly finds examples, often in a matter
of seconds.

Ultimately, we are interested in showing that~$\textit{AccountableSafety}$
holds true. To this end, we want Alloy to show that the
formula~$\textit{noAccountableSafety}$ does not have a model (unsatisfiable).
Similar to Apalache, Alloy allows us to show safety for bounded scopes.

\begin{lstlisting}[language=alloy,columns=fullflexible]
  pred noAccountableSafety {
    disagreement and (3.mul[#slashableNodes] < #Signature)
  }
  run noAccountableSafety for 10 but 6 Block, 6 Checkpoint,
                                     4 Signature, 6 FfgVote, 15 Vote, 5 int
\end{lstlisting}

Thus it is interesting to increase these parameter values as much as possible
to cover more scenarios, while keeping them small enough that the SAT solver
still terminates within reasonable time. Table~\ref{tab:alloy-mc} summarizes
our experiments with Alloy when checking $\textit{noAccountableSafety}$ with
Kissat.  Since a vote includes an FFG vote and a validator's signature, the
number of FFG votes and validators bound the number of votes. Hence, if we only
increase the number of votes without increasing the number of FFG votes, we may
miss the cases when longer chains of justifications must be constructed.

\subsection{Model Checking Results}\label{sec:alloy-results}

As we can see, we have managed to show \textit{AccountableSafety} for chain
configurations of up to 6 blocks, including the genesis block. While the
configurations up to 4 blocks are important to analyze, they do not capture the
general case, namely, of two finalized checkpoints building on top of justified
checkpoints that are not the genesis. Thus, configurations of 5 or more blocks
are the most important ones, as they capture the general case.

To see if we could go a bit further beyond 5 blocks, we have tried a
configuration with 7 blocks. This configuration happened to be quite challenging
for Alloy and Kissat. We have run Kissat for over 16 days, and it was stuck at
the remaining seven percent for many days. At that point, it has introduced:

\begin{itemize}
  \item 60 thousand Boolean variables,
  \item 582 thousand ``irredundant'' clauses and 199 thousand binary clauses,
  \item 1.3 billion conflicts,
  \item 56 million restarts.
\end{itemize}

We conjecture that the inductive structure of justified and finalized
checkpoints make it challenging for the solvers (both SAT and SMT) to analyze
the specification. Essentially, the solvers have to reason about chains of
checkpoints on top of chains of blocks.

\begin{table}
    \centering
    \begin{tabular}{rrrrrrrr}
        \tbh{\#}
            & \tbh{blk}
            & \tbh{chk}
            & \tbh{sig}
            & \tbh{ffg}
            & \tbh{votes}
            & \tbh{Time}
            & \tbh{Memory}
            \\ \toprule
        1 & 3 & 5 & 4 & 5  & 12 & 4s & 35M
            \\
        2 & 4 & 5 & 4 & 5 & 12 & 10s & 40M
            \\
        3 & 5 & 5 & 4 & 5 & 12 & 15s & 45M
            \\
        4 & 3 & 6 & 4 & 6 & 15 & 57s & 52M
            \\
        5 & 4 & 6 & 4 & 6 & 15 & 167s & 55M
            \\
        6 & 5 & 6 & 4 & 6 & 15 & 245s & 57M
            \\
        7 & 6 & 6 & 4 & 6 & 15 & 360s & 82M
            \\
        8 & 5 & 7 & 4 & 6 & 24 & 1h 27m & 156M
            \\
        9 & --- & --- & --- & --- & --- & --- & ---
            \\
        10 & 3 & 15 & 4 & 5 & 12 & 31s & 56M
            \\
        11 & 4 & 20 & 4 & 5 & 12 & 152s & 94M
            \\
        12 & 5 & 25 & 4 & 5 & 12 & 234s & 117M
            \\
        13 & 7 & 15 & 4 & 10 & 40 & over 16 days (timeout) & 300 MB
            \\ \bottomrule
    \end{tabular}
    \caption{Model checking experiments with Alloy and Kissat:
      \textbf{blk} is the number of blocks, \textbf{chk} is the number
      of checkpoints, \textbf{sig} is the number of validator signatures,
      \textbf{ffg} is the number of FFG votes, \textbf{votes} is the number
      of validator votes.
    }\label{tab:alloy-mc}
\end{table}

