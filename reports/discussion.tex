%! TeX root = report.tex

We have presented a series of specifications modeling the 3SF protocol from
various perspectives. Initially, we developed a direct translation of the
protocol's Python specification into \tlap{}, but this approach proved
unsatisfactory due to the reliance on recursion. To address this, we modified
the specification to use folds in place of recursion, theoretically enabling
model-checking with Apalache. However, this approach also proved impractical
due to the high computational complexity involved. Subsequently, we applied a
series of abstractions to improve the model-checking efficiency.

In addition to the \tlap{} specifications, we also introduced an SMT encoding
and an Alloy specification. The SMT encoding proved to be fairly performant,
while the Alloy specification demonstrated exceptional performance in
combination with the Kissat solver~\cite{SAT-Competition-2024-solvers}.

We summarize the key outcomes of the project:

\paragraph{Exhaustive checking of \textit{AccountableSafety}.} Our primary
objective was to verify the \textit{AccountableSafety} property of the 3SF
protocol. Model-checking this property proved to be computationally challenging
due to the unexpectedly high combinatorial complexity of the protocol.
Nonetheless, we performed systematic experiments across various specifications
in \tlap{}, Alloy, and SMT (CVC5), representing both a direct translation and
different levels of abstraction of the protocol. The largest instances we
exhaustively verified to satisfy \textit{AccountableSafety} include up to 7
checkpoints and 24 validator votes (see Table~\ref{tab:alloy-mc} in
Section~\ref{sec:alloy-results}). This comprehensive verification gives us
absolute confidence that the modeled protocol satisfies
\textit{AccountableSafety} for systems up to this size.

\paragraph{No falsification of \textit{AccountableSafety}.} In addition to the
instances where we conducted exhaustive model-checking, we ran experiments on
larger instances, which exceeded generous time limits and resulted in timeouts.
Even in these cases, no counterexamples to \textit{AccountableSafety} were
found. Furthermore, in instances where we deliberately introduced bugs into the
specifications (akin to mutation testing), Apalache, Alloy and CVC5 quickly
generated counterexamples. This increases our confidence that the protocol
remains accountably safe, even for system sizes substantially larger than
those we were able to exhaustively verify.

\paragraph{Advantages of human expertise over automated translation.} Applying
translation rules to derive checkable specifications from existing artifacts can
serve as a valuable starting point. However, such translations often introduce
inefficiencies because they cannot fully capture the nuances of the specific
context. This can result in suboptimal performance. Therefore, while
translations provide a baseline, manually crafting specifications from the
outset is usually more effective. When relying on translated specifications, it is
essential to apply manual optimizations to ensure both accuracy and efficiency.

\paragraph{Value of \tlap{}.} \tlap{} is a powerful language for specifying and
verifying distributed systems. Although our most promising experimental results
were derived from the Alloy specification, the insights gained through
iterative abstraction in \tlap{} were indispensable.\ \tlap{} enabled us to
start with an almost direct translation of the Python code and progressively
refine it into higher levels of abstraction. This iterative process provided a
deeper understanding of the protocol and laid the groundwork for the more
efficient Alloy specification. The connection between the Python specification
and the Alloy specification is definitely less obvious than the tower of
abstractions that we have built in~\tlap{}.

\paragraph{Value of producing examples.} Even though checking accountable
safety proved to be challenging, our specifications are not limited to proving
only accountable safety. They are also quite useful for producing examples. For
instance, both Apalache and Alloy are able to quickly produce examples of
configurations that contain justified and finalized checkpoints. We highlight
this unique value of specifications that are supported by model checkers:

\begin{itemize}

  \item Executable specifications in Python require the user to provide program
    inputs. These inputs can be also generated randomly, though in the case of
    3SF, this would be challenging: We expect the probability of hitting
    ``interesting'' values, such as producing finalizing checkpoints, to be
    quite low.

  \item Specifications in the languages supported by proof systems usually do
    not support model finding. The TLAPS Proof System is probably a unique
    exception here, as~\tlap{} provides a common playground for the prover and the
    model checkers~\cite{KonnovKM22}.

\end{itemize}

