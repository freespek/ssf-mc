%! TeX root = report.tex

\begin{tikzpicture}[node distance=2cm, >=latex]
    \tikzset{
        mynode/.style={%
            draw, rectangle, minimum width=3cm, minimum height=1cm, align=center,
            rounded corners
        }
    };

    % Define the nodes
    \node[mynode, minimum width=2cm, minimum height=1cm] (py)
        {\small\texttt{ffg.py}};

    \node[mynode, minimum width=4cm,
        minimum height=1cm, below=1cm of py] (spec1)
        {\textbf{\textsf{Spec 1}}: {\small\texttt{spec1-2/ffg\_recursive.tla}}};

    \node[mynode, minimum width=3cm,
        minimum height=1cm, right=1cm of spec1] (spec2)
        {\textbf{\textsf{Spec 2}}: {\small\texttt{spec1-2/ffg.tla}}};

    \node[mynode, minimum width=3cm,
        minimum height=1cm, below=1cm of spec2] (spec3)
        {\textbf{\textsf{Spec 3}}: {\small\texttt{spec3/ffg.tla}}};

    \node[mynode, minimum width=3cm,
        minimum height=1cm, left=1cm of spec3] (spec4)
        {\textbf{\textsf{Spec 4}}: {\small\texttt{spec4/ffg\_inductive.tla}}};

    \node[mynode, minimum width=3cm,
        minimum height=1cm, below=1cm of spec4] (spec4b)
        {\textbf{\textsf{Spec 4b}}};

    \node[mynode, minimum width=3cm,
        minimum height=1cm, below left=1cm and -2cm of spec3] (spec3b)
        {\textbf{\textsf{Spec 3b}}: SMT};

    \node[mynode, minimum width=3cm,
        minimum height=1cm, below right=1cm and -2cm of spec3] (spec3c)
        {\textbf{\textsf{Spec 3c}}: Alloy/SAT };

    \draw[->] (py) -- (spec1);
    \draw[->] (spec1) -- (spec2);
    \draw[->] (spec2) -- (spec3);
    \draw[->] (spec3) -- (spec4);
    \draw[->] (spec3) -- (spec3b);
    \draw[->] (spec3) -- (spec3c);
    \draw[->] (spec4) -- (spec4b);

\end{tikzpicture}
