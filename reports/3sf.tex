%! TeX root = report.tex

\section{Basic 3SF concepts}\label{sec:3sf}

In this section, we summarize the basic concepts of the 3SF protocol that this project depends on.
We refer the reader to the 3SF paper~\cite{d20243} for a more comprehensive explanation.

\paragraph*{Validators.} Participants of the protocol are referred to as \emph{validators} with $n$ being their total number.
Every validator is identified by a unique cryptographic identity and the public keys are common knowledge. 
Each validator has a \emph{stake} but for the purpose of this project each validator's stake is set to~1.
% If a validator $v_i$ misbehaves and a proof of this misbehavior is given, it loses a part of its stake ($v_i$ gets \emph{slashed}). 

\paragraph*{Slots.} Time is divided into \emph{slots}. 
In ideal conditions a new block (see below) is expected to be proposed at the beginning of each slot.

\paragraph{Blocks and Chains.} 
A \emph{block} is a pair of elements, denoted as \( B = (b,p) \). Here, \( b \) represents the \emph{block body} -- essentially, the main content of the block which contains a batch of transactions grouped together.
Each block body contains a reference pointing to its \emph{parent} block. 
The second element of the pair, \( p \geq 0 \), indicates the \emph{slot} where the block \( B \) is proposed.
By definition, if $B_p$ is the parent of $B$, then $B_p.p < B.p$.
The \emph{genesis block} is the only block that does not have a parent. Its slot is \( p = 0 \).
Given the definition above, each different block $B$ implicitly identifies a different finite \emph{chain} of blocks starting from block $B$, down to the genesis block, by recursively moving from a block to its parent.
Hence, there is no real distinction between a block and the chain that it identifies.
So, by chain $\chain$, we mean the chain identified by the block $\chain$.
We write $\chain_1 \preceq \chain_2$ to mean that $\chain_1$ is a non-strict prefix of $\chain_2$.
We say that $\chain_1$ \emph{conflicts} with $\chain_2$ if and only if neither $\chain_1 \preceq \chain_2$ nor $\chain_2 \preceq \chain_1$ holds. 

\paragraph{Checkpoints.}
In the protocol described in~\cite{d20243}, a \emph{checkpoint} is a tuple $(\chain, c)$, where \(\chain \) is a chain and \( c \) is a slot signifying where \( \chain \) is proposed for justification (this concept is introduced and explained below).
However, for efficiency reasons, in the specification targeted by this project, a \emph{valid} checkpoint $\C$ is a triple $(H, c, p)$ where $H$ is the hash of a chain $\chain$, $c$ is the slot at which chain $\chain$ is proposed for justification, as per the definition above, and $p =  \chain.p$.
The total pre-order among checkpoints is defined:
$\C \leq \C'$ if and only if either \(\C.c < \C'.c\) or, in the case where \(\C.c = \C'.c\), then \(\C.p \leq \C'.p\). 
Also, $\C < \C'$ means $\C \leq \C' \land \C \neq \C'$.


\paragraph*{FFG Votes.}
Validators cast two main types of votes: \textsc{ffg-vote}s and \textsc{vote}s. 
Each \textsc{vote} includes an \textsc{ffg-vote}.
The specification targeted by this project only deals with \textsc{ffg-vote}s as the extra information included in  \textsc{vote}s has no impact on AccountableSafety.
An \textsc{ffg-vote} is represented as a tuple \([\textsc{ffg-vote}, \C_1, \C_2, v_i]\), where {$v_i$ is the validator sending the \textsc{ffg-vote}\footnote{Digital signatures are employed to ensure that $v_i$ is the actual sender and it is assumed that such digital signatures are unforgeable.}, while} \(\C_1\) and \(\C_2\) are checkpoints.
These checkpoints are respectively referred to as the \emph{source} (\(\C_1\)) and the \emph{target} (\(\C_2\)) of the \textsc{ffg-vote}.
Such an \textsc{ffg-vote} is \emph{valid}
%(i.e., $\mathrm{valid}(\C_1 \to \C_2$)) 
if and only if both checkpoints are valid, 
\(\C_1.c < \C_2.c\) and  \(\C_1.\chain \preceq \C_2.\chain\). 
\textsc{ffg-vote}s effectively act as \emph{links} connecting the source and target checkpoints. Sometimes the whole \textsc{ffg-vote} is simply denoted as \(\C_1 \to \C_2\).

\paragraph*{Justification.}
A set of \textsc{ffg-vote}s is a \emph{supermajority set} if it contains valid \textsc{ffg-vote}s from at least \(\frac{2}{3}n\) distinct validators.
A checkpoint \(\C\) is considered \emph{justified} if it either corresponds to the genesis block, i.e., \(\C = (B_\text{genesis}, 0)\), or if there exists a supermajority set of links \(\{\C_i \to \C_j\}\) satisfying the following conditions. First,  for each link $\C_i \to \C_j$ in the set, {\(\C_i \to \C_j\) is valid and} \(\C_i.\chain \preceq \C.\chain \preceq \C_j.\chain\). Moreover, all source checkpoints \(\C_i\) in these links need to be already justified, and the checkpoint slot of \(\C_j\) needs to be the same as that of \(\C\) (\(\C_j.c=\C.c\)), for every \(j\). It is important to note that the source and target chain may vary across different votes.
% This justification rule is formalized by the binary function $\mathsf{J}(\V,\C)$ in Algorithm~\ref{alg:justification-finalization} which outputs \textsc{true} if and only if checkpoint $\C$ is justified according to the set of messages $\V$.
% Lastly, we say that a {chain \(\chain\)} \emph{is justified} if and only if there exists a justified checkpoint \(\C\) for which {\(\C.\chain = \chain\)}.

% \paragraph*{Greatest justified checkpoint and chain.}
% A checkpoint is considered justified in a view \(\V\) if \(\V\) contains a supermajority set of links justifying it. A \emph{justified} checkpoint which is maximal in \(\V\) with respect to the previously defined lexicographic ordering is referred to as the \emph{greatest justified checkpoint} in \(\V\), denoted as \(\GJ(\V)\). In the event of ties, they are broken arbitrarily. Chain \(\GJ(\V).\chain\) is referred to as the \emph{greatest justified chain}.

\paragraph*{Finality.}
A checkpoint \(\C\) is \emph{finalized} if it is justified and there exists a supermajority link with source \(\C\) and potentially different targets \(\C_j\) where \(\C_j.c = \C.c + 1\). A chain \(\chain\) is finalized if there exists a finalized checkpoint \(\C\) with \(\chain = \C.\chain\). The checkpoint \(\C = (B_\text{genesis}, 0)\) is finalized by definition.
% This finalization rule is formalized by the binary function $\mathsf{F}(\V,\C)$ in Algorithm~\ref{alg:justification-finalization} which outputs \textsc{true} if and only if checkpoint $\C$ is finalized according to the set of messages $\V$.
% {Given a view $\V$, a finalized checkpoint which is maximal in \(\V\) with respect to the previously defined lexicographic ordering is referred to as the \emph{greatest finalized checkpoint} in \(\V\), denoted as \(\GF(\V)\).
% In the event of ties, they are broken arbitrarily.
% % We say that a chain 
% We say that a chain $\chain$ is finalized according to a view $\V$ if and only if $\chain \preceq \GF(\V).\chain$.


\paragraph*{Slashing.}
A validator \(v_i\) is subject to slashing for sending two \emph{distinct} \textsc{ffg-vote}s \(\C_1 \to \C_2\) and \(\C_3 \to \C_4\) if either of the following conditions holds: {\(\mathbf{E_1}\) (Double voting)} if \(\C_2.c = \C_4.c\), implying that a validator must not cast distinct \textsc{ffg-vote}s for the same checkpoint slot; or {\(\mathbf{E_2}\) (Surround voting)} if \(\C_3 < \C_1\)
 and \(\C_2.c < \C_4.c\), indicating that a validator must not vote using a lower checkpoint as source and must avoid voting within the span of its other votes.


 \begin{definition}[AccountableSafety]
  \label{def:acc-safety}
  AccountableSafety holds if and only if, upon two conflicting chains being finalized, 
 {by having access to all messages sent,} it is possible to slash at least $\frac{n}{3}$ of the validators.
  % Finally, we also say that a chain is $f$-\emph{accountable} if the protocol outputting it has Accountable Safety with resilience $f$.
  % If a protocol $\Pi$ comprises $k$ subprotocols, $\Pi_i$, each of which outputs a chain, $\Chain_i$, we say that $\Chain_i$ is $f$-accountable if $\Pi_i$ is $f$-accountable.
\end{definition}

\subsection{Complexity of (model-checking) the protocol}

The 3SF protocol is intricate, with a high degree of combinatorial complexity,
making it challenging for automatic model checking. During our specification
work, we have observed multiple layers of complexity in the protocol:
\begin{itemize}
  \item The Python specification considers all possible graphs over all proposed
    blocks. From graph theory~\cite{cayley1878theorem}, we know that the number
    of labelled rooted forests on $n$ vertices is ${(n+1)}^{n-1}$. (Observe that
    this number grows faster than the factorial~$n!$.) This is the number of
    possible block graphs that the model checker has to consider for $n$ blocks.
  \item The protocol introduces a directed graph of checkpoints (pairs $(b,n)$
    of a block $b$ and an integer $n$) \emph{on top} of the block graph.
    Validator-signed votes form a third labeled directed graph over pairs of
    checkpoints. In addition, all of these edges have to satisfy arithmetic
    constraints.
  \item Justified and finalized checkpoints introduce an inductive structure
    that the model checker has to reason about. Essentially, the solvers have to
    reason about chains of checkpoints on top of chains of blocks.
  \item Finally, the protocol introduces set cardinalities, both for determining a
    quorum of validators and as a threshold for \textit{AccountableSafety}.
    Cardinalities are known to be a source of inefficiency in automated
    reasoning.
\end{itemize}
