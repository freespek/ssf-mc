%! TeX root = report.tex

\section{M2: Spec 2: Fold-Based Specification}

\SpecTwo{} addresses some of the limitations inherent in the straightforward
translation of \SpecOne{} from the Python executable specification.

\subsection{Translating Recursive \tlap{} Operators}

The primary goal of \SpecTwo{} is to maintain the semantic structure of
\SpecOne{} while eliminating recursion. In \SpecOne{}, (mutually) recursive
operators model key aspects of protocol behavior, such as the block tree and
block justification. Apalache does not natively support recursive
operators\footnote{\url{https://apalache-mc.org/docs/apalache/principles/recursive.html}},
thus it cannot be used immediately to model-check \SpecOne{}. While the
explicit-state \tlap{} model checker TLC supports recursive operators, it does
not scale to model-checking of this problem.

To resolve this, we reformulate \SpecOne{} into \SpecTwo{}, by substituting
(mutually) recursive constructs with bounded
\texttt{fold}~operations\footnote{In functional programming, \texttt{fold} is a
higher-order function that accepts a combining operation and an iterable data
structure, and applies the operation to each element of the data structure
to compute a single return value. \texttt{fold} is also known as
\texttt{reduce} in some languages.}, which enable the same iterative
computations to be performed in a non-recursive manner. Consider the following
example from \SpecOne{}:

\begin{lstlisting}[language=tla]
RECURSIVE is_ancestor_descendant_relationship(_, _, _)
is_ancestor_descendant_relationship(ancestor, descendant, node_state) ==
    IF ancestor = descendant THEN TRUE
    ELSE IF descendant = node_state.configuration.genesis THEN FALSE
    ELSE
        /\ has_parent(descendant, node_state)
        /\ is_ancestor_descendant_relationship(ancestor, get_parent(descendant, node_state), node_state)
\end{lstlisting}

Its corresponding fold-based formulation is shown below:

\begin{lstlisting}[language=tla]
is_ancestor_descendant_relationship(ancestor, descendant, node_state) ==
    LET FindAncestor(last_block_and_flag, slot) ==
        LET
            last_block == last_block_and_flag[1]
            flag == last_block_and_flag[2]
        IN
        IF flag THEN Pair(last_block, TRUE)
        ELSE IF last_block = node_state.configuration.genesis \/ ~has_parent(last_block, node_state) THEN Pair(last_block, FALSE)
        ELSE LET parent == get_parent(last_block, node_state) IN Pair(parent, parent = ancestor)
    IN
    ApaFoldSeqLeft(FindAncestor, Pair(descendant, descendant = ancestor), MkSeq(MAX_SLOT, LAMBDA i: i))[2]
\end{lstlisting}

\subsection{An Optimization: Flattening Nested Folds}

Initial model checking experiments with \SpecTwo{} revealed significant
challenges related to memory consumption, stemming from the high number of SMT
constraints emitted by Apalache for nested fold operations, which in turn mirror
the complexity of the original nested recursive structures from the Python
specification.

To address these issues, we introduce a manual optimization strategy that
involves flattening nested fold operations. This technique transforms nested
folds into a more manageable structure by employing additional \tlap{} state
variables, similar to memoization or prophecy variables.

For example, we introduce a new \tlap{} state variable
\texttt{PRECOMPUTED\_IS\_ANCESTOR\_DESCENDANT\_RELATIONSHIP} to store
precomputed ancestor-descendant relationships and initialize it with the results
of the fold operation above:

\begin{lstlisting}[language=tla]
LET all_blocks == get_all_blocks(single_node_state) IN
PRECOMPUTED__IS_ANCESTOR_DESCENDANT_RELATIONSHIP =
    [ descendant \in all_blocks |-> { ancestor \in all_blocks : is_ancestor_descendant_relationship(ancestor, descendant, single_node_state) } ]
\end{lstlisting}

Instead of re-evaluating the fold operation each time we need to check if two
blocks are in an ancestor-descendant relationship, we can directly access the
memoized result in a much more efficient map lookup:

\begin{lstlisting}[language=tla]
are_conflicting(chain1, chain2, node_state) ==
    /\ chain1 \notin PRECOMPUTED__is_ancestor_descendant_relationship[chain2]
    /\ chain2 \notin PRECOMPUTED__is_ancestor_descendant_relationship[chain1]
\end{lstlisting}

To further improve our confidence in the correctness of this optimization, we
could produce a proof in TLAPS or run Apalache to show functional equivalence.

\subsection{Checking the Specification}

We can query the specification for reachable protocol states using Apalache.
For example, we can check if a nontrivial finalized checkpoint exists by writing an
invariant that we expect not to hold. If the invariant below is violated,
Apalache will produce an example of a finalized checkpoint as a counterexample:

\begin{lstlisting}[language=tla]
\* Find a finalized checkpoint (in addition to the genesis checkpoint)
FinalizedCheckpoint_Example ==
    get_finalized_checkpoints(single_node_state) = { genesis_checkpoint(single_node_state) }
\end{lstlisting}

Obviously, we can also check \textit{AccountableSafety} by supplying it as an
invariant to Apalache:

\begin{lstlisting}[language=tla]
AccountableSafety ==
  LET
    finalized_checkpoints == get_finalized_checkpoints(single_node_state)
    finalized_blocks == { get_block_from_hash(checkpoint.block_hash, single_node_state) : checkpoint \in finalized_checkpoints }
    there_are_conflicting_finalized_blocks == \E block1, block2 \in finalized_blocks : are_conflicting(block1, block2, single_node_state)
    slashable_nodes == get_slashable_nodes(single_node_state.view_votes)
  IN there_are_conflicting_finalized_blocks => Cardinality(slashable_nodes) * 3 >= Cardinality(Nodes)
\end{lstlisting}

Table~\ref{tab:spec2} shows the results of model checking \SpecTwo{} with
Apalache. We can see that generating examples of reachable protocol states and
verifying \textit{AccountableSafety} is infeasible due to the high computational
complexity of the specification.

\begin{table}
    \centering
    \begin{tabular}{ll}
        \tbh{Property} & \tbh{Time} \\ \toprule
        Example: conflicting blocks & timeout ($>40$h) \\
        Example: finalized \& conflicting blocks & timeout ($>40$h) \\
        AccountableSafety & timeout ($>40$h) \\ \bottomrule
    \end{tabular}
    \caption{Model checking \SpecTwo{} with Apalache.}\label{tab:spec2}
\end{table}

These results are not surprising -- the solver has to consider both reachability
properties for all possible block graphs, and all possible FFG voting scenarios
on top of these graphs. To further evaluate \SpecTwo{}, we fix the block graph
-- this way the solver only has to reason about voting. We encode three example
block graphs: a single, linear chain (\texttt{SingleChain}), a minimal forked
chain of three blocks (\texttt{ShortFork}), and a forest of disconnected chains
(\texttt{Forest}). Table~\ref{tab:spec2_fixed} shows the results of model
checking \SpecTwo{} for these fixed block graphs.

\begin{table}
    \centering
    \begin{tabular}{llr}
      \tbh{Property} & \tbh{Block graph} & \tbh{Time} \\ \toprule
      Example: conflicting blocks & \texttt{SingleChain} & 1 min 3 sec \\
      Example: conflicting blocks & \texttt{ShortFork} & 52 sec \\
      Example: conflicting blocks & \texttt{Forest} & 2 min 21 sec \\ \midrule
      Example: fin.\ \& confl.\ blocks & \texttt{SingleChain} & 1 min 5
      sec \\
      Example: fin.\ \& confl.\ blocks & \texttt{ShortFork} & 10 hours
      49 min 47 sec \\
      Example: fin.\ \& confl.\ blocks & \texttt{Forest} & timeout
      ($>40$h) \\ \midrule
      AccountableSafety & \texttt{SingleChain} & 1 min 13 sec \\
      AccountableSafety & \texttt{ShortFork} & timeout ($>40$h) \\
      AccountableSafety & \texttt{Forest} & timeout ($>40$h) \\ \bottomrule
    \end{tabular}
    \caption{Model checking \SpecTwo{} for fixed block
    graphs.}\label{tab:spec2_fixed}
\end{table}

We can see that the solver can handle the single chain block graph (where
\textit{AccountableSafety} trivially holds due to absence of conflicting
blocks), but struggles with the more complex scenarios even when given a fixed
block graph. This suggests that the complexity inherent in the specification is
due to the high combinatorial complexity of voting scenarios, rather than just
the block graph.

\subsection{Discussion}~\footnote{\color{red}TODO: perhaps move this to a general
discussion section?} Applying translation rules to obtain checkable
specifications from other artifacts can provide a useful starting point.
However, such translations are also likely to introduce unwanted
inefficiencies. Translation rules cannot account for the nuances of the specific
context, which can lead to suboptimal performance. Consequently, leveraging
human ingenuity to develop specifications from the outset can be more effective.
When relying on translated specifications, it becomes crucial to conduct
manual optimizations to ensure efficiency.
