%! TeX root = report.tex

\section{Introduction}

At the time of writing this report, Ethereum is using
Gasper~\cite{buterin2020combining} as the underlying consensus protocol.
In Gasper, time is divided into slots, which represent intervals during which a new block is proposed to extend the blockchain and undergoes voting. 
Finalizing a block--ensuring that it is permanently added to the blockchain and cannot be reversed--typically requires 64 to 95 slots.
This delay in finality makes the network more vulnerable to potential
block reorganizations when the network conditions change, 
e.g., during periods of asynchronous network conditions.
In particular, this finalization delay heightens the network’s exposure to
Maximal Extractable Value (MEV) exploits, 
which could undermine the network’s integrity.
Additionally, the extended finalization period forces users to weigh the
trade-off between economic security and transaction speed.

To address these issues and speed up finality, D’Amato et al.~\cite{d20243} have recently introduced the \emph{3-slot-finality} (3SF) protocols 
for Ethereum that achieve finality within three slots after a proposal, hence realizing 3-slot finality.
This feature is particularly beneficial in practical scenarios where periods of synchrony and robust honest participation often 
last much longer than the time needed for finalization in the 3SF protocol.
Finally, the 3SF protocol enhances the practicality of large-scale blockchain networks by enabling the dynamically-available component, 
which handles honest participants who may go offline and come back online~\cite{pass2017sleepy}, 
to recover from extended asynchrony, provided at least two-thirds of validators remain honest and online for sufficient time. 

To that end, the 3SF protocols combine a partially synchronous finality gadget with two dynamically available consensus protocols – 
synchronous protocols that ensure safety and liveness even with fluctuating validator participation levels. 
This design is based on the \emph{ebb-and-flow} approach introduced in~\cite{neu2021ebb}. 
An ebb-and-flow protocol comprises two sub-protocols, each with its own confirmation rule, and each outputting a chain, with one serving 
as a prefix of the other. 
The first confirmation rule defines what is known as the \emph{available chain}, which provides liveness under dynamic participation
(and synchrony). 
The second confirmation rule defines the \emph{finalized chain}, and provides safety even under network partitions, but loses liveness 
either under asynchrony or in case of fluctuation in the participation level.

In this project, we targeted the 3SF specification in Python \footnote{URL to the Python specification:
\url{https://github.com/freespek/ssf-mc/3sf.txt}} as the case study focussing only on the finality gadget protocol, which is mostly specified in the file \texttt{ffg.py}.
Our main goal was to demonstrate 
\emph{Accountable Safety} of this protocol by the means of model checking. 
Accountable Safety is the property which ensures that if two conflicting chain (i.e. chains where neither is a prefix of the other) are finalized then, by having access to all messages sent, it is possible to identify at least 1/3 responsible participants. 

% \paragraph{}\colorbox{yellow}{TODO:}

% We have started this project with the Python specification of accountable
% safety in the 3SF Protocol\footnote{URL to the Python specification:
% \url{https://github.com/freespek/ssf-mc/3sf.txt}}. The main goal of the project
% was to demonstrate accountable safety of this protocol by the means of model
% checking.

We have chosen the specification language~\tlap{} and the model checker
Apalache for the following reasons.\ \tlap{} remains a goto language for
specifying consensus algorithms. Among the rich spectrum of
specifications~\cite{tla-examples}, the most notable for our project are the
specifications of Paxos~\cite{lamport2001paxos}, Raft~\cite{Ongaro14}, and
Tendermint~\cite{abs-1807-04938,TendermintSpec2020}. Since consensus algorithms
are quite challenging for classical model checkers like TLC, we choose
Apalache~\cite{Apalache2024,KT19,KonnovKM22}. This model checker utilizes the
SMT solver~Z3~\cite{MouraB08} in the background. Apalache was used for model
checking of agreement and accountable safety of
Tendermint~\cite{TendermintSpec2020}. As added benefit, four of the project
participants developed Apalache in the past and know its strenghts and
weaknesses.

\paragraph{Complexity of (model-checking) the protocol.} The 3SF protocol is
intricate, with a high degree of combinatorial complexity, making it challenging
to reason about. We have observed multiple layers of complexity in the protocol:
\begin{itemize}
  \item The Python specification considers all possible graphs over all proposed
    blocks. From graph theory~\cite{cayley1878theorem}, we know that the number
    of labelled rooted forests on $n$ vertices is ${(n+1)}^{n-1}$. (Observe that
    this number grows faster than the factorial~$n!$.) This is the number of
    possible block graphs that the model checker has to consider for $n$ blocks.
  \item The protocol introduces a directed graph of checkpoints (pairs $(b,n)$
    of a block $b$ and an integer $n$) \emph{on top} of the block graph.
    Validator-signed votes form a third labeled directed graph over pairs of
    checkpoints. In addition, all of these edges have to satisfy arithmetic
    constraints.
  \item Justified and finalized checkpoints introduce an inductive structure
    that the model checker has to reason about. Essentially, the solvers have to
    reason about chains of checkpoints on top of chains of blocks.
  \item Finally, the protocol introduces set cardinalities, both for determining a
    quorum of validators and as a threshold for \textit{AccountableSafety}.
    Cardinalities are known to be a source of inefficiency in automated
    reasoning.
\end{itemize}

\subsection{Key Outcomes}\label{sec:discussion}

% we embed the discussion right in the introduction
%! TeX root = report.tex

\section{Discussion}%
\label{sec:discussion}

We have presented a series of specifications modeling the 3SF protocol from
various perspectives. Initially, we developed a direct translation of the
protocol's Python specification into \tlap{}, but this approach proved
unsatisfactory due to the reliance on recursion. To address this, we modified
the specification to use folds in place of recursion, theoretically enabling
model-checking. However, this approach also proved impractical due to the high
computational complexity involved. Subsequently, we applied a series of
optimizations to improve the model's model-checking efficiency.

In addition to the \tlap{} specifications, we also introduced an SMT encoding
and an Alloy specification. The SMT encoding proved to be fairly performant,
while the Alloy specification demonstrated exceptional performance.

We summarize the key outcomes of the project:

\paragraph{Exhaustive checking of \textit{AccountableSafety}.} Our primary
objective was to verify the \textit{AccountableSafety} property of the 3SF
protocol. Model-checking this property proved to be computationally challenging
due to the unexpectedly high combinatorial complexity of the protocol.
Nonetheless, we performed systematic experiments across various specifications
in \tlap{}, Alloy, and SMT, representing both a direct translation and different
levels of abstraction of the protocol. The largest instances we exhaustively
verified to satisfy \textit{AccountableSafety} include up to 7 checkpoints and
24 validator votes (see Table~\ref{tab:alloy-mc}). This comprehensive
verification gives us absolute confidence that the modeled protocol satisfies
\textit{AccountableSafety} for systems up to this size.

\paragraph{No falsification of \textit{AccountableSafety}.} In addition to the
instances where we conducted exhaustive model-checking, we ran experiments on
larger instances, which exceeded generous time limits and resulted in timeouts.
Even in these cases, no counterexamples to \textit{AccountableSafety} were
found. Furthermore, in instances where we deliberately introduced bugs into the
specifications (akin to mutation testing), Apalache, Alloy and CVC5 quickly
generated counterexamples. This increases our confidence that the protocol
remains accountably safe, even for system sizes substantially larger than
those we were able to exhaustively verify.

\paragraph{Advantages of human expertise over automated translation.} Applying
translation rules to derive checkable specifications from existing artifacts can
serve as a valuable starting point. However, such translations often introduce
inefficiencies because they cannot fully capture the nuances of the specific
context. This can result in suboptimal performance. Therefore, while
translations provide a baseline, manually crafting specifications from the
outset may be more effective. When relying on translated specifications, it is
essential to apply manual optimizations to ensure both accuracy and efficiency.

\paragraph{Value of \tlap{}.} \tlap{} is a powerful language for specifying and
verifying distributed systems. Although our most promising experimental results
were derived from the Alloy specification, the insights gained through iterative
abstraction in \tlap{} were indispensable.\ \tlap{} enabled us to start with an
almost direct translation of the Python code and progressively refine it into
higher levels of abstraction. This iterative process provided a deeper
understanding of the protocol and laid the groundwork for the more efficient
Alloy specification.

\colorbox{yellow}{TODO: anything else?}


\subsection{Structure of the report}

\begin{figure}
  %! TeX root = report.tex

\begin{tikzpicture}[node distance=2cm, >=latex]
    \tikzset{%
        mynode/.style={%
            draw, rectangle, minimum width=3cm, minimum height=1cm, align=center,
            rounded corners
        }
    }

    % Define the nodes
    \node[mynode, minimum width=2cm, minimum height=1cm] (py)
        {\small\texttt{ffg.py}};

    \node[mynode, minimum width=4cm,
        minimum height=1cm, below=1cm of py] (spec1)
        {\SpecOne{}: {\small\texttt{spec1-2/ffg\_recursive.tla}}};

    \node[mynode, minimum width=3cm,
        minimum height=1cm, right=1cm of spec1] (spec2)
        {\SpecTwo{}: {\small\texttt{spec1-2/ffg.tla}}};

    \node[mynode, minimum width=3cm,
        minimum height=1cm, below=1cm of spec2] (spec3)
        {\SpecThree{}: {\small\texttt{spec3/ffg.tla}}};

    \node[mynode, minimum width=3cm,
        minimum height=1cm, left=1cm of spec3] (spec4)
        {\SpecFour{}: {\small\texttt{spec4/ffg\_inductive.tla}}};

    \node[mynode, minimum width=3cm,
        minimum height=1cm, below=1cm of spec4] (spec4b)
        {\SpecFourB{}};

    \node[mynode, minimum width=3cm,
        minimum height=1cm, below left=1cm and -2cm of spec3] (spec3b)
        {\SpecThreeB{}: SMT};

    \node[mynode, minimum width=3cm,
        minimum height=1cm, below right=1cm and -2cm of spec3] (spec3c)
        {\SpecThreeC{}: Alloy/SAT };

    \draw[a] (py) -- (spec1);
    \draw[a] (spec1) -- (spec2);
    \draw[a] (spec2) -- (spec3);
    \draw[a] (spec3) -- (spec4);
    \draw[a] (spec3) -- (spec3b);
    \draw[a] (spec3) -- (spec3c);
    \draw[a] (spec4) -- (spec4b);

\end{tikzpicture}

  \caption{The relation between the specification artifacts}\label{fig:artifacts}
\end{figure}

Figure~\ref{fig:artifacts} depicts the relations between the specifications
that we have produced in the project:

\begin{enumerate}
    \item We have started with the executable specification in Python.

    \item \SpecOne{}: This is the specification
        \texttt{spec1-2/ffg\_recursive.tla}. It is the result of a manual
        mechanical translation of the original executable specification in
        Python, which can be found in \texttt{ffg.py}. This specification is
        using mutually recursive operators, which are not supported by
        Apalache. As a result, we are not checking this specification. This
        specification is the result of our work in Milestones~1 and~3.
        It is discussed in Section~\ref{sec:spec1}.

    \item \SpecTwo{}: This is the specification \texttt{spec1-2/ffg.tla}. It is
        a manual adaptation of~\SpecOne{}. The main difference
        between~\SpecTwo{} and~\SpecOne{} is that~\SpecTwo{} uses ``folds''
        (also known as ``reduce'') instead of recursion. This specification is
        the result of our work in Milestones~1 and~2. It is discussed in
        Section~\ref{sec:spec2}.

    \item \SpecThree{}: This is the further abstraction of~\SpecTwo{} that uses
        the concept of a state machine, instead of a purely sequential
        specification (such as the Python code). This specification is the
        result of our work in Milestone~2. It is discussed in
        Section~\ref{sec:spec3}.

    \item \SpecFour{}: This is an extension of~\SpecThree{} that contains
        an inductive invariant in~\texttt{spec4/ffg\_inductive.tla}.
        This specification is the result of our work in Milestone~4.
        It is discussed in Section~\ref{sec:spec4}.

    \item \SpecFourB{} contains further abstractions and decomposition of
        configurations. This is the first~\tlap{} specification that allowed us
        to show accountable safety for models of very small size. This
        specification is the result of our work in Milestone~4.
        It is discussed in Section~\ref{sec:spec4b}.

    \item \SpecThreeB{} contains a specification in SMT using the theory of
        finite sets and cardinalities. In combination with the solver
        CVC5~\cite{BarbosaBBKLMMMN22}, this specification allows us to push
        verification of accountable safety even further. This specification is
        the result of our work in Milestone~4. It is discussed in
        Section~\ref{sec:smt}.

    \item \SpecThreeC{} contains a specification in
        Alloy~\cite{jackson2012software,alloytools}. With this specification,
        we manage to check all small configurations that cover the base case
        and one inductive step of the definitions. This specification is the
        result of our work in Milestone~4. It is discussed in
        Section~\ref{sec:alloy}.

    \item Section~\ref{section3} contains the translation rules and proofs
        that were conducted in Milestone~3.

\end{enumerate}

\subsection{Potential extensions of this project}\label{sec:future}

%! TeX root = report.tex

\begin{enumerate}

  \item \emph{Generating inputs to the Python specification.} As we have noted,
    the power of our~\tlap{} specifications is the ability to generate examples
    with Apalache. This would help the authors of the Python specification to
    produce tests for their specifications.

  \item \emph{Specifications of a refined protocol.} The current version of
    the Python specification is very abstract. On one hand, it is usually
    beneficial to specify a high-level abstraction. On the other hand, as we
    found, the current level of abstraction is quite close to the general inductive
    definitions of justified and finalized checkpoints. We should have better
    chances at model checking more refined protocol specifications.

  \item \emph{Transferring the Alloy encoding to Apalache.} As we have found in
    this project, Alloy offers richer options for fine tuning in terms of the
    search scope. Moreover, given steady advances in SAT solving, this gives us
    hope for achieving a faster feedback loop with model checking of
    distributed protocols such as the 3SF protocol.

\end{enumerate}



