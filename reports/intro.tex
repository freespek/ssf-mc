%! TeX root = report.tex

\section{Introduction}

TODO\footnote{RS, TH: A short intro to the protocol?}

Figure~\ref{fig:artifacts} depicts the relations between the specifications
that we have produced in the project:

\begin{enumerate}
    \item We have started with the executable specification in Python.

    \item \textbf{Spec 1}: This is the specification
        \texttt{spec1-2/ffg\_recursive.tla}. It is the result of a manual
        mechanical translation of the original executable specification in
        Python, which can be found in \texttt{ffg.py}. This specification is
        using mutually recursive operators, which are not supported by
        Apalache. As a result, we are not checking this specification. This
        specification is the result of our work in Milestone 1.

    \item \textbf{Spec 2}: This is the specification \texttt{spec1-2/ffg.tla}. It
        is a manual adaptation of Spec 1. The main difference between Spec 2
        and Spec 1 is that Spec 2 uses ``folds'' (also known as ``reduce'')
        instead of recursion.

    \item TBD

\end{enumerate}

\begin{figure}
  %! TeX root = report.tex

\begin{tikzpicture}[node distance=2cm, >=latex]
    \tikzset{%
        mynode/.style={%
            draw, rectangle, minimum width=3cm, minimum height=1cm, align=center,
            rounded corners
        }
    }

    % Define the nodes
    \node[mynode, minimum width=2cm, minimum height=1cm] (py)
        {\small\texttt{ffg.py}};

    \node[mynode, minimum width=4cm,
        minimum height=1cm, below=1cm of py] (spec1)
        {\SpecOne{}: {\small\texttt{spec1-2/ffg\_recursive.tla}}};

    \node[mynode, minimum width=3cm,
        minimum height=1cm, right=1cm of spec1] (spec2)
        {\SpecTwo{}: {\small\texttt{spec1-2/ffg.tla}}};

    \node[mynode, minimum width=3cm,
        minimum height=1cm, below=1cm of spec2] (spec3)
        {\SpecThree{}: {\small\texttt{spec3/ffg.tla}}};

    \node[mynode, minimum width=3cm,
        minimum height=1cm, left=1cm of spec3] (spec4)
        {\SpecFour{}: {\small\texttt{spec4/ffg\_inductive.tla}}};

    \node[mynode, minimum width=3cm,
        minimum height=1cm, below=1cm of spec4] (spec4b)
        {\SpecFourB{}};

    \node[mynode, minimum width=3cm,
        minimum height=1cm, below left=1cm and -2cm of spec3] (spec3b)
        {\SpecThreeB{}: SMT};

    \node[mynode, minimum width=3cm,
        minimum height=1cm, below right=1cm and -2cm of spec3] (spec3c)
        {\SpecThreeC{}: Alloy/SAT };

    \draw[a] (py) -- (spec1);
    \draw[a] (spec1) -- (spec2);
    \draw[a] (spec2) -- (spec3);
    \draw[a] (spec3) -- (spec4);
    \draw[a] (spec3) -- (spec3b);
    \draw[a] (spec3) -- (spec3c);
    \draw[a] (spec4) -- (spec4b);

\end{tikzpicture}

  \caption{The relation between the specification artifacts}\label{fig:artifacts}
\end{figure}

\begin{figure}
  %! TeX root = report.tex

% inside figure!
\centering
\begin{subfigure}{.35\textwidth}
  \centering
  \begin{tikzpicture}
    \tikzset{n/.style={circle, fill=black, inner sep=2pt}}

    \node[n] (n0) at (0,0) {};
    \node[n] (n1) at (1,.5) {};
    \node[n] (n1') at (1,-.5) {};
      
    \draw[a] (n1) -- (n0);
    \draw[a] (n1') -- (n0);
  \end{tikzpicture}

  \caption{Short fork [M3]}\label{fig:three}
\end{subfigure}
%
\begin{subfigure}{.3\textwidth}
  \centering
  \begin{tikzpicture}
    \tikzset{n/.style={circle, fill=black, inner sep=2pt}}

    \node[n] (n0) at (0,0) {};
    \node[n] (n1) at (1,.5) {};
    \node[n] (n2) at (2,.5) {};
    \node[n] (n1') at (1,-.5) {};
      
    \draw[a] (n1) -- (n0);
    \draw[a] (n2) -- (n1);
    \draw[a] (n1') -- (n0);
  \end{tikzpicture}

  \caption{Four blocks [M4a]}\label{fig:four-top}
\end{subfigure}
%
\begin{subfigure}{.3\textwidth}
  \centering
  \begin{tikzpicture}
    \tikzset{n/.style={circle, fill=black, inner sep=2pt}}

    \node[n] (n0) at (0,0) {};
    \node[n] (n1) at (1,.5) {};
    \node[n] (n1') at (1,-.5) {};
    \node[n] (n2') at (2,-.5) {};
      
    \draw[a] (n1) -- (n0);
    \draw[a] (n1') -- (n0);
    \draw[a] (n2') -- (n1');
  \end{tikzpicture}

  \caption{Four blocks [M4b]}\label{fig:four-bottom}
\end{subfigure}
%
\begin{subfigure}{.3\textwidth}
  \centering
  \begin{tikzpicture}
    \tikzset{n/.style={circle, fill=black, inner sep=2pt}}

    \node[n] (n0) at (0,0) {};
    \node[n] (n1) at (1,.5) {};
    \node[n] (n2) at (2,.5) {};
    \node[n] (n1') at (1,-.5) {};
    \node[n] (n2') at (2,-.5) {};
      
    \draw[a] (n1) -- (n0);
    \draw[a] (n2) -- (n1);
    \draw[a] (n1') -- (n0);
    \draw[a] (n2') -- (n1');
  \end{tikzpicture}

  \caption{Five blocks [M5a]}\label{fig:five1}
\end{subfigure}
%
\begin{subfigure}{.3\textwidth}
  \centering
  \begin{tikzpicture}
    \tikzset{n/.style={circle, fill=black, inner sep=2pt}}

    \node[n] (n0) at (0,0) {};
    \node[n] (n1) at (1,0) {};
    \node[n] (n2) at (2,.5) {};
    \node[n] (n2') at (2,-.5) {};
      
    \draw[a] (n1) -- (n0);
    \draw[a] (n2) -- (n1);
    \draw[a] (n2') -- (n1);
  \end{tikzpicture}

  \caption{Five blocks [M5b]}\label{fig:five2}
\end{subfigure}
%
\begin{subfigure}{.35\textwidth}
  \centering
  \begin{tikzpicture}
    \tikzset{n/.style={circle, fill=black, inner sep=2pt}}

    \node[n] (n0) at (0,0) {};
    \node[n] (n1) at (1,.5) {};
    \node[n] (n2) at (2,.5) {};
    \node[n] (n3) at (3,.5) {};
    \node[n] (n1') at (1,-.5) {};
    \node[n] (n2') at (2,-.5) {};
    \node[n] (n3') at (3,-.5) {};
      
    \draw[a] (n1) -- (n0);
    \draw[a] (n2) -- (n1);
    \draw[a] (n3) -- (n2);
    \draw[a] (n1') -- (n0);
    \draw[a] (n2') -- (n1');
    \draw[a] (n3') -- (n2');
  \end{tikzpicture}

  \caption{Seven blocks [M7]}\label{fig:seven1}
\end{subfigure}
%
\begin{subfigure}{.3\textwidth}
  \centering
  \begin{tikzpicture}
    \tikzset{n/.style={circle, fill=black, inner sep=2pt}}

    \node[n] (n0) at (0,0) {};
    \node[n] (n1) at (1,0) {};
    \node[n] (n2) at (2,0) {};
    \node[n] (n3) at (3,0) {};
    \node[n] (n4) at (4,0) {};
      
    \draw[a] (n1) -- (n0);
    \draw[a] (n2) -- (n1);
    \draw[a] (n3) -- (n2);
    \draw[a] (n4) -- (n3);
  \end{tikzpicture}

  \caption{Single chain}\label{fig:single}
\end{subfigure}
%
%
\begin{subfigure}{.35\textwidth}
  \centering
  \begin{tikzpicture}
    \tikzset{n/.style={circle, fill=black, inner sep=2pt}}

    \node[n] (n0) at (0,0) {};
    \node[n] (n1) at (1,0.5) {};
    \node[n] (n2) at (2,0.5) {};
    \node[n] (n3) at (3,0.5) {};
    \node[n] (n4) at (4,0.5) {};
    \node[n] (n5) at (1,-.5) {};
    \node[n] (n6) at (2,-.5) {};
    \node[n] (n7) at (3,-.5) {};
    \node[n] (n8) at (1,-1) {};
    \node[n] (n9) at (1,-1.5) {};
    \node[n] (n10) at (2,-1.5) {};
      
    \draw[a] (n1) -- (n0);
    \draw[a] (n2) -- (n1);
    \draw[a] (n3) -- (n2);
    \draw[a] (n4) -- (n3);
    \draw[a] (n5) -- (n0);
    \draw[a] (n6) -- (n5);
    \draw[a] (n7) -- (n6);

    \draw[a] (n10) -- (n9);
  \end{tikzpicture}

  \caption{Forest}\label{fig:forest}
\end{subfigure}
%
\begin{subfigure}{.3\textwidth}
  \centering
  \begin{tikzpicture}
    \tikzset{n/.style={circle, fill=black, inner sep=2pt}}

    \node[n] (n0) at (0,0) {};
    \node[n] (n1) at (1,0) {};
    \node[n] (n2) at (2,0) {};
    \node[n] (n3) at (3,0) {};
      
    \draw[a] (n1) -- (n0);
    \draw[a] (n2) -- (n1);
    \draw[a] (n3) -- (n2);
    \draw[a,bend right=45] (n3) to (n1);
  \end{tikzpicture}

  \caption{Impossible chains [I1]}\label{fig:tricky1}
\end{subfigure}
%
\begin{subfigure}{.3\textwidth}
  \centering
  \begin{tikzpicture}
    \tikzset{n/.style={circle, fill=black, inner sep=2pt}}

    \node[n] (n0) at (0,0) {};
    \node[n] (n1) at (1,0) {};
    \node[n] (n2) at (2,0) {};
    \node[n] (n3) at (3,0) {};
      
    \draw[a] (n1) -- (n0);
    \draw[a] (n2) -- (n1);
    \draw[a] (n3) -- (n2);
    \draw[a,bend right=45] (n1) to (n3);
  \end{tikzpicture}

  \caption{Impossible chains [I2]}\label{fig:tricky2}
\end{subfigure}


  \caption{Small configurations of block graphs}\label{fig:block-graphs}
\end{figure}

