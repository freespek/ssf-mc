%! TeX root = report.tex

\begin{enumerate}

  \item \emph{Prove some of the other properties guaranteed by 3SF.} This project focused on verifying AccountableSafety, arguably the most critical property that the 3SF protocol must satisfy.
  However, it is also arguably the most challenging to verify through model checking, as it is the only property involving all honest participants that must hold true under fully asynchronous network conditions.
  In contrast, proving that honest nodes never commit slashable offenses (a non-distributed system property dependent only on the behavior of a single node) or properties reliant on network synchrony, such as reorg-resilience and dynamic-availability, is expected to be simpler, albeit this would require extending the \tlap{} encoding to include the behavior of honest nodes, which was unnecessary for verifying AccountableSafety.


  \item \emph{Generating inputs to the Python specification.} As we have noted,
    the power of our~\tlap{} specifications is the ability to generate examples
    with Apalache. This would help the authors of the Python specification to
    produce tests for their specifications.

  \item \emph{Specifications of a refined protocol.} The current version of the
    Python specification is very abstract. On one hand, it is usually
    beneficial to specify a high-level abstraction. On the other hand, as we
    found, the current level of abstraction is quite close to the general
    inductive definitions of justified and finalized checkpoints. We expect a
    refined protocol specification to be more amenable to model checking.

  \item \emph{Transferring the Alloy encoding to Apalache.} As we have found in
    this project, Alloy offers richer options for fine tuning in terms of the
    search scope. In combination with steady advances in SAT solving, adapting
    the Alloy encoding to Apalache would improve model checking performance,
    ultimately leading to a faster feedback loop and faster specification
    development.

\end{enumerate}

