%! TeX root = report.tex

\section{M3: Translating Python specifications to \tlap{}}

In this section, we present our results on translating a subset of Python that
is used to write executable specifications in the projects such as
\texttt{ssf-mc}\footnote{\url{https://github.com/saltiniroberto/ssf}}. Since we
have done the translation by hand, our rules are currently formalized on paper.
Additionally, in case of non-trivial rules, we give correctness proofs.

Since the Python subset uses the package $\texttt{pyrsistent}$, we assume that
the expressions are typed according to the package types, which can be found in
the $\texttt{typing}$ module. In the following, given a Pyrsistent type $\tau$,
we will denote its corresponding type in the Apalache type
system~1.2\footnote{\url{https://apalache-mc.org/docs/adr/002adr-types.html}}
with $\htau$. Table~\ref{tab:types} shows the types mapping.

\begin{table}[!h]
    \centering
    \begin{tabular}{cc}
        \tbh{Pyrsistent type} & \tbh{Apalache type}
            \\\toprule
        bool & Bool \\\midrule
        int & Int \\\midrule
        str & Str \\\midrule
        $\PSet[\tau]$ & $\Set(\htau)$ \\\midrule
        $\PVec[\tau]$ & $\List(\htau) \defeq \{ es\colon \Seq(\htau) \}$ \\\midrule
        $\PMap[\tau_1, \tau_2]$ & $\htau_1 \rightarrow \htau_2$ \\\midrule
        $\Callable[[\tau_1], \tau_2]$ & $\tau_1 \Rightarrow \tau_2$ \\\midrule
    \end{tabular}
    \caption{Mapping the Pyrsistent
             types to Apalache types}\label{tab:types}
\end{table}

Note that instead of using the standard type $\Seq(\htau)$ of \tlap{}, which
represents 1-indexed sequences, we use an alternative module
\texttt{Lists}\footnote{\url{https://github.com/konnov/tlaki/blob/main/src/Lists.tla}},
which represents 0-indexed sequences. To that end, we introduce the type
notation:

\[ \List(\htau) \coloneqq \{ es\colon \Seq(\htau) \} \]

The translation rules can be easily adapted to~$\Seq(\htau)$ instead
of~$\List(\htau)$.

\subsection{Translation rules}

We will be using the \href{https://github.com/konnov/tlaki/blob/main/src/Lists.tla}{$\texttt{Lists}$} module, in lieu of $\texttt{Sequences}$, to better match the 0-indexing convention of Python. To that end, we introduce the type notation:
\[
\List(\htau) \coloneqq \{ es\colon \Seq(\htau) \}
\]
that is, the instantiation of the $\texttt{list}$ alias defined in $\texttt{Lists.tla}$ with the concrete type $\hat{t}$.

\subsubsection{Singleton vector}
\href{https://github.com/saltiniroberto/ssf/blob/7ea6e18093d9da3154b4e396dd435549f687e6b9/high_level/common/pythonic_code_generic.py#L15-L16}{Source}.


%     a: t
%   ==========   pvector_of_one_element(a) 
%     e: t^
% ==========================================
%         List(<< e >>) : List(t^)

\begin{mathpar}
    \inferrule*[right=(\textsc{Vec})]
    {
        \inferrule{a\colon \tau}{e \colon \htau}
        \\
        \mathrm{pvector\_of\_one\_element}(a)
    }{
        \List(\tup{e})\colon \List(\htau)
    }
\end{mathpar}
A singleton Python vector is translated to a single-element list, and annotated as such.

\subsubsection{Vector concatenation}
\href{https://github.com/saltiniroberto/ssf/blob/7ea6e18093d9da3154b4e396dd435549f687e6b9/high_level/common/pythonic_code_generic.py#L19-L20}{Source}.


%    a: PVec[t]        b: PVec[t]
%  ===============   ===============   pvector_concat(a, b) 
%    e: List(t^)       f: List(t^)
% ============================================================
%                  Concat(e,f) : List(t^)

\begin{mathpar}
    \inferrule*[right=(\textsc{Concat})]
    {
        \inferrule{a\colon \PVec[\tau]}{e \colon \List(\htau)}
        \\
        \inferrule{b\colon \PVec[\tau]}{f \colon \List(\htau)}
        \\
        \mathrm{pvector\_concat}(a, b)
    }{
        \Concat(e,f) \colon \List(\htau)
    }
\end{mathpar}
Vector concatenation is translated to the list concatenation.

\subsubsection{Set sequentialization}
\href{https://github.com/saltiniroberto/ssf/blob/7ea6e18093d9da3154b4e396dd435549f687e6b9/high_level/common/pythonic_code_generic.py#L23-L24}{Source}.


%          a: PSet[t]         
%        ==============   from_set_to_pvector(a) 
%          e: Set(t^)       
% ======================================================
%   ApaFoldSet( Push, List(<<>>: Seq(t^)), e ) : List(t^)

\begin{mathpar}
    \inferrule*[right=(\textsc{SetToVec})]
    {
        \inferrule{a\colon \PSet[\tau]}{e \colon \Set(\htau)}
        \\
        s \coloneqq \tup{}\colon \Seq(\htau)
        \\
        \mathrm{from\_set\_to\_pvector}(a)
    }{
        \ApaFoldSet( \Push, \List(s), e ) \colon \List(\htau)
    }
\end{mathpar}
We use fold, to create a sequence (in some order) from the set.

\subsubsection{ Empty set}
\href{https://github.com/saltiniroberto/ssf/blob/7ea6e18093d9da3154b4e396dd435549f687e6b9/high_level/common/pythonic_code_generic.py#L27-L28}{Source}.


%   pset_get_empty : PSet[t]
% ============================
%        {} : Set(t^)

\begin{mathpar}
    \inferrule*[right=(\textsc{EmptySet})]
    {
        \mathrm{pset\_get\_empty}() \colon \PSet[t]
    }{
        \{\} \colon \Set(\htau)
    }
\end{mathpar}
The only relevant part here is that we need a type annotation on the Python side to correctly annotate the empty set in \tlap{}.

\subsubsection{ Set union}
\href{https://github.com/saltiniroberto/ssf/blob/7ea6e18093d9da3154b4e396dd435549f687e6b9/high_level/common/pythonic_code_generic.py#L31-L32}{Source}.


%     a: PSet[t]       b: PSet[t]
%   ==============   ==============   pset_merge(a, b) 
%     e: Set(t^)       f: Set(t^)
% ======================================================
%                 e \cup f : Set(t^)

\begin{mathpar}
    \inferrule*[right=(\textsc{Union})]
    {
        \inferrule{a\colon \PSet[\tau]}{e \colon \Set(\htau)}
        \\
        \inferrule{b\colon \PSet[\tau]}{f \colon \Set(\htau)}
        \\
        \mathrm{pset\_merge}(a, b)
    }{
        e \cup f \colon \Set(\htau)
    }
\end{mathpar}
Set union is translated to the \tlap{} native set union.

\subsubsection{ Set flattening}
\href{https://github.com/saltiniroberto/ssf/blob/7ea6e18093d9da3154b4e396dd435549f687e6b9/high_level/common/pythonic_code_generic.py#L35-L36}{Source}.


%          a: PSet[PSet[t]]         
%        ====================   pset_merge_flatten(a) 
%          e: Set(Set(t^))       
% =====================================================
%                 UNION e : Set(t^)

\begin{mathpar}
    \inferrule*[right=(\textsc{BigUnion})]
    {
        \inferrule{ a\colon \PSet[\PSet[\tau]]}{e \colon \Set(\Set(\htau))}
        \\
        \mathrm{pset\_merge\_flatten}(a)
    }{
        \UNION e \colon \Set(\htau)
    }
\end{mathpar}
Set flattening is translated to the \tlap{} native big $\UNION$.

\subsubsection{Set intersection}
\href{https://github.com/saltiniroberto/ssf/blob/7ea6e18093d9da3154b4e396dd435549f687e6b9/high_level/common/pythonic_code_generic.py#L42-L43}{Source}.


%     a: PSet[t]       b: PSet[t]
%   ==============   ==============   pset_intersection(a, b) 
%     e: Set(t^)       f: Set(t^)
% =============================================================
%                    e \cap f : Set(t^)

\begin{mathpar}
    \inferrule*[right=(\textsc{Intersection})]
    {
        \inferrule{a\colon \PSet[\tau]}{e \colon \Set(\htau)}
        \\
        \inferrule{b\colon \PSet[\tau]}{f \colon \Set(\htau)}
        \\
        \mathrm{pset\_intersection}(a, b)
    }{
        e \cap f \colon \Set(\htau)
    }
\end{mathpar}
Set intersection is translated to the \tlap{} native set intersection.

\subsubsection{Set difference}
\href{https://github.com/saltiniroberto/ssf/blob/7ea6e18093d9da3154b4e396dd435549f687e6b9/high_level/common/pythonic_code_generic.py#L46-L47}{Source}.


%     a: PSet[t]       b: PSet[t]
%   ==============   ==============   pset_difference(a, b) 
%     e: Set(t^)       f: Set(t^)
% ===========================================================
%                       e \ f : Set(t^)

\begin{mathpar}
    \inferrule*[right=(\textsc{SetDiff})]
    {
        \inferrule{a\colon \PSet[\tau]}{e \colon \Set(\htau)}
        \\
        \inferrule{b\colon \PSet[\tau]}{f \colon \Set(\htau)}
        \\
        \mathrm{pset\_difference}(a, b)
    }{
        e \setminus f \colon \Set(\htau)
    }
\end{mathpar}
Set difference is translated to the \tlap{} native set difference.

\subsubsection{Singleton set}
\href{https://github.com/saltiniroberto/ssf/blob/7ea6e18093d9da3154b4e396dd435549f687e6b9/high_level/common/pythonic_code_generic.py#L50-L51}{Source}.

%     a: t
%   =========   pset_get_singleton(a) 
%     e: t^
% =====================================
%            { e } : Set(t^)

\begin{mathpar}
    \inferrule*[right=(\textsc{Singleton})]
    {
        \inferrule{a\colon \tau}{e \colon \htau}
        \\
        \mathrm{pset\_get\_singleton}(a)
    }{
        \{e\} \colon \Set(\htau)
    }
\end{mathpar}
A singleton Python set is translated to a \tlap{} native single-element set.

\subsubsection{Set extension}
\href{https://github.com/saltiniroberto/ssf/blob/7ea6e18093d9da3154b4e396dd435549f687e6b9/high_level/common/pythonic_code_generic.py#L54-L55}{Source}.


%     a: PSet[t]       b: t
%   ==============   =========  pset_add(a, b) 
%     e: Set(t^)       f: t^
% ==============================================
%             e \cup { f } : Set(t^) 

\begin{mathpar}
    \inferrule*[right=(\textsc{SetExt})]
    {
        \inferrule{a\colon \PSet[\tau]}{e \colon \Set(\htau)}
        \\
        \inferrule{b\colon \tau}{f \colon \htau}
        \\
        \mathrm{pset\_add}(a, b)
    }{
        e \cup \{f\} \colon \Set(\htau)
    }
\end{mathpar}
A set extension is translated to a combination of union and singleton-set construction. Semantically, this is the equivalence
\[
\mathrm{pset\_add}(a,b) = \mathrm{pset\_merge}(a, \mathrm{pset\_get\_singleton}(b))
\]

\subsubsection{Element choice}
\href{https://github.com/saltiniroberto/ssf/blob/7ea6e18093d9da3154b4e396dd435549f687e6b9/high_level/common/pythonic_code_generic.py#L58-L60}{Source.}


%     a: PSet[t]
%   ==============   pset_pick_element(a) 
%     e: Set(t^)
% =========================================
%          CHOOSE x \in e: TRUE: t^

\begin{mathpar}
    \inferrule*[right=(\textsc{Choice})]
    {
        \inferrule{a\colon \PSet[\tau]}{e \colon \Set(\htau)}
        \\
        \mathrm{pset\_pick\_element}(a)
    }{
        (\CHOOSE x \in e\colon\TRUE) \colon\htau
    }
\end{mathpar}
We translate this to the built in deterministic choice in \tlap{}. We cannot account for the dynamic non-emptiness requirement. Instead, in that scenario, the value of this expression is some unspecified element of the correct type.

\subsubsection{Set filter}
\href{https://github.com/saltiniroberto/ssf/blob/7ea6e18093d9da3154b4e396dd435549f687e6b9/high_level/common/pythonic_code_generic.py#L63-L70}{Source}.


%     a: Callable[[t], bool]        b: PSet[t]
%   ===========================   ==============   pset_filter(a, b) 
%          e: t^ -> bool            f: Set(t^)
% ====================================================================
%                       { x \in f: e[x] }: Set(t^)

\begin{mathpar}
    \inferrule*[right=(\textsc{Filter})]
    {
        \inferrule{a\colon \Callable[[\tau], \bool]}{e \colon \htau \to \Bool}
        \\
        \inferrule{b\colon \PSet[\tau]}{f \colon \Set(\htau)}
        \\
        \mathrm{pset\_filter}(a, b)
    }{
        \{ x \in f \colon e[x] \} \colon \Set(\htau)
    }
\end{mathpar}
Set filtering is translated to the \tlap native filter operation.

\subsubsection{ Set maximum}
\href{https://github.com/saltiniroberto/ssf/blob/7ea6e18093d9da3154b4e396dd435549f687e6b9/high_level/common/pythonic_code_generic.py#L74-L76}{Source}.


%     a: PSet[t]       b: Callable[[t], T]        
%   ==============   =======================   pset_max(a, b) 
%     e: Set(t^)           f: t^ -> T^            
% =============================================================
%       CHOOSE max \in e: \A x \in e: Le(f[x], f[max])

\begin{mathpar}
    \inferrule*[right=(\textsc{Max})]
    {
        \inferrule{a\colon \PSet[\tau]}{e \colon \Set(\htau)}
        \\
        \inferrule{b\colon \Callable[[\tau], T]}{f \colon \htau \to \hat{T}}       
        \\
        \mathrm{pset\_max}(a, b)
    }{
        (\CHOOSE m \in e\colon \forall x \in e\colon \Le(f[x], f[m]))\colon \htau
    }
\end{mathpar}
Here, the translation depends on the type $T$ (resp. type $\hat{T}$), since there is no built-in notion of ordering in \tlap{}. 
\paragraph{Instance 1:} If $\hat{T}$ is an integer type, then 
\begin{lstlisting}[language=tla,columns=fullflexible]
Le(x,y) $\defeq$ x $\le$ y
\end{lstlisting}
\paragraph{Instance 2:} If $\hat{T}$ is a tuple type $\tup{\Int,\Int}$, it is instead 
% \begin{align*}
% \Le(x,y) \defeq&\;\IF x[1] > y[1]\\
%   &\THEN\FALSE\\
%   &\ELSE\;\IF x[1] < y[1]\\
%        &\phantom{\ELSE}\THEN \TRUE\\
%        &\phantom{\ELSE}\ELSE x[2] \le y[2]\\
% \end{align*}
\begin{lstlisting}[language=tla,columns=fullflexible]
Le(x,y) $\defeq$ 
  IF x[1] > y[1]
  THEN FALSE
  ELSE IF x[1] < y[1]
       THEN TRUE
       ELSE x[2] $\le$ y[2]
\end{lstlisting}

\subsubsection{ Set sum}
\href{https://github.com/saltiniroberto/ssf/blob/7ea6e18093d9da3154b4e396dd435549f687e6b9/high_level/common/pythonic_code_generic.py#L79-L80}{Source}.


%             a: PSet[int]
%           ================   pset_sum(a) 
%             e: Set(int)
% ===========================================================
%   LET Plus(x,y) == x + y IN ApaFoldSet(Plus, 0, e ): int

\begin{mathpar}
    \inferrule*[right=(\textsc{Sum})]
    {
        \inferrule{a\colon \PSet[\pyint]}{e \colon \Set(\Int)}
        \\
        \mathrm{pset\_sum}(a)
    }{
        \ApaFoldSet(+, 0, e) \colon \Int
    }
\end{mathpar}
We translate a set sum as a fold of the $+$ operator over the set.

\subsubsection{ Set emptiness check}
\href{https://github.com/saltiniroberto/ssf/blob/7ea6e18093d9da3154b4e396dd435549f687e6b9/high_level/common/pythonic_code_generic.py#L83-L84}{Source}.


%      a: PSet[t]         
%    ==============   pset_is_empty(a) 
%      e: Set(t^)       
% ======================================
%             e = {} : bool

\begin{mathpar}
    \inferrule*[right=(\textsc{IsEmpty})]
    {
        \inferrule{a\colon \PSet[\tau]}{e \colon \Set(\htau)}
        \\
        \mathrm{pset\_is\_empty}(a)
    }{
        e = \{\} \colon \Bool
    }
\end{mathpar}
The emptiness check is translated to a comparison with the explicitly constructed empty set.

\subsubsection{ Vector-to-Set conversion}
\href{https://github.com/saltiniroberto/ssf/blob/7ea6e18093d9da3154b4e396dd435549f687e6b9/high_level/common/pythonic_code_generic.py#L87-L88}{Source}.


%     a: PVec[t]         
%   ===============   from_pvector_to_pset(a) 
%     e: List(t^)       
% =============================================
%   { At(e, i) : i \in Indices(e) }: Set(t^)            

\begin{mathpar}
    \inferrule*[right=(\textsc{VecToSet})]
    {
        \inferrule{a\colon \PVec[\tau]}{e \colon \List(\htau)}
        \\
        \mathrm{from\_pvector\_to\_pset}(a)
    }{
        \{ \At(e, i)\colon i \in \Indices(e) \} \colon \Set(\htau)  
    }
\end{mathpar}
We translate the set-conversion, by mapping the accessor method over $\Indices$.

\subsubsection{ Set mapping}
\href{https://github.com/saltiniroberto/ssf/blob/7ea6e18093d9da3154b4e396dd435549f687e6b9/high_level/common/pythonic_code_generic.py#L91-L97}{Source}.


%     a: Callable[[t1], t2]       b: PSet[t1]
%   =========================   ==============   pset_map(a, b) 
%         e: t1^ -> t2^           f: Set(t1^)
% ===============================================================
%                 { e[x]: x \in f}: Set(t2^)

\begin{mathpar}
    \inferrule*[right=(\textsc{Map})]
    {
        \inferrule{a\colon \Callable[[\tau_1], \tau_2]}{e \colon \htau_1 \to \htau_2}
        \\
        \inferrule{b\colon \PSet[\tau_1]}{f \colon \Set(\htau_1)}
        \\
        \mathrm{pset\_map}(a, b)
    }{
        \{ e[x]\colon x \in f\} \colon \Set(\htau_2)
    }
\end{mathpar}
Set mapping is translated to the \tlap{} native map operation.

\subsubsection{ Function domain inclusion check}
\href{https://github.com/saltiniroberto/ssf/blob/7ea6e18093d9da3154b4e396dd435549f687e6b9/high_level/common/pythonic_code_generic.py#L100-L101}{Source}.


%     a: PMap[t1, t2]       b: t1
%   ===================   =========   pmap_has(a, b) 
%      e: t1^ -> t2^        f: t1^
% ===================================================
%               f \in DOMAIN e: bool 

\begin{mathpar}
    \inferrule*[right=(\textsc{InDom})]
    {
        \inferrule{a\colon \PMap[\tau_1, \tau_2]}{f \colon \htau_1 \to \htau_2}
        \\
        \inferrule{b\colon \tau_1}{e \colon \htau_1}
        \\
        \mathrm{pmap\_has}(a, b)
    }{
        e \in \DOMAIN f\colon \Bool
    }
\end{mathpar}
Function domain inclusion checking is translated to the \tlap{} native set-inclusion operation for $\DOMAIN$.

\subsubsection{ Function application}
\href{https://github.com/saltiniroberto/ssf/blob/7ea6e18093d9da3154b4e396dd435549f687e6b9/high_level/common/pythonic_code_generic.py#L104-L106}{Source}.


%     a: PMap[t1, t2]       b: t1
%   ===================   =========   pmap_get(a, b) 
%      e: t1^ -> t2^        f: t1^
% ===================================================
%                   e[f]: t2^

\begin{mathpar}
    \inferrule*[right=(\textsc{App})]
    {
        \inferrule{a\colon \PMap[\tau_1, \tau_2]}{f \colon \htau_1 \to \htau_2}
        \\
        \inferrule{b\colon \tau_1}{e \colon \htau_1}
        \\
        \mathrm{pmap\_get}(a, b)
    }{
        f[e]\colon \htau_2
    }
\end{mathpar}
Function application is translated to the \tlap{} native function application. We cannot account for the dynamic domain-membership requirement. Instead, in that scenario, the value of this expression is some unspecified element of the correct type.


\subsubsection{ Empty function}
\href{https://github.com/saltiniroberto/ssf/blob/7ea6e18093d9da3154b4e396dd435549f687e6b9/high_level/common/pythonic_code_generic.py#L109-L110}{Source}.


%          pmap_get_empty : PMap[t1, t2]
% ===============================================
%   SetAsFun({}: Set(<<t1^, t2^>>)): t1^ -> t2^

\begin{mathpar}
    \inferrule*[right=(\textsc{EmptyFun})]
    {
        \mathrm{pmap\_get\_empty}()\colon \PMap[\tau_1,\tau_2]
        \\
        s \coloneqq \{\}\colon \Set(\tup{\htau_1, \htau_2}) 
    }{
      	\SetAsFun(s)\colon \htau_1 \to \htau_2
    }
\end{mathpar}
We use Apalache's $\SetAsFun$, since we only need to annotate the empty set with the correct tuple type. The native construction via $[ \_ \mapsto \_]$ would require us to invent a codomain value, which we might not have access to if $\tau_1 \ne \tau_2$.

\subsubsection{ Function update}
\href{https://github.com/saltiniroberto/ssf/blob/7ea6e18093d9da3154b4e396dd435549f687e6b9/high_level/common/pythonic_code_generic.py#L113-L114}{Source}.


%     a: PMap[t1, t2]       b: t1       c: t2
%   ===================   =========   ==========   pmap_set(a, b, c) 
%      e: t1^ -> t2^        f: t1^      g: t2^
% ===================================================================
%                 [e EXCEPT ![f] = g] : t1^ -> t2^

\begin{mathpar}
    \inferrule*[right=(\textsc{Update})]
    {
        \inferrule{a\colon \PMap[\tau_1, \tau_2]}{f \colon \htau_1 \to \htau_2}
        \\
        \inferrule{b\colon \tau_1}{x \colon \htau_1}
        \\
        \inferrule{c\colon \tau_2}{y \colon \htau_2}
        \\
        \mathrm{pmap\_set}(a, b, c)
    }{
        [ v \in (\DOMAIN f \cup \{x\}) \mapsto \IF v = x \THEN y \ELSE f[x] ] \colon \htau_1 \to \htau_2
    }
\end{mathpar}
While one might intuitively want to translate map updates using the \tlap{} native $\EXCEPT$, we cannot, since $\EXCEPT$ \href{https://lamport.azurewebsites.net/tla/book-21-07-04.pdf}{does not allow for domain extensions}, whereas $\mathrm{pmap\_set}$ \href{https://pyrsistent.readthedocs.io/en/latest/api.html#pyrsistent.PMap.set}{does}. By definition
\[
[f \EXCEPT ![x] = y] \defeq [v \in \DOMAIN f \mapsto \IF v = x \THEN y \ELSE f[x] ]
\]
where, most notably, the domain of $[f \EXCEPT ![x] = y]$ is exactly the domain of $f$. We adapt the above definition to (possibly) extend the domain.

\subsubsection{ Function combination}
\href{https://github.com/saltiniroberto/ssf/blob/7ea6e18093d9da3154b4e396dd435549f687e6b9/high_level/common/pythonic_code_generic.py#L117-L118}{Source}.


%                 a: PMap[t1, t2]       b: PMap[t1, t2]
%               ===================   ===================   pmap_merge(a, b) 
%                  e: t1^ -> t2^         f: t1^ -> t2^
% ============================================================================================
%   [ x \in (DOMAIN e \cup DOMAIN f) |-> IF x \in DOMAIN f THEN f[x] ELSE e[x] ]: t1^ -> t2^

\begin{mathpar}
    \inferrule*[right=(\textsc{FnMerge})]
    {
        \inferrule{a\colon \PMap[\tau_1, \tau_2]}{f \colon \htau_1 \to \htau_2}
        \\
        \inferrule{b\colon \PMap[\tau_1, \tau_2]}{g \colon \htau_1 \to \htau_2}
        \\
        \mathrm{pmap\_merge}(a,b)
    }{
        [x \in (\DOMAIN f \cup \DOMAIN g) \mapsto \IF x \in \DOMAIN g \THEN g[x] \ELSE f[x]]\colon \htau_1\to\htau_2
    }
\end{mathpar}
Function combination is translated to a new function, defined over the union of both domains. Note that the second map/function dominates in the case of key/domain collisions.


\subsubsection{ Function domain}
\href{https://github.com/saltiniroberto/ssf/blob/7ea6e18093d9da3154b4e396dd435549f687e6b9/high_level/common/pythonic_code_generic.py#L121-L122}{Source}.


%     a: PMap[t1, t2]   
%   ===================   pmap_keys(a) 
%      e: t1^ -> t2^    
% ======================================
%           DOMAIN e: Set(t1^)

\begin{mathpar}
    \inferrule*[right=(\textsc{FnDomain})]
    {
        \inferrule{a\colon \PMap[\tau_1, \tau_2]}{e \colon \htau_1 \to \htau_2}
        \\
        \mathrm{pmap\_keys}(a)
    }{
        \DOMAIN e\colon \Set(\htau_1)
    }
\end{mathpar}
We translate this to the \tlap{} native $\DOMAIN$.


\subsubsection{ Function codomain}
\href{https://github.com/saltiniroberto/ssf/blob/7ea6e18093d9da3154b4e396dd435549f687e6b9/high_level/common/pythonic_code_generic.py#L125-L126}{Source}.


%     a: PMap[t1, t2]   
%   ===================   pmap_values(a) 
%      e: t1^ -> t2^    
% ========================================
%    { e[x]: x \in DOMAIN e }: Set(t2^)

\begin{mathpar}
    \inferrule*[right=(\textsc{FnCodomain})]
    {
        \inferrule{a\colon \PMap[\tau_1, \tau_2]}{e \colon \htau_1 \to \htau_2}
        \\
        \mathrm{pmap\_values}(a)
    }{
        \{ e[x]\colon x \in \DOMAIN e \}\colon \Set(\htau_2)
    }
\end{mathpar}
We translate this by mapping the function over its $\DOMAIN$.


\subsection{ Meta-rules}

In order to facilitate translation to the \tlap{} fragment supported by Apalache, we introduce a set of \tlap{}-to-\tlap{} rules, which allow us to
\begin{enumerate}
\item formulate translations from Python to \tlap{} in the intuitive way, potentially introducing constructs like recursion, and then
\item pair them with a \tlap{}-to-\tlap{} rule, ending in a supported fragment.
\end{enumerate}

\subsubsection{Bounded recursion rule}

Assume we are given a $\RECURSIVE$ operator $\op$. W.l.o.g. we can take the arity to be $1$, since any operator of higher arity can be expressed as an arity $1$ operator over tuples or records.
We assume $\op$ has the following shape:
\begin{lstlisting}[language=tla,columns=fullflexible]
RECURSIVE R(_)
\* $@$type (a) => b;
R(x) ==
  IF P(x)
  THEN e
  ELSE G(x, R(f(x))
\end{lstlisting}
%
i.o.w., we have:
\begin{itemize}
  \item A termination condition $P$
  \item A "default" value $e$, returned if the argument satisfies the termination condition
  \item A general case operator $G$, which invokes a recursive computation of $R$ over a modified parameter given by the operator $\bb$.
\end{itemize}
%
The following needs to hold true, to ensure recursion termination: for every $x\colon a$, there exists a finite sequence $x = v_1, \dots, v_n$, such that
\begin{itemize}
\item $P(v_n)$ holds
\item $v_{i+1} = \bb(v_i)$ for all $1 \le i < n$
\item $P(v_i)$ does not hold for any $1 \le i < n$ (i.o.w., this is the shortest sequence with the above two properties)
\end{itemize}
%
We will attempt to express the recursive operator $\op$ with a non-recursive "iterative" operator $\nrop$ of arity $2$, which takes an additional parameter: a constant $N$. The non-recursive operator will have the property that, for any particular choice of $x$, $\nrop(x, N)$ will evaluate to $\op(x)$ if $n < N$ (i.e. if the recursion stack of $\op$ has height of at most $N$).
%
To that end, we first define:
\begin{lstlisting}[language=tla,columns=fullflexible]
\* $@$type (a, Int) => Seq(a);
Stack(x, N) ==
  LET 
    \* $@$type: (Seq(a), Int) => Seq(a);
    step(seq, i) ==
      IF i > Len(seq) \/ P(seq[1])
      THEN seq
      ELSE <<f(seq[1])>> \o seq \* Alternatively, we can append here and reverse the list at the end
  IN ApaFoldSeqLeft( step, <<x>>, MkSeq(N, LAMBDA i: i) )
\end{lstlisting}
%
We can see that $\Chain(x,N)$ returns the sequence $\tup{v_n, ..., x}$ if $N$ is sufficiently large. We can verify whether or not that is the case, by evaluating $P(\Chain(x, N)[1])$. If it does not hold, the $N$ chosen is not large enough, and needs to be increased. Using $\Chain$ we can define a fold-based non-recursive operator $\op^*$, such that $\op^*(x) = \op(x)$ under the above assumptions:

\begin{lstlisting}[language=tla,columns=fullflexible]
\* $@$type (a, Int) => b;
I(x, N) ==
  LET stack == Stack(x, N) IN
  LET step(cumul, v) == G(v, cumul) IN
  ApaFoldSeqLeft( step, e, Tail(stack) )
\end{lstlisting}
%
Then, $\op^*(x) = \nrop(x, N_0)$ for some sufficiently large specification-level constant $N_0$. Alternatively,
\begin{lstlisting}[language=tla,columns=fullflexible]
\* $@$type (a, Int) => b;
I(x, N) ==
  LET stack == Stack(x, N) IN
  LET step(cumul, v) == G(v, cumul) IN
  IF P(stack[1])
  THEN ApaFoldSeqLeft(step, e, Tail(stack))
  ELSE CHOOSE x \in {}: TRUE 
\end{lstlisting}
%
In this form, we return $\CHOOSE x \in {}: \TRUE$, which is an idiom meaning "any value" (of the correct type), in the case where the $N$ chosen was not large enough. Tools can use this idiom to detect that $\nrop(x,N)$ did not evaluate to the expected value of $\op(x)$. 

\paragraph{Example.} Consider the following operator:
\begin{lstlisting}[language=tla,columns=fullflexible]
RECURSIVE R(_)
\* $@$type (Int) => Int;
R(x) ==
  IF x <= 0
  THEN 0
  ELSE x + R(x-1)
\end{lstlisting}
where $P(x) = x \le 0$, $G(a,b) = a + b$, and $\bb(x) = x - 1$. For this operator, we know that $\op(4) = 10$. By the above definitions:
\begin{lstlisting}[language=tla,columns=fullflexible]
\* $@$type (Int, Int) => Seq(Int);
Stack(x, N) ==
  LET 
    \* $@$type: (Seq(Int), Int) => Seq(Int);
    step(seq, i) ==
      IF i > Len(seq) \/ seq[1] <= 0
      THEN seq
      ELSE <<seq[1] - 1>> \o seq
  IN ApaFoldSeqLeft( step, <<x>>, MkSeq(N, LAMBDA i: i) )
\end{lstlisting}
%
We can compute the above $\Chain$ with two different constants $N$, $2$ and $100$, and observe that $\Chain(4, 2) = \tup{2, 3, 4}$ and $\Chain(4, 100) = \tup{0, 1, 2, 3, 4}$. 
We are able to tell whether we have chosen sufficiently large values for $N$ after the fact, by evaluating $P(\Chain(x,N)[1])$. 
For our $P(x) = x \le 0$, we see $\neg P(\Chain(4, 2)[1])$, and $P(\Chain(4, 100)[1])$, so we can conclude that we should not pick $N=2$, but $N=100$ suffices. 
Of course it is relatively easy to see, in this toy example, that the recursion depth is exactly $4$, but we could use this post-evaluation in cases where the recursion depth is harder to evaluate from the specification, to determine whether we need to increase the value of $N$.

\noindent Continuing with the next operator:
\begin{lstlisting}[language=tla,columns=fullflexible]
\* $@$type (Int, Int) => Int;
I(x, N) ==
  LET stack == Stack(x, N)
  IN ApaFoldSeqLeft( +, 0, Tail(stack) )
\end{lstlisting}
%
we see that $\nrop(4, 2) = 7 \ne 10 = \op(4)$, but $\nrop(4, 100) = 10 = \op(4)$.
As expected, choosing an insufficiently large value of $N$ will give us an incorrect result, but, as stated above, we know how to detect whether we have chosen an appropriate $N$.

\paragraph{ Optimization for associative $G$.}
In the special case where $G$ is associative, that is, $G(a, G(b, c)) = G(G(a, b), c)$ for all $a,b,c$, we can make the entire translation more optimized, and single-pass. Since $\nrop(x,N)$, for sufficiently large $N$, computes 
\[
G(v_1, G(v_2, ... (G(v_{n-2}, G(v_{n-1}, e)))))
\]
%
and $G$ is associative by assumption, then computing
\[
G(G(G(G(v_1, v_2), ...), v_{n-1}), e)
\]
gives us the same value. This computation can be done in a single pass:
\newpage
\begin{lstlisting}[language=tla,columns=fullflexible]
IForAssociative(x, N) ==
  IF P(x)
  THEN e
  ELSE
    LET 
      \* $@$type: (<<a, a>>, Int) => <<a, a>>;
      step(pair, i) == \* we don't use the index `i`
        LET partialAppChain == pair[1]
            currentElemInSeq == pair[2]
        IN
          IF P(currentElemInSeq)
          THEN pair
          ELSE
            LET nextElemInSeq == b(currentElemInSeq)
            IN << G(partialAppChain, IF P(nextElemInSeq) e ELSE nextElemInSeq), nextElemInSeq >>
    IN ApaFoldSeqLeft( step, <<x, x>>, MkSeq(N, LAMBDA i: i) )[1]
\end{lstlisting}

\subsubsection{ Mutual recursion cycles}

Assume we are given a collection of $n$ operators $\op_1, \dots, \op_n$ (using the convention $\op_{n+1} = \op_1$), with types $\op_i\colon (a_i) \Rightarrow a_{i+1}$ s.t. $a_{n+1} = a_1$, in the following pattern:

\begin{lstlisting}[language=tla,columns=fullflexible]
RECURSIVE R_i(_)
\* $@$type (a_i) => a_{i+1};
R_i(x) == G_i(x, R_{i+1}(f_i(x)))
\end{lstlisting}
%
Then, we can inline any one of these operators, w.l.o.g. $\op_1$, s.t. we obtain a primitive-recursive operator:
\begin{lstlisting}[language=tla,columns=fullflexible]
RECURSIVE R(_)
\* $@$type: (a_1) => a_1;
R(x) ==
  G_1( x, 
    G_2( f_1(x),
      G_3( f_2(f_1(x)),
        G_4( ...
          G_n( f_{n-1}(f_{n-2}(...(f_1(x)))), 
               R(f_n(f_{n-1}(...(f_1(x)))))
            )
          )
        )
      )
    )
\end{lstlisting}
for which $\op(x) = \op_1(x)$ for all $x$, and $\op$ terminates iff $\op_1$ terminates.

\subsubsection{One-to-many recursion}

Suppose we are given, for each value $x: a$, a set $V(x)\colon \Set(c)$, and an operator $h\colon \Set(c) \Rightarrow Set(a)$, s.t. $h(V(x)$ is exactly the set of values $v$, for which we are required to recursively compute $\op(v)$, in order to compute $\op(x)$. 
Further, assume that there exists a mapping $\gamma$ from $a$ to nonnegative integers, with the property that, for any $x$ of type $a$ the following holds:
\[
\forall y \in h(V(x))\colon \gamma(y) < \gamma(x) 
\]
%
If one cannot think of a more intuitive candidate for $\gamma$, one may always take $\gamma(t)$ to be the recursion stack-depth required to compute $\op(t)$ (assuming termination). It is easy to see that such a definition satisfies the above condition.

\noindent Let $\op$ have the following shape:
\begin{lstlisting}[language=tla,columns=fullflexible]
RECURSIVE Op(_)
\* $@$type (a) => b;
Op(x) ==
  IF P(x)
  THEN e
  ELSE G(x, F(h(V(x)), Op))
\end{lstlisting}
%
where $F(S, T(\_)) \coloneqq \{s \in S\colon T(s)\}$ or $F(S, T(\_)) \coloneqq \{T(s)\colon s \in S\}$ (i.e. a map or a filter).
%
We can translate this type of recursion, to the one above, by introducing a map-base recursive operator $\mop$, for which we will ensure
\[
\op(x) = \mop([ v \in \{x\} \mapsto V(x) ])[x]
\] 
if $\op(x)$ terminates. We define the necessary operators in Figure \ref{fig1}. Using these operator definitions, we can show the following theorem:
\begin{figure}[ht]
\caption{$\mop$ and auxiliary operators \label{fig1}}
\begin{lstlisting}[language=tla,columns=fullflexible]
\* $@$type: (a -> Set(c)) => a -> Set(c);
b(map) ==
  LET newDomain == UNION {h(map[v]): v \in DOMAIN map}
  IN [ newDomainElem \in newDomain |-> V(newDomainElem) ]

\* $@$type: (a -> Set(c), a -> b) => a -> b;
mapG(currentRecursionStepMap, partialValueMap) ==
  LET domainExtension == DOMAIN currentRecursionStepMap IN
  LET 
    \* $@$type: (a) => b;
    evalOneKey(k) ==
      LET OpSubstitute(x) == partialValueMap[x] 
      IN G(k, F(h(currentRecursionStepMap[k]), OpSubstitute))
  IN [
    x \in (domainExtension \union DOMAIN partialValueMap) |->
      IF x \in DOMAIN partialValueMap
      THEN partialValueMap[x]
      ELSE IF P(x)
           THEN e
           ELSE evalOneKey(x)
  ]

RECURSIVE mapOp(_)
\* $@$type (a -> Set(c)) => a -> b;
mapOp(map) ==
  IF \A x \in DOMAIN map: P(x)
  THEN [ x \in DOMAIN map |-> e ]
  ELSE mapG(map, mapOp(b(map)))
\end{lstlisting}
\end{figure}

\newcommand{\thmBody}{
Let $f$ be a function, s.t. for any $x \in D_f$ it is the case that $f[x] = V(x)$. Then, for
$g \coloneqq \mop(f)$
\[
\forall x \in D_g \ .\ g[x] = \op(x)
\]
}
 
\begin{theorem}\label{thm}
\thmBody
\end{theorem}
From this theorem, the following corollary trivially follows:
\newcommand{\corollaryBody}{
For any $x\colon a$
\[
\op(x) = \mop([ v \in \{x\} \mapsto V(x) ])[x]
\]
}
 
\begin{corollary}\label{corollary}
\corollaryBody
\end{corollary}
The equivalence proofs are available in the appendix. The termination proof must still be made on a case-by-case basis, as it depends on $h$ and $V$.








