%! TeX root = ./report.tex

\section{\SpecFourB{}: Decomposition \& abstractions}\label{sec:spec4b}

Disappointed with the results in Section~\ref{sec:spec4}, we have decided to
push abstractions even further. These abstractions helped us to verify
accountable safety for the configuration in Figure~\ref{fig:three}.
Unfortunately, they do not scale to larger configurations. In any case, we
find these abstractions quite important for further research on model checking
of algorithms similar to 3SF\@.

\subsection{Constraints over set cardinalities}

\SpecThree{} contains several comparisons over set cardinalities. For example:

\begin{equation}
    3 * \textit{Cardinality}(\textit{validatorsWhoCastJustifyingVote}) \ge 2 \cdot N
    \label{eq:card-comparison}
\end{equation}

In the general case, the symbolic model checker has to encode constraints for
the cardinality computation in Equation~(\ref{eq:card-comparison}). If a set
$S$ contains up to $n$~elements, Apalache produces $O(n^2)$~constraints for
$\textit{Cardinality(S)}$.

To partially remediate the above issue in~\SpecFour{}, we introduce a
constant~$T$ for the upper bound on the number of faulty process, and further
refine Equation~(\ref{eq:card-comparison}) to:

\begin{equation}
    \textit{Cardinality}(\textit{validatorsWhoCastJustifyingVote}) \ge 2 \cdot T + 1
    \label{eq:card-comparison2}
\end{equation}

Apalache applies an optimized translation rule for
Equation~(\ref{eq:card-comparison2}). Essentially, the solver has to find $2
\cdot T + 1$ set elements to show that Equation~(\ref{eq:card-comparison2})
holds true. This gives us a linear number of constraints, instead of a
quadratic one. For example, when $T=1$ and the set may contain up to~$n$
elements, the model checker produces $3 \cdot O(n)$ constraints to check
Equation~(\ref{eq:card-comparison2}). A similar optimization is used in the
specification of Tendermint~\cite{TendermintSpec2020}.

\subsection{Quorum sets}

To further optimize the constraints over set cardinalities, we have applied the
well-known pattern of replacing cardinality tests with quorum sets. For
example, this approach is used in the specification of Paxos\footnote{%
    \tlap{} specification of Paxos:
\url{https://github.com/tlaplus/Examples/blob/master/specifications/Paxos/Paxos.tla}.}

To this end, we introduce quorum sets such as in Figure~\ref{fig:quorum-sets}.

\begin{figure}[!h]
    \includegraphics[width=\textwidth]{images/quorum-sets}
    \caption{Quorum sets}\label{fig:quorum-sets}
\end{figure}

By using quorum sets, we further replace cardinality comparisons like in
Equation~(\ref{eq:card-comparison2}) with membership tests like in
Equation~(\ref{eq:card-comparison3}):

\begin{equation}
    \textit{validatorsWhoCastJustifyingVote} \in \textit{LargeQuorums}
    \label{eq:card-comparison3}
\end{equation}


\subsection{Decomposition of chain configurations}\label{sec:decomposition}

Recall Figure~\ref{figFork} from Section~\ref{sec:spec4-indinv}, which poses
constraints on the chains and the fork points. While these constraints should
be easier for an SMT solver than general reachability properties, they
still produce a number of arithmetic constraints. On the other hand, when we do
model checking for small parameters, there is a relatively small set of
possible chain configurations.  Figure~\ref{fig:block-graphs} shows some of
these configurations for the graphs of 3 to 7 blocks.

This observation led us to the following idea. Instead of using the constraints
over blocks such as in Figure~\ref{figFork}, we introduce one instance per
chain configuration. It is thus sufficient to verify accountable safety for all
of these instances and aggregate the model checking results.

Figure~\ref{fig:indinit-c3} shows an initialization predicate for the
configuration shown in Figure~\ref{fig:five2}. This predicate replaces the
general initialization predicate that we discussed in
Section~\ref{sec:spec4-indinv}.

\begin{figure}
    \includegraphics[width=\textwidth]{images/indinit-c3}
    \caption{Initialization predicate for the
             configuration~\text{M5b}}\label{fig:indinit-c3}
\end{figure}

\subsection{Model checking experiments}

Tables~\ref{tab:spec4b-experiments} and~\ref{tab:spec4b-inductiveness}
summarizes our experiments with Apalache for various configurations. One
interesting effect of the optimizations, especially of the ones presented in
Section~\ref{sec:decomposition}, is a significant drop in the memory
consumption of the SMT solver. In our experiments, Z3 required from 700~MB to
1.5~GB\@. While this is still a factor of 20 in comparison to the Alloy
experiments in Section~\ref{sec:alloy}, this is significantly better than our
initial experiments with~\SpecTwo{} and~\SpecThree{}, which required up to
20~GB of RAM\@. Interestingly, inductiveness checks in
Table~\ref{tab:spec4b-inductiveness} take significantly longer than in the case
of~\SpecFour{}. We conjecture that this is caused by the need to check more
specialized graphs in conjunction with steps.

We notice significant variations in the running times on different
configurations. This is probably due to the effect of different runs of the Z3
SMT solve having great variety of running times, which is well-known in the
computer-aided community. To further confirm these variations, we could run
multiple experiments with `hyperfine`.

\begin{table}
    \centering
    \begin{tabular}{lllrr}
        \tbh{Configuration}
            & \tbh{Instance}
            & \tbh{Init}
            & \tbh{Memory}
            & \tbh{Time}
            \\ \toprule
        M3: Fig.~\ref{fig:three}
            & \texttt{MC\_ffg\_b1\_ffg5\_v12}
            & \texttt{Init\_C1}
            & 1.2 GB
            & 11h 31min
            \\
        M4a: Fig.~\ref{fig:four-top}
            & \texttt{MC\_ffg\_b3\_ffg5\_v12}
            & \texttt{Init\_C4}
            & XXX
            & XXX
            \\
        M4b: Fig.~\ref{fig:four-bottom}
            & \texttt{MC\_ffg\_b3\_ffg5\_v12}
            & \texttt{Init\_C2}
            & 1.3 GB
            & 1day 6h
            \\
        M5a: Fig.~\ref{fig:five1}
            & \texttt{MC\_ffg\_b3\_ffg5\_v12}
            & \texttt{Init\_C3}
            & 1.2 GB
            & 1h 53min
            \\
        M5b: Fig.~\ref{fig:five2}
            & \texttt{MC\_ffg\_b3\_ffg5\_v12}
            & \texttt{Init\_C1}
            & XXX
            & XXX
            \\
            \bottomrule
    \end{tabular}
    \caption{Checking accountable safety
             against~\SpecFourB{}}\label{tab:spec4b-experiments}
\end{table}

\begin{table}
    \centering
    \begin{tabular}{lllrr}
        \tbh{Instance}
            & \tbh{Init}
            & \tbh{Invariant}
            & \tbh{Memory}
            & \tbh{Time}
            \\ \toprule
        \texttt{MC\_ffg\_b1\_ffg5\_v12}
            & \texttt{Init}
            & \texttt{IndInv}
            & 580 MB
            & 7s
            \\
        \texttt{MC\_ffg\_b3\_ffg5\_v12}
            & \texttt{Init}
            & \texttt{IndInv}
            & 700 MB
            & 7s
            \\
        \texttt{MC\_ffg\_b1\_ffg5\_v12}
            & \texttt{Init\_C1}
            & \texttt{IndInv}
            & 1.4 GB
            & 2min 8s
            \\
        \texttt{MC\_ffg\_b3\_ffg5\_v12}
            & \texttt{Init\_C1}
            & \texttt{IndInv}
            & 1.8 GB
            & 19min 10s
            \\
        \texttt{MC\_ffg\_b3\_ffg5\_v12}
            & \texttt{Init\_C2}
            & \texttt{IndInv}
            & 1.6 GB
            & 13min 16s
            \\
        \texttt{MC\_ffg\_b3\_ffg5\_v12}
            & \texttt{Init\_C3}
            & \texttt{IndInv}
            &  1.6 GB
            & 17min 39s
            \\
        \texttt{MC\_ffg\_b3\_ffg5\_v12}
            & \texttt{Init\_C4}
            & \texttt{IndInv}
            & 1.6 GB
            & 16min 23s
            \\
            \bottomrule
    \end{tabular}
    \caption{Checking inductiveness
             for~\SpecFourB{}}\label{tab:spec4b-inductiveness}
\end{table}

