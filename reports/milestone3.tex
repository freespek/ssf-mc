\documentclass{article}
\usepackage[english]{babel}
\usepackage[utf8]{inputenc}
\usepackage{mathpartir}
\usepackage{amsthm}
\usepackage{amsmath}
\usepackage{amssymb}
\usepackage{mathtools}

\newtheorem{theorem}{Theorem}[section]
\newtheorem{lemma}[theorem]{Lemma}

\newcommand{\iteDef}[4]{
  #1 \coloneqq \left\{
\begin{array}{ll}
      #2 &; #3 \\
      #4 &; \text{otherwise}\\
\end{array} 
\right. 
}

\newcommand{\tlap}{$\textsc{TLA}^{+}$}

\newcommand{\nat}{\mathbb N_0}

\newcommand{\op}{\textsc{Op}}

\newcommand{\mop}{\textsc{mapOp}}
\newcommand{\mapg}{\textsc{mapG}}

\begin{document}

\section{Notation}
Let $f$ be any \tlap function. We use the shorthand $D_f \coloneqq \text{\bf DOMAIN } f$.
We use $\nat$ to refer to the set of all natural numbers (i.e. nonnegative integers).

\section{Proofs}

We assume the following precondition: There exists a map $\gamma: a \to \nat$, with the following property:
\[
\forall x\colon a \ .\ \forall y \in h(V(x)) \ .\ \gamma(y) < \gamma(x) 
\]

For example, $\gamma(x) = x.\text{slot}$.


% \begin{lemma}\label{lemma1}
% Let $x$ be any value of type $a$, s.t. $P(x)$ holds. Then, $h(V(x)) = \emptyset$
% \end{lemma}

% \begin{proof}
% By definition $h(V(x))$ is defined s.t. computing $\op(x)$ requires us to recursively compute $Op(v)$ for each $v \in h(V(x))$. Since for $x$ it is the case that $P(x)$ holds, and $\op$ is defined as:
% \[
% \iteDef{\op(x)}{e}{P(x)}{G(x, F(h(V(x)), \op))}
% \]
% we do not need to consider the second branch of the definition to compute $\op(x)$. Thus, we do not need to recursively compute $\op(v)$ for any $v$, and $h(V(x))$ is empty.
% \end{proof}

We define an auxiliary function $\alpha$ that assigns every function of the type $a \to Set(c)$ a value in $\nat \cup \{-\infty\}$, defined as:
\[
\alpha(f) \coloneqq \sup\left\{ \gamma(v) \mid v \in D_f \right\}
\]

\begin{lemma}\label{lemma2}
Let $f$ be a function, s.t. for any $x \in D_f$ it is the case that $f[x] = V(x)$. Then
\[
\forall f\colon (a \to Set(c)) \ .\ \alpha(f) \ge 0 \Rightarrow \alpha(b(f)) < \alpha(f)
\]
\end{lemma}

\begin{proof}
Assume $\alpha(f) \ge 0$. This is trivially equivalent to saying $D_f \ne \emptyset$.
We see that $b$ is defined as
\[
b(f)[y] \coloneqq V(y)
\]
where 
\[
D_{b(f)} \coloneqq \bigcup_{v \in D_f} h(f[v])
\]
It follows that 
\begin{align*} 
\alpha(b(f)) &= \sup\left\{ \gamma(v) \mid v \in D_{b(f)} \right\} \\
&= \sup\left\{ \gamma(v) \mid v \in \bigcup_{w \in D_f} h(f[w]) \right\} \\
&= \sup\left\{ \gamma(v) \mid v \in \bigcup_{w \in D_f} h(V(w)) \right\} \\
\end{align*}

By the precondition, we know that 
\[
\forall x\colon a \ .\ \forall y \in h(V(x)) \ .\ \gamma(y) < \gamma(x) 
\]

If we take an arbitrary $v \in \bigcup_{w \in D_f} h(V(w))$, there exists a
$w \in D_f$, s.t. $v \in h(V(w))$. The precondition then implies, that $\gamma(v) < \gamma(w)$.
Additionally, since $w \in D_f$, by the definition of $\alpha(f)$, it follows that $\gamma(w) \le \alpha(f)$.
Since all sets in question are finite, $D_{b(f)}$ is either empty, in which case $\alpha(b(f)) = -\infty$ and the lemma trivially holds, or the supremum is actually a maximum, and strict inequality is maintained when we infer
\[
\alpha(b(f)) = \sup\left\{ \gamma(v) \mid v \in \bigcup_{w \in D_f} h(V(w)) \right\} < \alpha(f)
\]
\end{proof}
As a trivial corollary to this lemma, we observe that $\alpha(b(f)) = \alpha(f)$ iff $\alpha(f) = -\infty$.


% \begin{theorem}
% \[
% \op(x) = \mop([ v \in \{x\} \mapsto V(x) ])[x]
% \]
% \end{theorem}
% \begin{proof}
% We prove this theorem in two steps below: first for $x$ where $P(x)$ holds, and then for the general case.
% \end{proof}

% Let us recall some definitions:
% \[
% \iteDef{\mop(f)}{[ x \in D_f \mapsto e ]}{\forall x \in D_f \ .\ P(x)}{\mapg(f, \mop(f'))}
% \]
% and
% \[
% \iteDef{\mapg(f, m)[x]}{m[x]}{x \in D_m}{G(x, F(h(f[x]), m))}
% \]


% \begin{lemma}
% Assume $P(x)$ holds. Then, 
% \[
% \op(x) = \mop([ v \in \{x\} \mapsto V(x) ])[x]
% \]
% \end{lemma}

% \begin{proof}
% By definition, $\op(x) = e$. If we take $f = [ v \in \{x\} \mapsto V(x) ]$, it satisfies the condition
% \[
% \forall x \in D_f \ .\ P(x)
% \]
% so $\mop(f) = [ y \in \{x\} \mapsto e ]$ and it trivially follows that
% \[
% \op(x) = e = [ x \in D_f \mapsto e ] \mop([ v \in {x} \mapsto V(x) ])[x]
% \]
% \end{proof}

\begin{theorem}
Let $f$ be a function, s.t. for any $x \in D_f$ it is the case that $f[x] = V(x)$. Then, for
$g \coloneqq \mop(f)$
\[
\forall x \in D_g \ .\ g[x] = \op(x)
\]
\end{theorem}
Recall the following definitions:
\[
\iteDef{\mop(f)}{[ x \in D_f \mapsto e ]}{\forall x \in D_f \ .\ P(x)}{\mapg(f, \mop(f'))}
\]
and
\[
% \iteDef{\mapg(f, m)[x]}{m[x]}{x \in D_m}{G(x, F(h(f[x]), m))}
\mapg(f, m)[x] \coloneqq \left\{
\begin{array}{ll}
      m[x] &; x \in D_m \\
      e &; x \notin D_m \land P(x) \\
      G(x, F(h(f[x]), m)) &; \text{otherwise}\\
\end{array} 
\right. 
\]

\begin{proof}
We prove this by using induction over $\alpha(f)$.

Base case $\alpha(f) = -\infty$: This can only be true if $D_f$ is empty. However, it is then vacuously true that 
\[
\forall x \in D_f \ .\ P(x)
\]
and it follows from the definition of $\mop(f)$ that $D_g$ is empty too. In that case, any universally quantified statement over $D_g$ vacuously holds.

General case: Let $\alpha(f) = N \ge 0$ and assume the theorem holds for any $f'$, for which $\alpha(f') < N$.
We observe that $f' \coloneqq b(f)$ satisfies all of the requirements. By definition, 
\[
b(f)[y] \coloneqq V(y)
\]
for every element in its domain, so the precondition of the theorem is met. Additionally, by Lemma \ref{lemma2}, we know that $\alpha(f') < N$, so we can use the induction assumption to conclude 
\[
\forall x \in D_{\mop(f')} \ .\ \mop(f')[x] = \op(x)
\]
% We now look to the definition of $\mop$, to see how it is expressed in terms of $f'$:
% \[
% \iteDef{\mop(f)}{[ x \in D_f \mapsto e ]}{\forall x \in D_f \ .\ P(x)}{\mapg(f, \mop(f'))}
% \]
% and
% \[
% % \iteDef{\mapg(f, m)[x]}{m[x]}{x \in D_m}{G(x, F(h(f[x]), m))}
% \mapg(f, m)[x] \coloneqq \left\{
% \begin{array}{ll}
%       m[x] &; x \in D_m \\
%       e &; P(x) \\
%       G(x, F(h(f[x]), m)) &; \text{otherwise}\\
% \end{array} 
% \right. 
% \]
If it is the case that $\forall x \in D_f \ .\ P(x)$, then $\mop(f)[x] = e$ for all domain elements, since 
$D_f = D_{\mop(f)}$, by the same reasoning we used in the case where $D_f$ was empty. Similarly, for each such $x$, $\op(x) = e$ by the definition of $\op$, since $P(x)$ holds.

Next, we select an arbitrary $x \in D_{\mop(f)}$. We look at the case split in $\mapg$:
\begin{enumerate}
\item If $x \in D_{\mop(f')}$, by the induction hypothesis, we know that $\mop(f')[x] = \op(x)$.
\item If $x \notin D_{\mop(f')}$, but $P(x)$ holds, we know $\op(x) = e$. Here, $\mapg(f, \mop(f'))$ trivially evaluates to $e$ as well.
\item Otherwise, it remains to be shown that
\[
G(x, F(h(f[x]), \mop(f'))) = \op(x)
\]
Since we know $P(x)$ does not hold for this $x$, it follows that
\[
\op(x) = G(x, F(h(V(x)), Op))
\]
so it suffices to see that 
\[
F(h(f[x]), \mop(f')) = F(h(V(x)), Op)
\]
By the property of $f$, $h(f[x]) = h(V(x))$. As $x \in D_f$, $h(f[x]) \subseteq D_{f'} \subseteq D_{\mop(f')}$, so for every element $v \in h(V(x))$ it is the case that $Op(v) = \mop(f')[v]$. Since $F$ is either a map or a filter, the equality above follows.

\end{enumerate}

\end{proof}
\end{document}

