\subsection{Meta-Rules for Translating Recursive Definitions}
\label{subsec:recrules}

So far, we have been dealing with the primitive building blocks. In order to
facilitate translation of recursive Python code to the \tlap{} fragment
supported by Apalache, we introduce a set of \tlap{}-to-\tlap{} rules, which
allow us to:

\begin{enumerate}
  \item Formulate translations from Python to \tlap{} in the
    intuitive way, potentially introducing constructs like recursion, and then

  \item Pair them with a \tlap{}-to-\tlap{} rule, ending in a supported
      fragment.

\end{enumerate}

We have to do that, as the \texttt{ssf} code extensively uses recursive
definitions.

\subsubsection{Bounded recursion rule}

Assume we are given a $\RECURSIVE$ operator $\op$. Without loss of generality,
we can take the arity to be $1$, since any operator of higher arity can be
expressed as an arity $1$ operator over tuples or records.

Further, we assume that the operator $\op$ has the following shape:

\begin{lstlisting}[language=tla,columns=fullflexible]
RECURSIVE R(_)
\* $@$type (a) => b;
R(x) ==
  IF P(x)
  THEN e
  ELSE G(x, R(next(x))
\end{lstlisting}
%

In other words, we have:

\begin{itemize}
  \item A termination condition $P$
  \item A "default" value $e$, returned if the argument satisfies the termination condition
  \item A general case operator $G$, which invokes a recursive computation of $R$ over a modified parameter given by the operator $\bb$.
\end{itemize}
%

The following needs to hold true, to ensure recursion termination: For every
$x\colon a$, there exists a finite sequence $x = v_1, \dots, v_n$
that satisfies the following conditions:

\begin{itemize}
\item $P(v_n)$ holds
\item $v_{i+1} = \bb(v_i)$ for all $1 \le i < n$
\item $P(v_i)$ does not hold for any $1 \le i < n$ (i.o.w., this is the shortest sequence with the above two properties)
\end{itemize}
%

We will attempt to express the recursive operator $\op$ with a non-recursive
``iterative'' operator $\nrop$ of arity $2$, which takes an additional
parameter: a constant $N$. The non-recursive operator will have the property
that, for any particular choice of $x$, $\nrop(x, N)$ will evaluate to $\op(x)$
if $n < N$ (i.e., if the recursion stack of $\op$ has height of at most $N$).

%
To that end, we first define:
\begin{lstlisting}[language=tla,columns=fullflexible]
\* $@$type (a, Int) => Seq(a);
Stack(x, N) ==
  LET 
    \* $@$type: (Seq(a), Int) => Seq(a);
    step(seq, i) ==
      IF i > Len(seq) \/ P(seq[1])
      THEN seq
      ELSE 
        \* Alternatively, we can append here and reverse the list at the end
        <<next(seq[1])>> \o seq 
  IN ApaFoldSeqLeft( step, <<x>>, MkSeq(N, LAMBDA i: i) )
\end{lstlisting}
%

Here, $\ApaFoldSeqLeft$ is the left-fold over sequences, that is, an operator
for which:

\begin{align*}
\ApaFoldSeqLeft(O, v, \tup{}) =& v \\
\ApaFoldSeqLeft(O, v, \tup{x_1,\dots, x_n}) =& O(O(O(v, x_1), x_2), \dots, x_n)
\end{align*}
and $\MkSeq$ is the sequence constructor, for which 
\[
\MkSeq(N, O) = \tup{O(1), \dots, O(N)}
\]
%

We can see that $\Chain(x,N)$ returns the sequence $\tup{v_n, ..., x}$ if $N$
is sufficiently large.  We can verify whether or not that is the case, by
evaluating $P(\Chain(x, N)[1])$. If it does not hold, the $N$ chosen is not
large enough, and needs to be increased. Using $\Chain$ we can define a
fold-based non-recursive operator $\op^*$, such that $\op^*(x) = \op(x)$ under
the above assumptions:

\begin{lstlisting}[language=tla,columns=fullflexible]
\* $@$type (a, Int) => b;
I(x, N) ==
  LET stack == Stack(x, N) IN
  LET step(cumul, v) == G(v, cumul) IN
  ApaFoldSeqLeft( step, e, Tail(stack) )
\end{lstlisting}
%
Then, $\op^*(x) = \nrop(x, N_0)$ for some sufficiently large specification-level constant $N_0$. Alternatively,
\begin{lstlisting}[language=tla,columns=fullflexible]
\* $@$type (a, Int) => b;
I(x, N) ==
  LET stack == Stack(x, N) IN
  LET step(cumul, v) == G(v, cumul) IN
  IF P(stack[1])
  THEN ApaFoldSeqLeft(step, e, Tail(stack))
  ELSE CHOOSE x \in {}: TRUE 
\end{lstlisting}
%
In this form, we return $\CHOOSE x \in {}: \TRUE$, which is an idiom meaning "any value" (of the correct type), in the case where the $N$ chosen was not large enough. Tools can use this idiom to detect that $\nrop(x,N)$ did not evaluate to the expected value of $\op(x)$. 

\paragraph{Example.} Consider the following operator:
\begin{lstlisting}[language=tla,columns=fullflexible]
RECURSIVE R(_)
\* $@$type (Int) => Int;
R(x) ==
  IF x <= 0
  THEN 0
  ELSE x + R(x-1)
\end{lstlisting}
where $P(x) = x \le 0$, $G(a,b) = a + b$, and $\bb(x) = x - 1$. For this operator, we know that $\op(4) = 10$. By the above definitions:
\begin{lstlisting}[language=tla,columns=fullflexible]
\* $@$type (Int, Int) => Seq(Int);
Stack(x, N) ==
  LET 
    \* $@$type: (Seq(Int), Int) => Seq(Int);
    step(seq, i) ==
      IF i > Len(seq) \/ seq[1] <= 0
      THEN seq
      ELSE <<seq[1] - 1>> \o seq
  IN ApaFoldSeqLeft( step, <<x>>, MkSeq(N, LAMBDA i: i) )
\end{lstlisting}
%
We can compute the above $\Chain$ with two different constants $N$, $2$ and $100$, and observe that $\Chain(4, 2) = \tup{2, 3, 4}$ and $\Chain(4, 100) = \tup{0, 1, 2, 3, 4}$. 
We are able to tell whether we have chosen sufficiently large values for $N$ after the fact, by evaluating $P(\Chain(x,N)[1])$. 
For our $P(x) = x \le 0$, we see $\neg P(\Chain(4, 2)[1])$, and $P(\Chain(4, 100)[1])$, so we can conclude that we should not pick $N=2$, but $N=100$ suffices. 
Of course it is relatively easy to see, in this toy example, that the recursion depth is exactly $4$, but we could use this post-evaluation in cases where the recursion depth is harder to evaluate from the specification, to determine whether we need to increase the value of $N$. \hfill $\triangleleft$

\noindent Continuing with the next operator:
\begin{lstlisting}[language=tla,columns=fullflexible]
\* $@$type (Int, Int) => Int;
I(x, N) ==
  LET stack == Stack(x, N)
  IN ApaFoldSeqLeft( +, 0, Tail(stack) )
\end{lstlisting}
%
we see that $\nrop(4, 2) = 7 \ne 10 = \op(4)$, but $\nrop(4, 100) = 10 = \op(4)$.
As expected, choosing an insufficiently large value of $N$ will give us an incorrect result, but we know how to detect whether we have chosen an appropriate $N$.

\paragraph{Optimization for associative $G$.}
In the special case where $G$ is associative, that is, $G(a, G(b, c)) = G(G(a, b), c)$ for all $a,b,c$, we can make the entire translation more optimized, and single-pass. Since $\nrop(x,N)$, for sufficiently large $N$, computes 
\[
G(v_1, G(v_2, ... (G(v_{n-2}, G(v_{n-1}, e)))))
\]
%
and $G$ is associative by assumption, then computing
\[
G(G(G(G(v_1, v_2), ...), v_{n-1}), e)
\]
gives us the same value. This computation can be done in a single pass:
\begin{lstlisting}[language=tla,columns=fullflexible]
IForAssociative(x, N) ==
  IF P(x)
  THEN e
  ELSE
    LET 
      \* $@$type: (<<a, a>>, Int) => <<a, a>>;
      step(pair, i) == \* we don't use the index `i`
        LET partial == pair[1]
             currentElem == pair[2]
        IN
          IF P(currentElem)
          THEN pair
          ELSE
            LET nextElem == next(currentElem)
            IN << G(partial, IF P(nextElem) e ELSE nextElem), nextElem >>
    IN ApaFoldSeqLeft( step, <<x, x>>, MkSeq(N, LAMBDA i: i) )[1]
\end{lstlisting}

\subsubsection{ Mutual recursion cycles}

Assume we are given a collection of $n$ operators $\op_1, \dots, \op_n$ (using the convention $\op_{n+1} = \op_1$), with types $\op_i\colon (a_i) \Rightarrow a_{i+1}$ s.t. $a_{n+1} = a_1$, in the following pattern:

\begin{lstlisting}[language=tla,columns=fullflexible]
RECURSIVE R_i(_)
\* $@$type (a_i) => a_{i+1};
R_i(x) == G_i(x, R_{i+1}(next_i(x)))
\end{lstlisting}
%
Then, we can inline any one of these operators, w.l.o.g. $\op_1$, s.t. we obtain a primitive-recursive operator:
\begin{lstlisting}[language=tla,columns=fullflexible]
RECURSIVE R(_) 
\* $@$type: (a_1) => a_1;
R(x) ==
  G_1( x, 
    G_2( next_1(x),
      G_3( next_2(next_1(x)),
        G_4( ...
          G_n( next_{n-1}(next_{n-2}(...(next_1(x)))), 
               R(next_n(next_{n-1}(...(next_1(x)))))
            )
          )
        )
      )
    )
\end{lstlisting}
for which $\op(x) = \op_1(x)$ for all $x$, and $\op$ terminates iff $\op_1$ terminates.

\subsubsection{One-to-many recursion}

Suppose we are given, for each value $x: a$, a finite set $V(x)\colon \Set(a)$ s.t. $V(x)$ is exactly the set of values $v$, for which we are required to recursively compute $\op(v)$, in order to compute $\op(x)$. 
Further, assume that there exists a potential function $\gamma$ from $a$ to nonnegative integers, with the property that, for any $x$ of type $a$ the following holds:
\[
\forall y \in V(x)\colon \gamma(y) < \gamma(x) 
\]
%
If one cannot think of a more intuitive candidate for $\gamma$, one may always take $\gamma(t)$ to be the recursion stack-depth required to compute $\op(t)$ (assuming termination). It is easy to see that such a definition satisfies the above condition.

\paragraph{Example.} In the 3SF example, for each checkpoint $x$, the set $V(x)$ would be the set of all source-checkpoints belonging to FFG votes, which could be used to justify $x$ (and those checkpoints need to be recursively justified, forming a chain all the way back to genesis). Additionally, $\gamma$ would assign each checkpoint $x$ the value $x.\text{slot}$. \hfill $\triangleleft$

\noindent Let $\op$ have the following shape:
\begin{lstlisting}[language=tla,columns=fullflexible]
RECURSIVE R(_)
\* $@$type (a) => b;
R(x) ==
  IF P(x)
  THEN e
  ELSE G(x, F(V(x), Op))
\end{lstlisting}
%
where $F(S, T(\_)) \coloneqq \{s \in S\colon T(s)\}$ or $F(S, T(\_)) \coloneqq \{T(s)\colon s \in S\}$ (i.e. a map or a filter).
%

\paragraph{Example.} In the 3SF example, for $\mathrm{is\_justified\_checkpoint}$, $P$ checks whether $x$ is the genesis checkpoint, and $e = \TRUE$. $G$ is the main computation, which determines whether or not a non-genesis checkpoint is justified, by finding quorums of validators, wherein each validator cast an FFG vote justifying $x$, but where the source checkpoint was recursively justified (i.e. belonged to $V(x)$). \hfill $\triangleleft$
\\
\\
We give a translation scheme, which reduces this more generalized form of recursion to the one given in the previous section, by defining a map-based recursive operator $\mop$, for which we will ensure
\[
\op(x) = \mop([ v \in \{x\} \mapsto V(x) ])[x]
\] 
if $\op(x)$ terminates. We define the necessary operators in Figure \ref{fig1}. Using these operator definitions, we can show the following theorem:
\begin{figure}[ht]
\begin{lstlisting}[language=tla,columns=fullflexible]
\* $@$type: (a -> Set(a)) => a -> Set(a);
next(map) ==
  LET newDomain == UNION {map[v]: v \in DOMAIN map}
  IN [ newDomainElem \in newDomain |-> V(newDomainElem) ]

\* $@$type: (a -> Set(a), a -> b) => a -> b;
Gm(currentRecursionStepMap, partialValueMap) ==
  LET domainExtension == DOMAIN currentRecursionStepMap IN
  LET 
    \* $@$type: (a) => b;
    evalOneKey(k) ==
      LET OpSubstitute(x) == partialValueMap[x] 
      IN G(k, F(currentRecursionStepMap[k], OpSubstitute))
  IN [
    x \in (domainExtension \union DOMAIN partialValueMap) |->
      IF x \in DOMAIN partialValueMap
      THEN partialValueMap[x]
      ELSE IF P(x)
           THEN e
           ELSE evalOneKey(x)
  ]

RECURSIVE Rm(_)
\* $@$type (a -> Set(a)) => a -> b;
Rm(map) ==
  IF \A x \in DOMAIN map: P(x)
  THEN [ x \in DOMAIN map |-> e ]
  ELSE Gm(map, Rm(next(map)))
\end{lstlisting}
\caption{$\mop$ and auxiliary operators \label{fig1}}
\end{figure}

\newcommand{\thmBody}{
Let $f$ be a function, s.t. for any $x \in \DOMAIN f$ it is the case that $f[x] = V(x)$. Then, for
$g \coloneqq \mop(f)$:
\[
\forall x \in \DOMAIN g \colon g[x] = \op(x)
\]
}
 
\begin{theorem}\label{thm}
\thmBody
\end{theorem}
From this theorem, the following corollary trivially follows:
\newcommand{\corollaryBody}{
For any $x\colon a$
\[
\op(x) = \mop([ v \in \{x\} \mapsto V(x) ])[x]
\]
}
 
\begin{corollary}\label{corollary}
\corollaryBody
\end{corollary}
The equivalence proofs are available in the appendix. 
The termination proof must still be made on a case-by-case basis, as it depends on $h$ and $V$.


